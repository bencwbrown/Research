\documentclass{article}


\usepackage{arxiv}

\usepackage[utf8]{inputenc} % allow utf-8 input
\usepackage[T1]{fontenc}    % use 8-bit T1 fonts
\usepackage{hyperref}       % hyperlinks
\usepackage{url}            % simple URL typesetting
\usepackage{booktabs}       % professional-quality tables
\usepackage{amsfonts}       % blackboard math symbols
\usepackage{nicefrac}       % compact symbols for 1/2, etc.
\usepackage{microtype}      % microtypography
\usepackage{lipsum}		% Can be removed after putting your text content
\usepackage{amsmath} 
\usepackage{amssymb}
\usepackage{graphicx}
\usepackage{epstopdf}
\usepackage{url}
\usepackage{setspace}
\usepackage{amsthm}
\usepackage{mathrsfs}
\usepackage{enumitem}
\usepackage{parskip}
\usepackage{IEEEtrantools}
\usepackage{mathtools}
\usepackage{tensor}
\usepackage{yfonts}
\usepackage{dsfont}
\usepackage{braket}

\usepackage{pgfplots}
\pgfplotsset{compat=1.15}

\usetikzlibrary{arrows}

\definecolor{zzttqq}{rgb}{0.6,0.2,0}
\definecolor{uuuuuu}{rgb}{0.26666666666666666,0.26666666666666666,0.26666666666666666}
\definecolor{xdxdff}{rgb}{0.49019607843137253,0.49019607843137253,1}

\newtheorem{theorem}{Theorem}[section]
\newtheorem{lemma}[theorem]{Lemma}
\newtheorem*{lemma*}{Lemma}
\newtheorem{prop}[theorem]{Proposition}
\newtheorem{corollary}[theorem]{Corollary}
\newtheorem{defn}[theorem]{Definition\rm}
\newtheorem{conjecture}[theorem]{Conjecture}
\newtheorem{remark}{\it Remark\/}
\newtheorem{example}{Example}
\newtheorem{fact}{Fact}

\newcommand{\st}{\ensuremath{:}}% such that
\newcommand{\ie}{\emph{i.e.} }
\newcommand{\eg}{\emph{e.g.} }
\newcommand{\cf}{\emph{cf.} }
\newcommand{\ra}{\rightarrow}
\newcommand{\la}{\leftarrow}
\newcommand{\lra}{\longrightarrow}
\newcommand{\lla}{\longleftarrow}
\newcommand{\lbracket}{\left(}
\newcommand{\rbracket}{\right)}
\newcommand{\half}{\frac{1}{2}}

\newcommand{\al}{\alpha}
\newcommand{\w}{\omega}
\newcommand{\W}{\Omega}
\newcommand{\m}{\mu}
\newcommand{\n}{\nu}
\newcommand{\e}{\epsilon}
\newcommand{\kahler}{K\"ahler }
\newcommand{\hyperkahler}{hyperk\"ahler }
\newcommand{\into}{\hookrightarrow}
\newcommand{\PP}{\mathbb{P}}
\newcommand{\RR}{\mathbb{R}}
\newcommand{\CC}{\mathbb{C}}
\newcommand{\QQ}{\mathbb{Q}}
\newcommand{\FF}{\mathbb{F}}
\newcommand{\ZZ}{\mathbb{Z}}
\newcommand{\NN}{\mathbb{N}}
\newcommand{\HH}{\mathbb{H}}
\newcommand{\vp}{\varphi}
\newcommand{\mcA}{\mathcal{A}}
\newcommand{\mcE}{\mathcal{E}}
\newcommand{\mcF}{\mathcal{F}}
\newcommand{\mcG}{\mathcal{G}}
\newcommand{\mcH}{\mathcal{H}}
\newcommand{\mcL}{\mathcal{L}}
\newcommand{\mcO}{\mathcal{O}}
\newcommand{\mfg}{\mathfrak{g}}
\newcommand{\mfh}{\mathfrak{h}}
\newcommand{\mft}{\mathfrak{t}}
\newcommand{\mc}[1]{\mathcal{#1}}
\newcommand{\mf}[1]{\mathfrak{#1}}

\newcommand{\pbrackets}[1]{\left( #1 \right)}
\newcommand{\bbrackets}[1]{\left[ #1 \right]}

\newcommand{\dbar}{\bar{\partial}}
\newcommand{\mrr}{\mu_{\mathbb{R}}}
\newcommand{\mcc}{\mu_{\mathbb{C}}}
\newcommand{\prr}{\phi_{\mathbb{R}}}
\newcommand{\pcc}{\phi_{\mathbb{C}}}

\DeclareMathOperator{\Lie}{Lie}
\DeclareMathOperator{\Aut}{Aut}
\DeclareMathOperator{\Tr}{Tr}
\DeclareMathOperator{\Id}{Id}
\DeclareMathOperator{\Image}{Im}
\DeclareMathOperator{\Ad}{Ad}
\DeclareMathOperator{\Diff}{Diff}
\DeclareMathOperator{\Vect}{Vect}
\DeclareMathOperator{\Sympl}{Sympl}
\DeclareMathOperator{\Span}{Span}
\DeclareMathOperator{\ind}{ind}
\DeclareMathOperator{\Td}{Td}
\DeclareMathOperator{\Ch}{Ch}
\DeclareMathOperator{\Ind}{Ind}
\DeclareMathOperator{\pt}{pt}
\DeclareMathOperator{\rk}{rk}
\DeclareMathOperator{\coker}{coker}
\DeclareMathOperator{\Pf}{Pf}
\DeclareMathOperator{\Vol}{Vol}
\DeclareMathOperator{\Res}{Res}

\DeclareMathOperator{\GL}{GL}
\DeclareMathOperator{\SO}{SO}
\DeclareMathOperator{\UU}{U}
\DeclareMathOperator{\Sp}{Sp}

\DeclareMathOperator{\HK}{HK}
\DeclareMathOperator{\stable}{st}

\newcommand\restr[2]{{% we make the whole thing an ordinary symbol
		\left.\kern-\nulldelimiterspace % automatically resize the bar with \right
		#1 % the function
		\vphantom{\big|} % pretend it's a little taller at normal size
		\right|_{#2} % this is the delimiter
}}

\title{Hypertoric Manifolds}

\date{}	% Here you can change the date presented in the paper title
%\date{} 					% Or removing it

%\author{
%  David S.~Hippocampus\thanks{Use footnote for providing further
%    information about author (webpage, alternative
%    address)---\emph{not} for acknowledging funding agencies.} \\
%  Department of Computer Science\\
%  Cranberry-Lemon University\\
%  Pittsburgh, PA 15213 \\
%  \texttt{hippo@cs.cranberry-lemon.edu} \\
%% examples of more authors
%   \And
% Elias D.~Striatum \\
%  Department of Electrical Engineering\\
%  Mount-Sheikh University\\
%  Santa Narimana, Levand \\
%  \texttt{stariate@ee.mount-sheikh.edu} \\
%% \AND
%% Coauthor \\
%% Affiliation \\
%% Address \\
%% \texttt{email} \\
%% \And
%% Coauthor \\
%% Affiliation \\
%% Address \\
%% \texttt{email} \\
%% \And
%% Coauthor \\
%% Affiliation \\
%% Address \\
%% \texttt{email} \\
%}

\begin{document}
	\maketitle
	
	\begin{abstract}
		Preprint on toric \hyperkahler manifolds.
	\end{abstract}

	\section{Toric Hyperk{\"a}hler Manifolds}
	
	\subsection{Symplectic Quotients, \cite{Hausel2002}}

	Fix the standard Euclidean bilinear form on $\CC^{n}$,
	\[
		g(z,w) = \sum_{i = 1}^{n}\lbracket \Re(z_{i})\Re(w_{i}) + \Im(z_{i})\Im(w_{i} \rbracket.
	\]
	The corresponding \kahler form is
	\[
		\w(z,w) = g(iz,w) = \sum_{i = 1}^{n} \lbracket \Re(z_{i}) \Im(w_{i}) - \Im(z_{i}) \Re(w_{i}) \rbracket.
	\]
	Let $A = [u_{1}, \ldots, u_{n}]$ be a $(d \times n)$-matrix whose $(d \times d)$-minors are relatively prime. Choose now an $n \times (n - d)$-matrix $B = [b_{1}, \ldots, b_{n}]^{T}$ that makes the following sequence exact:
	\[
		\{0\} \lra \ZZ^{n-d} \overset{B}{\lra} \ZZ^{n} \overset{A}{\lra} \ZZ^{d} \lra \{0\}.
	\]
	The choice of $B$ is equivalent to choosing a basis in $\ker(A)$.
	
	\subsection{Hyperk{\"a}hler Quotients}
	
	Let $\HH$ be the quaternions, the $4$-dimensional $\RR$-vector space with basis $\{1, i, j, k\}$ equipped with an associative algebra structure defined by
	\[
		i^{2} = j^{2} = k^{2} = ijk = -1.
	\]
	Left-multiplication by $i$ (respectively $j$ and $k$) define the following respective complex structures on $\HH$,
	\[
		I, J, K : \HH \lra \HH; \qquad I^{2} = J^{2} = K^{2} = IJK = -\Id_{\HH}.
	\]
	Equipping $\HH$ with the flat metric $g$ arising from the standard Euclidean scalar-product on $\HH \cong \RR^{4}$, with $\{1,i,j,k\}$ providing an orthonormal basis. This is called a \emph{\hyperkahler metric} since it is a \kahler metric with respect to each individual complex structure, $I$, $J$, and $K$. This also means that the so-called \emph{\kahler forms}, given by
	\[
		\w_{I}(X,Y) = g(IX, Y), \qquad \w_{J}(X,Y) = g(JX, Y), \qquad \w_{K}(KX,Y) = g(KX,Y), \qquad \text{for tangent vectors } X, Y,
	\]
	are closed differential $2$-forms.
	
	A special orthogonal transformation with respect to this metric is said to \emph{preserve the \hyperkahler structure} if it commutes with all three complex structures, $I, J$, and $K$; or equivalently, it preserves the K{\"a}hler forms, $\w_{I}$, $\w_{J}$, and $\w_{K}$. The group of such transformations, the \emph{unitary symplectic group} $\Sp(1)$, is generated by the right-multiplication action by the unit quaternions.
	
	A maximal abelian subgroup $T_{\RR}^{1} \cong \UU(1) \subset \Sp(1)$ is then specified by a choice of unit quaternion, and we break the $I, J, K$ symmetry by choosing a maximal torus, generated by right-multiplication by the unit quaternion $i$. Hence $\UU(1)$ acts on $\HH$ from the right by sending
	\[
		\xi \mapsto \xi\exp(ti), \qquad \exp(ti) \in \UU(1) \subset \RR \oplus \RR i \cong \CC.
	\]
	The moment map for this action $\mu_{1} : \HH \ra \RR$ with respect to the symplectic form $\w_{1}$ is then given by
	\[
		\mu_{1}(x + yi + uj + vk) = \mu_{1}\left((x+yi) + (v - ui)k\right) = \half\left(x^{2} + y^{2} - u^{2} - v^{2}\right).
	\]
	
	\subsection{}
	
	\begin{prop}[\cite{Proudfoot2004}]
		Suppose that $\alpha$ and $(\alpha,0)$ are regular values for $\mu$ and $\mu_{HK}$, respectively. Then the cotangent bundle $T^{\ast}X$ is isomorphic to an open subset of $M$, and is dense if it is non-empty.
	\end{prop}

	\begin{proof}
		Let $Y = \Set{ (z,w) \in \mu_{\CC}^{-1}(0)^{\stable} | z \in (\CC^{n})^{\stable} }$, where $z$ is semi-stable with respect to $\alpha$ for the $G_{\CC}$-action on $\CC^{n}$, so that we have $X \cong (\CC^{n})^{\stable} / G_{\CC}$. Let $[z] \in X$ be the representative of $z \in (\CC^{n})^{\stable}$. The tangent space $T_{[z]}X$ is equal to the quotient of $T_{z}\CC^{n}$ by the tangent space to the $G_{\CC}$-orbit through $z$,
		\[
			T_{[z]}X = T_{z}\CC^{n} / T_{[z]}(G_{\CC} \cdot z).
		\]
		Therefore,
		\[
			T_{[z]}^{\ast}X \cong \Set{ w \in T_{z}^{\ast}\CC^{n} | w(\hat{v}_{z}) = 0, \text{ for all } v \in \mfg_{\CC} }= \Set{ w \in (\CC^{n})^{\ast} | \mu_{\CC}(z,w) = 0 }.  
		\]
		Then, by letting $[z] \in X$ vary, we have
		\[
			T^{\ast}X \cong \Set{ (z,w) | z \in (\CC^{n})^{\stable} \text{ and } \mu_{\CC}(z,w) = 0 }/G_{\CC} = Y/G_{\CC}.
		\]
		As each $z$-coordinate in $Y$ is semi-stable, $Y$ is an open subset of $\mu_{\CC}^{-1}(0)$, and is dense if non-empty.
	\end{proof}
	
	
	
	
	
	
	
	
	

	\section{Cotangent Spaces to Extended Core Components}
	
	Let $M_{\lambda} = \pbrackets{\mu_{\RR}^{-1}(\lambda) \cap \mu_{\CC}^{-1}(0)}/K$ be a toric \hyperkahler manifold. Define
	\begin{equation*}
		\CC_{A} := \Set{ (z_{i}, w_{i}) \in \CC^{2n}\ | \ w_{i} = 0\, \text{if } i \in A, \text{ and } z_{i} = 0\, \text{if } i \not\in A } \cong \CC^{n} \subset \HH^{n}.
	\end{equation*}
	
	\begin{lemma}[\cite{Konno2002}]\label{cotangent:1}
		Let $M_{\lambda}$ be a toric \hyperkahler manifold. If $\mcE_{A}$ is non-empty, then its holomorphic cotangent bundle $T^{\ast}\mcE_{A}$ is contained in $M_{\lambda}$ as an open subset.
	\end{lemma}
	
	Fix a subset $A \subset \Set{1, \ldots, n}$, and define
	\begin{equation*}
		(x_{i}^{(A)}, y_{i}^{(A)}) :=
		\begin{cases}
			(z_{i}, w_{i}), \quad &\text{if } i \in A, \\
			(w_{i}, -z_{i}), \quad &\text{if } i \not\in A.
		\end{cases}
	\end{equation*}

	Then $x^{(A)} = (x_{1}^{(A)}, \ldots, x_{n}^{(A)})$ is a point in the vector space $\CC_{A}^{n}$, and $y^{(A)} = (y_{1}^{(A)}, \ldots, y_{n}^{(A)})$ is a point in the dual space $\pbrackets{\CC_{A}^{n}}^{\ast}$. That is, we identify the cotangent bundle $T^{\ast}\CC_{A}^{n}$ with $\HH^{n}$ as above.
	
	\subsection{K{\"a}hler Quotients}
	
	The K{\"a}hler quotient $X = \mu^{-1}(0) / N$ can be identified with the quotient of an open subset of $\CC^{n}$ by the complexified torus $N^{\CC}$ as follows: every orbit in $\CC^{n}$ of $T_{\CC}^{n}$ is of the form
	\[
		\CC_{A}^{n} = \Set{ (z_{1}, \ldots, z_{n}) | z_{i} = 0 \text{ if } i \in A},
	\]
	for some subset $A \subset \{1,\ldots, n\}$. If $F$ is a face of $\Delta$ of codimension $r$, then $F$ is defined by the intersection of $r$ hyperplanes $\cap_{j = 1}^{r}H_{i_{j}}$. 
	
	
	\section{Symplectic Cutting}
	
	\subsection{Compactifying the Extended Core}
	
	Let $S^{1}$ act on $M$ by rotating the cotangent fibres, that is, for $\tau \in S^{1}$,
	\[
		\tau \cdot [z; w] = [z; \tau w].
	\]
	This $S^{1}$-action is Hamiltonian, with moment map
	\[
		\Phi : M \lra (\RR)^{\ast}; \qquad [z:w] \longmapsto \frac{1}{2} \|w\|^{2}.
	\]
	Let $S_{A}^{1}$ denote the residual $S^{1}$-action on $M$ restricted to the extended core component 
	\[
		\mcE_{A} = \Set{ [z_{1}: \ldots z_{n}; w_{1}, \ldots, w_{n}] | w_{0} = 0 \text{ if } i \in A, \text{ and } z_{i} = 0 \text{ if } i \not\in A }.
	\]
	Now the \emph{global }$S^{1}$-action does not act on the cotangent fibres of $M$ as a subtorus of $T^{n}$, but it does when \emph{restricted} to each component of the extended core, $\mcE_{A}$. Indeed,
	\[
		\tau \cdot [z; w] = [z; \tau w] = [z_{1} : \ldots : z_{n} ; \tau w_{1} : \ldots : \tau w_{n} ] = [\tau_{1 }z_{1} : \ldots : \tau_{n}z_{n} ; \tau_{1}^{-1}w_{1} : \ldots : \tau_{n}^{-1} w_{n} ],
	\]
	where
	\[
		\tau_{i} :=
		\begin{cases}
			\tau^{-1}, \qquad &\text{if } i \in A, \\
			1, \qquad &\text{if } i \not\in A,
		\end{cases}
	\]
	which shows that the $S^{1}$-action restricted to each individual $\mcE_{A}$ acts as a subtorus of the original torus $T^{n}$.
	
	Denote by $S_{A}^{1}$ the image of $S^{1}$ in $T^{n}$ when considered as a subtorus restricted to each individual $\mcE_{A}$, and let $\jmath_{A} : S^{1} \hookrightarrow T^{n}$ be the respective inclusion homomorphism, so we have $S_{A}^{1} := \jmath_{A}(S^{1}) \lhd T^{n}$.
	
	On the Lie algebra level, we have that
	\[
		(\jmath_{A})_{\ast} : \Lie(S_{A}^{1})  \lra \mft^{n}; \qquad \xi \longmapsto ( \xi_{1}, \ldots, \xi_{n}),
	\]
	where we analogously define
	\[
		\xi_{i} :=
		\begin{cases}
			-1, \qquad &\text{if } i \in A, \\
			0, \qquad &\text{if } i \not\in A.
		\end{cases}
	\]
	
	Since $S_{A}^{1}$ acts as the subtorus $\jmath_{A}(S^{1})$ of $T^{n}$ on each $\mcE_{A}$, the moment map $\Phi_{A} := \restr{\Phi}{\mcE_{A}}$ for this action is given by composing $\mu_{\RR}$ with the dual of the inclusion $(\jmath_{A})_{\ast}$, so
	\[
	\begin{split}
		\Phi_{A}[z,w] = \left( \jmath_{A}^{\ast} \circ \mu_{\RR} \right)[z;w] &= \jmath_{A}^{\ast} \left( \frac{1}{2} \sum_{i = 1}^{n} \left( |z_{i}|^{2} - |w_{i}|^{2} \right) e^{i} \right) \\
		&= -\frac{1}{2} \sum_{i \in A} |z_{i}|^{2} \jmath_{A}^{\ast}(e^{i}) \\
		&= \frac{1}{2} \sum_{i \not\in A} |w_{i}|^{2} \jmath_{A}^{\ast}(e^{i}) \\
		&= \left\langle \mu_{\RR}[z;w],\, \xi_{A} \right\rangle \\
		&= \mu_{\RR}^{A}[z;w],
	\end{split}
	\]
	where $\xi_{A} = -\sum_{i \in A} \xi_{i}$, and $\mu_{\RR}^{A}[z;w]$ is the component of $\mu_{\RR}[z;w]$ in the $\xi_{A}$-direction.
	
	\subsection{Moment Polyptychs}
	
	
	
	
	
	
	\bibliographystyle{unsrt}  
	\bibliography{hypertoric-localisation}  %%% Remove comment to use the external .bib file (using bibtex).
	%%% and comment out the ``thebibliography'' section.
	
	%%% Comment out this section when you \bibliography{references} is enabled.
	
\end{document}
