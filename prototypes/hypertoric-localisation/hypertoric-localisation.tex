\documentclass[11pt]{amsart}

\usepackage[parfill]{parskip}    
\usepackage{graphicx}
\usepackage{amssymb, amsfonts,mathabx}
\usepackage{epstopdf}

\usepackage{amsmath, amsthm}
\usepackage[all,cmtip]{xy}

\usepackage{tikz}
\usepackage{tikz-cd}
\usetikzlibrary{matrix,arrows,patterns,calc,through,backgrounds,fadings, decorations}
\usetikzlibrary{decorations.pathreplacing}

\newtheorem{theorem}{Theorem}[section]
\newtheorem{lemma}[theorem]{Lemma}
\newtheorem*{lemma*}{Lemma}
\newtheorem{proposition}[theorem]{Proposition}
\newtheorem{corollary}[theorem]{Corollary}
\newtheorem{definition}[theorem]{Definition\rm}
\newtheorem{conjecture}[theorem]{Conjecture}
\newtheorem{remark}{\it Remark\/}
\newtheorem{example}{Example}

\newcommand{\st}{\ensuremath{:}}% such that
\newcommand{\ie}{\emph{i.e.} }
\newcommand{\eg}{\emph{e.g.} }
\newcommand{\cf}{\emph{cf.} }
\newcommand{\ra}{\rightarrow}
\newcommand{\la}{\leftarrow}
\newcommand{\lra}{\longleftarrow}
\newcommand{\lla}{\longleftarrow}
\newcommand{\lbracket}{\left(}
\newcommand{\rbracket}{\right)}


\newcommand{\al}{\alpha}
\newcommand{\w}{\omega}
\newcommand{\m}{\mu}
\newcommand{\n}{\nu}
\newcommand{\e}{\epsilon}
\newcommand{\K}{K\"ahler }
\newcommand{\HK}{hyperk\"ahler }
\newcommand{\into}{\hookrightarrow}
\newcommand{\PP}{\mathbb{P}}
\newcommand{\RR}{\mathbb{R}}
\newcommand{\CC}{\mathbb{C}}
\newcommand{\QQ}{\mathbb{Q}}
\newcommand{\FF}{\mathbb{F}}
\newcommand{\ZZ}{\mathbb{Z}}
\newcommand{\NN}{\mathbb{N}}
\newcommand{\HH}{\mathbb{H}}
\newcommand{\vp}{\varphi}
\newcommand{\mcE}{\mathcal{E}}
\newcommand{\mcF}{\mathcal{F}}
\newcommand{\mcG}{\mathcal{G}}
\newcommand{\mcH}{\mathcal{H}}
\newcommand{\mcL}{\mathcal{L}}
\newcommand{\mcO}{\mathcal{O}}
\newcommand{\mfg}{\mathfrak{g}}
\newcommand{\mfh}{\mathfrak{h}}
\newcommand{\mft}{\mathfrak{t}}
\newcommand{\mc}[1]{\mathcal{#1}}
\newcommand{\mf}[1]{\mathfrak{#1}}

\newcommand{\dbar}{\bar{\partial}}
\newcommand{\mrr}{\mu_{\mathbb{R}}}
\newcommand{\mcc}{\mu_{\mathbb{C}}}
\newcommand{\prr}{\phi_{\mathbb{R}}}
\newcommand{\pcc}{\phi_{\mathbb{C}}}

\DeclareMathOperator{\Lie}{\text{Lie}}
\DeclareMathOperator{\Aut}{Aut}
\DeclareMathOperator{\Tr}{Tr}
\DeclareMathOperator{\Image}{Im}
\DeclareMathOperator{\Ad}{Ad}
\DeclareMathOperator{\Diff}{Diff}
\DeclareMathOperator{\Vect}{Vect}
\DeclareMathOperator{\Sympl}{Sympl}
\DeclareMathOperator{\Span}{Span}
\DeclareMathOperator{\ind}{ind}
\DeclareMathOperator{\Td}{Td}
\DeclareMathOperator{\Ch}{Ch}

\usepackage{hyperref}

\title{Hypertoric Manifolds and Equivariant Localisation}
\author{Benjamin C. W. Brown}
\address[Benjamin Brown]{School of Mathematics and Maxwell Institute, The University of Edinburgh, Peter Guthrie Tait Road, Edinburgh EH9 3FD, United Kingdom}
\email{B.Brown@ed.ac.uk}
\date{\today}  
\thanks{}                                         
\begin{document}

\maketitle
 

\section{Index Theory} \label{sec:index-theory}

\subsection{Non-Equivariant Index Formula} \label{subsec:non-equivariant-index-formula}

For a holomorphic vector bundle $\mcL$ over a complex $n$-dimensional variety $M$, the \emph{index} $\ind(\dbar, \mcL)$w is defined as

\begin{equation*}
	\ind(\dbar, \mcL) := \sum\limits_{k = 0}^{n}(-1)^{k}\dim H^{k}(M; \mcL).
\end{equation*}

Viewing the index $\ind(\dbar, \mcL)$ as the Euler characteristic $\chi(M, \mcL)$ of the vector bundle $\mcL$, we can apply the Atiyah-Singer index theorem, which we state below, to express the index as an integral over $M$ of the product of the Todd class $\Td(TM)$ of the tangent bundle $TM \ra M$ over $M$, and the Chern character $\Ch(\mcL) := \exp(c_{1}(\mcL))$ of $\mcL$, where $c_{1}(\mcL)$ is the first Chern class of $\mcL$.

\begin{theorem}[Atiyah-Singer Index Theorem, \cite{MAIS1968}]
	\label{thm:atiyah-singer-index-theorem}
	Let $M$ be a compact complex manifold, $\mcL$ a holomorphic vector bundle over $M$. Let

	\begin{equation*}
		\Td(TM) = \prod \frac{x_{i}}{1 - e^{-x_{i}}}
	\end{equation*}

	be the Todd class of the complex vector bundle $TM \ra M$, where the $x_{i}$ are the Chern roots of $TM$. Then the Euler characteristic $\chi(M, \mcL)$ of the sheaf of germs of holomorphic sections of $\mcL$ is given by
	
	\begin{equation*}
		\chi(M, \mcL) = \int_{M} \Td(M)\cdot \Ch(\mcL).
	\end{equation*}
\end{theorem}

\subsection*{Example}

Let $M = \CC\PP^{1}$ and let $\mcL$ be the line bundle $\mcO(k)$ for some positive integer $k$. If $\langle \xi \rangle = H^{2}(M; \ZZ)$, \ie $\xi$ is the generator of $H^{2}(\CC\PP^{1}; \ZZ)$, then $c_{1}(\mcL) = k\xi$, and thus the Chern character of $\mcL$ is
\begin{equation*}
	\Ch(\mcL) = e^{c_{1}(\mcL)} = \sum_{j=0}^{\infty} (k\xi)^{j} = 1 + k\xi
\end{equation*}
(the higher powers of $\xi$ vanish since $\dim_{\CC} M = 1)$.
	
For $n$-dimensional complex projective space $\CC\PP^{n}$, both the total Chern class 

\begin{equation*}
	c(\CC\PP^{n}) := c(T\CC\PP^{n}) := 1 + c_{1} + c_{2} + c_{3} + \ldots,
\end{equation*}

and the Todd class $\Td(T\CC\PP^{n})$ for the tangent bundle $T\CC\PP^{n} \ra \CC\PP^{n}$, can be calculated using the exact Euler sequence,
\begin{figure}[h!]
	\begin{tikzcd}
		\{0\} \arrow[r] & \mathcal{O} \arrow[r] & \mathcal{O}(1)^{\oplus(n+1)} \arrow[r] & T\mathbb{CP}^{n} \arrow[r] & \{0\},
	\end{tikzcd}
\end{figure}
along with the multiplicativity of the total Chern class and the Todd class,
\begin{equation*}
	c(\mcF \oplus \mcG) = c(\mcF)\cdot c(\mcG), \qquad \Td(\mcF \oplus \mcG) = \Td(\mcF) \cdot \Td(\mcG),
\end{equation*}

which yields

\begin{equation*}
	c(\CC\PP^{n}) = c(T\CC\PP^{n} \oplus \mcO) = c(\mcO(1)^{\oplus(n+1)}) = (1 + \xi)^{n+1},
\end{equation*}
and
\begin{equation*}
	\Td(T\CC\PP^{n}) = \Td(T\CC\PP^{n} \oplus \mcO ) = \Td(\mcO(1)^{\oplus (n+1)}) = \Td\lbracket\mcO(1)\rbracket^{n+1} = \lbracket \frac{\xi}{1 - e^{-\xi}} \rbracket^{n+1}.
\end{equation*}
	
This expression can be expanded as a formal power series which, for $n = 1$ in our example with the complex projective line $\CC\PP^{1}$, gets us

\begin{equation*}
	c(\CC\PP^{1}) = (1 + \xi)^{2} = 1 + 2\xi, \qquad \Td(T\CC\PP^{1}) = 1 + \tfrac{1}{2}c_{1}(T\CC\PP^{1}) = 1 + \xi.
\end{equation*}

Finally, applying the Atiyah-Singer index theorem \ref{thm:atiyah-singer-index-theorem}, we have

\begin{equation*}
	\chi(\CC\PP^{1}, \mcL) = \int_{\CC\PP^{1}} \Td(\CC\PP^{1}) \cdot \Ch(\mcL) = \int_{\CC\PP^{1}} (1 + \xi) \cdot (1 + k\xi ) = \int_{\CC\PP^{1}} 1 + (k+1)\xi = k + 1.
\end{equation*}

\subsection*{Example}

Now we let $M = \CC\PP^{2}$, and let $\mcL = \mcO(k)$ and $\langle \xi \rangle = H^{2}(M, \ZZ)$ again as above. Now we have

\begin{equation*}
	c(\mcL) = e^{c_{1}(\mcL)} = 1 + k\xi + k^{2}\xi^{2},
\end{equation*}

and 
\begin{equation*}
	\begin{split}
		c(T\CC\PP^{2}) &= 1 + c_{1} + c_{2} = (1 + \xi)^{3} = 1 + 3\xi + 3\xi^{2}, \\ 
		\Td(T\CC\PP^{2}) &= 1 + \frac{c_{1}}{2} + \frac{c_{1}^{2} + c_{2}}{12} = 1 + \frac{3}{2}\xi + \frac{9\xi^{2} + 3\xi^{2}}{12} = 1 + \frac{3}{2}\xi + \xi^{2}.
	\end{split}
\end{equation*}

Hence by the Atiyah-Bott index theorem \ref{thm:atiyah-singer-index-theorem},

\begin{equation*}
	\begin{split}
		\chi(M, \mcL) &= \int_{M} \Td(TM) \cdot \Ch(\mcL) = \int_{M} \lbracket 1 + \tfrac{3}{2}\xi + \xi^{2} \rbracket \cdot \lbracket 1 + k\xi + k^{2}\xi^{2} \rbracket \\
		&= \int_{M} (k^{2} + \tfrac{3}{2}k + 1)\xi^{2} + O(\xi) = k^{2} + \tfrac{3}{2}k + 1.
	\end{split}
\end{equation*}

\subsection*{Example}

Let $M = \CC\PP^{3}$, and let $\mcL$, $\xi$, etc. be as above. Then

\begin{equation*}
	\begin{split}
		\Ch(\mcL) &= 1 + k\xi + (k\xi)^{2} + (k\xi)^{3}, \\
		c(TM) &= (1 + \xi)^{4} = 1 + 4\xi + 6\xi^{2} + 4\xi^{3}, \\
		\Td(TM) &= 1 + \frac{c_{1}}{2} + \frac{c_{1}^{2} + c_{2}}{12} + \frac{c_{1}c_{2}}{24} = 1 + 2\xi + \frac{11}{6}\xi^{2} + \xi^{3}.
	\end{split}
\end{equation*}

Then by the Atiyah-Bott Index theorem \ref{thm:atiyah-singer-index-theorem},

\begin{equation*}
	\begin{split}
		\chi(M, \mcL) &= \int_{M} \Td(TM) \cdot \Ch(\mcL) = \int_{M} \lbracket 1 + 2\xi + \frac{11}{6}\xi^{2} + \xi^{3} \rbracket \cdot \lbracket 1 + k\xi + k^{2}\xi^{2} + k^{3}\xi^{3} \rbracket \\
		&= \int_{M} \lbracket k^{3} + 2k^{2} + \frac{11}{6}k + 1 \rbracket\xi^{3} + O(\xi^{2}) = 
	\end{split}
\end{equation*}

\subsection{Equivariant Index Theorems}

\subsubsection{Equivariant Characteristic Classes}

%==============================================================================

\section{Compactifying the Hypertoric Variety via Symplectic Cutting}

\subsection{Set-Up}

We will use the $S^{1}$-action to symplectically cut the toric \HK manifold $M$ in order to compactify it as follows: consider the product $M \times \CC$, where now $S^{1}$ acts on $M \times \CC$ as
$$
e^{i\theta} \cdot \big( [z,w], \xi   \big) = \big( [z,e^{i\theta}], e^{i\theta}\xi\big),
$$
which is hamiltonian with moment map
\begin{equation*}
	\begin{split}
		\mu_{\text{cut}}: M \times \CC &\longrightarrow \RR_{\geq 0}, \\
		\mu_{\text{cut}}\big( [z,w], \xi  \big) &= \Phi[z,w] + \tfrac{1}{2}|\xi|^{2} - \e,
	\end{split}
\end{equation*}
for some $\e \in \RR_{\geq 0}$. Then we have
\begin{equation*}
	\begin{split}
		\mu_{\text{cut}}^{-1}(0) &= \big\{ ([z,w],\xi) \in M \times \CC \st \|w\|^{2} + |\xi|^{2} = 2\e    \big\} \\
		&= \big\{ [z,w] \in M \st \|w\|^{2} = 2\e    \big\} \bigsqcup \big\{ ([z,w],\xi) \in M \times \CC \st |\xi| = \pm\sqrt{2\e - \|w\|^{2}} \big\} \\
		&= \big\{ [z,w] \in M \st \|w\|^{2} = 2\e    \big\} \bigsqcup \big\{ ([z,w],\xi) \in M \times \CC \st \xi = e^{i\arg(\xi)}\sqrt{2\e - \|w\|^{2}}    \big\} \\
		&= \Phi^{-1}(\e) \bigsqcup (M \times S^{1}) \\
		&=: \Sigma_{1} \bigsqcup \Sigma_{2},
	\end{split}
\end{equation*}
where $\Sigma_{1}$ is just the level-set of $\Phi$ at the level $\e$ in $M$, and $\Sigma_{2} = M \times S^{1}$ is exhibited as a trivial $S^{1}$-bundle over $\Sigma_{2}$, using the globally defined section
\begin{equation*}
	M \rightarrow M \times S^{1}, \qquad [z,w] \longmapsto \big( [z,w], e^{i\theta}\sqrt{2\e - \|w\|^{2}}\big), \qquad e^{i\theta} \in S^{1}.
\end{equation*}
Finally, taking the quotient of $\m_{\text{cut}}^{-1}(0)$ by the $S^{1}$-action, we obtain the symplectic cut
\begin{equation*}
	M_{\leq \e} := \m_{\text{cut}}^{-1}(0)/S^{1} = \Sigma_{1}/S^{1} \bigsqcup \Sigma_{2}/S^{1},
\end{equation*}
where $\Sigma_{1}/S^{1} = \Phi^{-1}(\e)/S^{1}$ is just the symplectic reduction, and where $\Sigma_{2}/S^{1}$ is diffeomorphic to $M$ for $\|w\|^{2} < 2\e$, which we denote by $M_{<\epsilon}$.

\subsection{Restriction to the Extended Core Component, $\mc{E}_{A}$}

Since the residual circle $S^{1}$-action acts as a subgroup of the original torus $T^{n}$ when restricted to each component $\mc{E}_{A}$ of the extended core $\mc{E}$, we can described combinatorially the resulting configuration of the hyperplane arrangement in $(\RR^{d})^{\ast}$ from taking the cut. For each component, let $j_{A}: \mf{s}^{1} \rightarrow \RR^{n}$ be the derivative of the inclusion of $S^{1}$ into $T^{n}$ on the Lie algebra level, that is
\begin{equation*}
	j_{A}( \xi  ) = (\xi_{1},\ldots, \xi_{n}),\qquad \text{where } \xi_{i} = 
	\begin{cases}
		-1\qquad&\text{if } i\in A,\\
		0\qquad &\text{if } i \not\in A,
	\end{cases}
\end{equation*}
so that its image in $\RR^{n}$ generates a circle subgroup $S^{1}$ in $T^{n}$ that depends on each component $\mc{E}_{A}$. Then the moment map for this restriction for the $S^{1}$-action is
\begin{equation*}
	\Phi[z,w] = j_{A}^{\ast} \circ \mrr[z,w] = \bigg\langle \mrr(z,w), \sum_{i\in A}\xi_{i} u_{i} \bigg\rangle,
\end{equation*}
and so from our above discussion of how we constructed the symplectic cut, the image in $(\RR^{d})^{\ast}$ of the symplectic quotient $\Phi^{-1}(\e)/S^{1}$ is
\begin{equation*}
	\prr(\Phi^{-1}(\e)) = \bigg\{ y \in \Delta_{A} \st \bigg\langle y, \sum_{i\in A}\xi_{i}u_{i}\bigg\rangle + \e = 0 \bigg\} =: H_{A}
\end{equation*}
which introduces an inward-pointing half-space
\begin{equation*}
	F_{A} := \bigg\{ y \in \Delta_{A} \st \bigg\langle y, \sum_{i\in A}\hbar_{i}u_{i}\bigg\rangle + \e \geq 0 \bigg\}
\end{equation*}
such that the image of the extended core component $\mc{E}_{A}$ after being compactified is the original convex polytope $\Delta_{A}$ intersected with $H_{A}$. One can also see clearly that the symplectic quotient $\Phi^{-1}(\e)/S^{1}$ has the restricted $S^{1}$-action as its stabiliser subgroup since, by definition of $H_{A}$, the moment map $\Phi|_{\mc{E}_{A}}$ equals the hyperplane $H_{A}$, \ie $\Phi|_{\mc{E}_{A}}$ is constant along $\Phi^{-1}(\epsilon)/S^{1}$.

\begin{remark}
	If we had used instead the following action for $S^{1}$
	\begin{equation*}
		e^{i\theta}\cdot \big( [z,w], \xi \big) = \big( [z,e^{i\theta}w], e^{-i\theta}\xi \big)
	\end{equation*}
	with respective moment map
	\begin{equation*}
		\m_{\text{cut}} ([z,w],\xi) = \frac{1}{2}\|w\|^{2} - \frac{1}{2}|\xi|^{2} - \e,
	\end{equation*}
	and taken the cut, then the resulting then we would obtain the other ``discarded half'' $\mf{M}_{>\e}$ of the hypertoric manifold $\mf{M}$ along with the symplectic quotient $\Phi^{-1}(\e)/S^{1}$ with the opposite orientation:
	\begin{equation*}
		\mf{M}_{\geq \e} = \mf{M}_{> \e} \bigsqcup \Big(-(\Phi^{-1}(\e)/S^{1})\Big).
	\end{equation*}
	The component $M_{\e}$ is non-compact however, so we focus on $M_{<\e}$.
\end{remark}



























\providecommand{\bysame}{\leavevmode\hbox to3em{\hrulefill}\thinspace}
\providecommand{\href}[2]{#2}

\bibliographystyle{unsrt}
\bibliography{index-theorems}

\end{document}  










