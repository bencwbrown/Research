\documentclass{article}


\usepackage{arxiv}
%\usepackage{ebgaramond}
\usepackage{garamondx}
% \usepackage{CormorantGaramond}

\usepackage[utf8]{inputenc} % allow utf-8 input
\usepackage[T1]{fontenc}    % use 8-bit T1 fonts
\usepackage{hyperref}       % hyperlinks
\usepackage{url}            % simple URL typesetting
\usepackage{booktabs}       % professional-quality tables
\usepackage{amsfonts}       % blackboard math symbols
\usepackage{nicefrac}       % compact symbols for 1/2, etc.
\usepackage{microtype}      % microtypography
% \usepackage{lipsum}		% Can be removed after putting your text content
\usepackage{amsmath} 
\usepackage{amssymb}
\usepackage{graphicx}
\usepackage{epstopdf}
\usepackage{url}
\usepackage{setspace}
\usepackage{amsthm}
\usepackage{mathrsfs}
\usepackage{enumitem}
\usepackage{parskip}
\usepackage{IEEEtrantools}
\usepackage{mathtools}
\usepackage{tensor}
\usepackage{yfonts}
\usepackage{dsfont}

%%%%%%%%%%%%%%%%%%% Custom packages
\usepackage{braket}
\usepackage{todo}
\usepackage{xargs}                      % Use more than one optional parameter in a new commands
\usepackage{tikz}
\usepackage{tikz-cd}

% xy (for commutative diagrams)

%\usepackage[all]{xy}
%\input xy
%\xyoption{all}


%%%% Draft  package and commands %%%%
\usepackage{draft}
%\draftfalse
\newnote{nb}{red}

%%%%%%%%%%%%%%%%%%

\usetikzlibrary{arrows}

\newtheorem{theorem}{Theorem}[section]
\newtheorem{lemma}[theorem]{Lemma}
\newtheorem*{lemma*}{Lemma}
\newtheorem{prop}[theorem]{Proposition}
\newtheorem{corollary}[theorem]{Corollary}
\newtheorem{defn}[theorem]{Definition\rm}
\newtheorem{conjecture}[theorem]{Conjecture}
\newtheorem{remark}{\it Remark\/}
\newtheorem{example}{Example}
\newtheorem{fact}{Fact}

\newcommand{\st}{\ensuremath{:}}% such that
\newcommand{\ie}{\emph{i.e.} }
\newcommand{\eg}{\emph{e.g.} }
\newcommand{\cf}{\emph{cf.} }
\newcommand{\ra}{\rightarrow}
\newcommand{\la}{\leftarrow}
\newcommand{\lra}{\longrightarrow}
\newcommand{\lla}{\longleftarrow}
\newcommand{\lbracket}{\left(}
\newcommand{\rbracket}{\right)}

\newcommand{\al}{\alpha}
\newcommand{\w}{\omega}
\newcommand{\W}{\Omega}
\newcommand{\m}{\mu}
\newcommand{\n}{\nu}
\newcommand{\e}{\epsilon}
\newcommand{\K}{K\"ahler }
\newcommand{\into}{\hookrightarrow}
\newcommand{\PP}{\mathbb{P}}
\newcommand{\RR}{\mathbb{R}}
\newcommand{\CC}{\mathbb{C}}
\newcommand{\QQ}{\mathbb{Q}}
\newcommand{\FF}{\mathbb{F}}
\newcommand{\ZZ}{\mathbb{Z}}
\newcommand{\NN}{\mathbb{N}}
\newcommand{\HH}{\mathbb{H}}
\newcommand{\vp}{\varphi}
\newcommand{\mcA}{\mathcal{A}}
\newcommand{\mcC}{\mathcal{C}}
\newcommand{\mcE}{\mathcal{E}}
\newcommand{\mcF}{\mathcal{F}}
\newcommand{\mcG}{\mathcal{G}}
\newcommand{\mcH}{\mathcal{H}}
\newcommand{\mcL}{\mathcal{L}}
\newcommand{\mcO}{\mathcal{O}}
\newcommand{\mcR}{\mathcal{R}}
\newcommand{\mfg}{\mathfrak{g}}
\newcommand{\mfh}{\mathfrak{h}}
\newcommand{\mfk}{\mathfrak{k}}
\newcommand{\mft}{\mathfrak{t}}
\newcommand{\mc}[1]{\mathcal{#1}}
\newcommand{\mf}[1]{\mathfrak{#1}}
\newcommand{\krr}{k_{\RR}}
\newcommand{\kcc}{k_{\CC}}

\newcommand{\sslash}{\mathbin{/\mkern-6mu/}}
\newcommand{\sssslash}{\mathbin{/\mkern-6mu/\mkern-6mu/\mkern-6mu/}}

\newcommand{\pbrackets}[1]{\left( #1 \right)}
\newcommand{\bbrackets}[1]{\left[ #1 \right]}
\newcommand{\norm}[1]{|#1|^{2}}
\newcommand{\tuple}[2]{(#1, \ldots, #2)}
\newcommand{\half}{\frac{1}{2}}
\newcommand{\thalf}{\tfrac{1}{2}}

\newcommand{\dbar}{\bar{\partial}}
\newcommand{\bmu}{\bar{\mu}}
\newcommand{\mrr}{\mu_{\mathbb{R}}}
\newcommand{\mcc}{\mu_{\mathbb{C}}}
\newcommand{\prr}{\phi_{\mathbb{R}}}
\newcommand{\pcc}{\phi_{\mathbb{C}}}

\DeclareMathOperator{\Lie}{Lie}
\DeclareMathOperator{\Aut}{Aut}
\DeclareMathOperator{\Tr}{Tr}
\DeclareMathOperator{\Image}{Im}
\DeclareMathOperator{\Ad}{Ad}
\DeclareMathOperator{\Diff}{Diff}
\DeclareMathOperator{\Vect}{Vect}
\DeclareMathOperator{\Sympl}{Sympl}
\DeclareMathOperator{\Span}{Span}
\DeclareMathOperator{\ind}{ind}
\DeclareMathOperator{\Td}{Td}
\DeclareMathOperator{\Ch}{Ch}
\DeclareMathOperator{\Ind}{Ind}
\DeclareMathOperator{\pt}{pt}
\DeclareMathOperator{\rk}{rk}
\DeclareMathOperator{\coker}{coker}
\DeclareMathOperator{\Pf}{Pf}
\DeclareMathOperator{\Vol}{Vol}
\DeclareMathOperator{\Res}{Res}
\DeclareMathOperator{\HK}{HK}

\DeclareMathOperator{\GL}{GL}
\DeclareMathOperator{\SO}{SO}
\DeclareMathOperator{\UU}{U}

\newcommand\restr[2]{{% we make the whole thing an ordinary symbol
		\left.\kern-\nulldelimiterspace % automatically resize the bar with \right
		#1 % the function
		\vphantom{\big|} % pretend it's a little taller at normal size
		\right|_{#2} % this is the delimiter
}}

\title{Geometric Quantisation of Hypertoric Manifolds by Symplectic Cutting}

\date{}	% Here you can change the date presented in the paper title
%\date{} 					% Or removing it

%\author{
%  David S.~Hippocampus\thanks{Use footnote for providing further
%    information about author (webpage, alternative
%    address)---\emph{not} for acknowledging funding agencies.} \\
%  Department of Computer Science\\
%  Cranberry-Lemon University\\
%  Pittsburgh, PA 15213 \\
%  \texttt{hippo@cs.cranberry-lemon.edu} \\
%% examples of more authors
%   \And
% Elias D.~Striatum \\
%  Department of Electrical Engineering\\
%  Mount-Sheikh University\\
%  Santa Narimana, Levand \\
%  \texttt{stariate@ee.mount-sheikh.edu} \\
%% \AND
%% Coauthor \\
%% Affiliation \\
%% Address \\
%% \texttt{email} \\
%% \And
%% Coauthor \\
%% Affiliation \\
%% Address \\
%% \texttt{email} \\
%% \And
%% Coauthor \\
%% Affiliation \\
%% Address \\
%% \texttt{email} \\
%}

\begin{document}
	\maketitle
	
	\begin{abstract}
		Lorem ipsum.
	\end{abstract}
	
	\section{Introduction}
	
	Lorem ipsum.
	
	\section{Background}
	
	\section{Toric Hyperk\"ahler manifolds}
	
	\subsection{Hyperk\"ahler Quotients}
	
	Hyperk\"ahler manifolds are Riemannian manifolds $(M,g)$ with three complex structures $J_{1}, J_{2}$, and $J_{3}$, satisfying the quaternionic multiplication identities, which are compatible with the metric $g$. By compatible we mean that
	\[
		\w_{1}(v,w) = g(J_{1}v,w),\quad \w_{2}(v,w) = g(J_{2}v,w),\quad \w_{3}(v,w) = g(J_{3}v,w),
	\]
	with each $\w_{i}$ a symplectic $2$-form on $M$, so that each quintuple $(M, g, J_{i}, \w_{i})$ constitutes a K\"ahler manifold in their own individual right for $i = 1, 2, 3$; combining these three K\"ahler structures together endows $M$ with a \emph{hyperk\"ahler structure}.
		
	Let us fix a complex structure, $J_{1}$, say. Then the complex-valued $2$-form $\w_{2} + i\w_{3}$ is closed, non-degenerate, and holomorphic with respect to $J_{1}$. Therefore $\w_{\CC} := \w_{2} + i\w_{3}$ is a \emph{holomorphic-symplectic} $2$-form on $M$, and so any hyperk\"ahler manifold can be viewed as a \emph{holomorphic-symplectic} manifold with respect to the complex structure $J_{1}$, real symplectic form $\w_{\RR} := \w_{1}$, and holomorphic symplectic form $\w_{\CC}$.
	
	\begin{example}
		The quaternionic vector space $\HH^{n}$ is a flat hyperk\"ahler manifold whose metric $g$ arises from the Euclidean inner-product on $\RR^{4n}\cong \HH^{n}$, which is K\"ahler with respect to all three complex structures, $J_{1}, J_{2}$, and $J_{3}$ that are identified with $i, j$, and $k$ respectively, via left-multiplication. Fixing the complex structure $I_{1}$ and introducing the coordinates $z_{j} = x_{j} + i y_{j}$ and $w_{j} = u_{j} + i v_{j}$, with $1 \leq j \leq n$, we can identify $\HH^{4n}$ with $\CC^{n} \times \CC^{n}$ via the map
		\[
		x_{j} + iy_{j} + ju_{j} + kv_{j} \mapsto (x_{j} + iy_{j}, u_{j} + iv_{j}).
		\]
		Then the symplectic forms on $\HH^{n}$ are
		\begin{equation*}
			\begin{split}
				\w_{1} &= \sum_{j=1}^{n} dx_{j} \wedge dy_{j} + du_{j} \wedge dv_{j}, \\
				\w_{2} &= \sum_{j=1}^{n} dx_{j} \wedge du_{j} + dv_{j} \wedge dy_{j}, \\
				\w_{3} &= \sum_{j=1}^{n} dx_{j} \wedge dv_{j} + dy_{j} \wedge du_{j}.
			\end{split}
		\end{equation*}
		Recalling that we fixed $J_{1}$, let us write
		\[
		z_{j} = x_{j} + iy_{j}, \quad w_{j} = u_{j} + iv_{j},
		\]
		and express the complex-symplectic form for $\HH^{n}$,
		\[
		\w_{\CC} = \w_{2} + i\w_{3} = \sum_{j=1}^{n} \left( dx_{j} \wedge du_{j} + dv_{j} \wedge dy_{j} \right) + i \left( dx_{j} \wedge dv_{j} + dy_{j} \wedge du_{j} \right) = \sum_{j=1}^{n} dz_{j} \wedge dw_{j}.
		\]
		We can observe now that
		\[
		\w_{\CC} = -d\theta, \quad \text{ where } \quad \theta = \sum_{j=1}^{n} w_{j}dz_{j} \text{ is the canonical symplectic $1$-form for $T^{\ast}\CC^{n}$,}
		\]
		so, after fixing $J_{1}$, we have the identification $(\HH^{n}, \w_{\CC}) \cong (T^{\ast}\CC^{n}, \w_{\text{can}})$.
	\end{example}
	
	
	Given a compact Lie group $G$, we say that an action of $G$ on $M$ is \emph{tri-symplectic} if it preserves each of the symplectic forms $\w_{i}$ which, from the compatibility condition , implies that the action also preserves the metric $g$ and complex structures $J_{i}$; so the action is additionally \emph{isometric} and \emph{tri-holomorphic}. We say that a tri-holomorphic action of $G$ on $M$ is \emph{tri-Hamiltonian} if it is Hamiltonian for each symplectic structure, meaning that there exists three corresponding $G$-equivariant moment maps
	\begin{equation}
		\begin{split}
			\mu_{1}, \mu_{2},\mu_{3} : &M \ra \mfg^{\ast}, \\
			d_{p}\mu_{i}(\xi, X) &= \w_{i}(\xi, X_{p}),\quad \text{at each point } p \in M,
		\end{split}
	\end{equation}
	where $\xi \in T_{p}M$, and $X \in \mfg$. Equivalently, such an action of $G$ on a hyperk\"ahler manifold $M$ is \emph{hyper-Hamiltonian} if it is Hamiltonian with respect to $\w_{\RR}$ and holomorphic Hamiltonian with respect to $\w_{\CC}$, with respective $G$-equivariant moment maps 
	\[
		\mu_{\RR} := \mu_{1} : M \ra \mfg^{\ast}, \quad \text{and} \quad \mu_{\CC} := \mu_{2} + i\mu_{3} : M \ra \mfg_{\CC}^{\ast} \cong \mfg^{\ast} \otimes \CC.
	\]
	
	The moment maps $\mu_{\RR}$ and $\mu_{\CC}$ may then be combined into what we call a \emph{hyperk\"ahler moment map}
	\[
		\mu_{\HK} := \mu_{\RR} \oplus \mu_{\CC} : M \ra \mfg^{\ast} \oplus \mfg_{\CC}^{\ast} \cong \mfg^{\ast} \otimes \RR^{3}.
	\]
	Suppose that $\lambda \in \mfg^{\ast} \oplus \mfg_{\CC}^{\ast}$ is a regular value of $\mu_{\HK}$ and central under the coadjoint action of $G$ on $\mfg^{\ast} \otimes \RR^{3}$, so that $\mu_{\HK}^{-1}(\lambda)$ is a $G$-invariant submanifold of $M$. The \emph{hyperk\"ahler quotient} is defined as
	\[
		M \sssslash_{\lambda}G := \mu_{\HK}^{-1}(\lambda)/G,
	\]
	which also has the structure of a hyperk\"ahler manifold provided that $G$ acts freely on $\mu_{\HK}^{-1}(\lambda)$. Denoting $M_{\lambda} := M \sssslash_{\lambda} G$, when the group $G$ is understood, then: 
	
	\begin{theorem}[\cite{HKLR87}]
		Let $M$ be a hyperk\"ahler manifold equipped with a hyper-Hamiltonian action of a Lie group $G$, and corresponding hyperk\"ahler moment map $\mu_{\HK} : M \ra \mfg^{\ast} \otimes \RR^{3}$. Suppose that $\lambda \in \mfg^{\ast} \otimes \RR^{3}$ is a regular value for $\mu_{\HK}$, and invariant under the coadoint action of $G$. If $G$ acts freely on $\mu_{\HK}^{-1}(\lambda)$, then the hyperk\"ahler quotient $M_{\lambda}$ is a hyperk\"ahler manifold. Moreover, if $G$ is compact and $M$ is complete, then $M_{\lambda}$ is a complete hyperk\"ahler manifold.
	\end{theorem}

	More generally, if $G$ does not act freely on $\mu_{\HK}^{-1}(\lambda)$ but only locally free, then the hyperk\"ahler quotient still has a hyperk\"ahler structure but is a \emph{hyperk\"ahler orbifold} rather than a manifold.
	
	
	
	\subsection{Toric Hyperk\"ahler Manifolds}
	
	Let us quickly go over the construction of toric hyperk\"ahler manifolds; consider the quaternionic vector space $\HH^{n}$, identified with $T^{\ast}\CC^{n}$ after fixing $J_{1}$. The standard linear action of the $n$-dimensional real torus $T^{n}$ induces an action on the cotangent bundle $T^{\ast}\CC^{n}$, and this action is hyper-Hamiltonian. The hyperk\"ahler $T^{n}$-moment map $\mu_{\HK} = \mu_{\RR} + \mu_{\CC}$ on $T^{\ast}\CC^{n}$ is given as follows: let $\{e_{j}\}_{j=1}^{n}$ be the standard basis of the lattice $\mft_{\ZZ}^{n} \subset \mft^{n} \cong \RR^{n}$, and let $\{\e_{j}\}_{j=1}^{n}$ be its dual basis for the dual lattice $(\mft_{\ZZ}^{n})^{\ast} \subset (\mft^{n})^{\ast}$. For $(z,w) = (z_{1},\ldots, z_{n}, w_{1},\ldots w_{n}) \in T^{\ast}\CC^{n}$, where the $z_{j}$ are the base variables of $\CC^{n}$, and $w_{j}$ are the variables of the cotangent fibre, we have
	\begin{align*}%\label{eq:phi-for-Tn}
		\phi_{\RR}(z,w) &= \frac{1}{2} \sum_{j=1}^{n} \left( |z_{j}|^2 - |w_{j}|^2 \right) \e_{j} \in (\mft^{n})^{\ast},
		\mbox{ and}\\
		\phi_{\CC}(z,w) &= \sum_{i=j}^{n} (z_{j} w_{j}) \e_{j} \in (\mft_{\CC}^{n})^{\ast}.
	\end{align*}
	Suppose that $K$ is a $k$-dimensional subtorus of $T^{n}$ whose Lie algebra $\mfk \subset \mft^{n}$ is generated by rational vectors; we wish to take the hyperk\"ahler quotient of $T^{\ast}\CC^{n}$ with respect to the $K$-action induced from the inclusion $\imath : K \ra T^{n}$. Such a subtorus $K$ is determined by a collection of non-zero integral vectors, $\{u_{j}\}_{j=1}^{n}$ in $\mft^{d}$ (where $d = n - k$), which we do not necessarily assume to be primitive, that generate $\mft^{d}$. This data can be represented in the following short exact sequences of vector spaces
	\begin{equation}\label{eq:Delzant-Lie}
		\xymatrix @R=-0.2pc {
			0 \ar[r] & \mfk = \mft^{k} \ar[r]^{\iota} & \mft^{n} \ar[r]^{\pi} & \mft^{d} \ar[r] & 0, \\
			&                                & e_{j} \ar@{|->}[r] & u_{j} & \\
		}
	\end{equation}
	and its dualised exact sequence
	\begin{equation}\label{eq:Delzant-Lie-dual}
		\xymatrix @R=-0.2pc {
			0 \ar[r] & (\mft^{d})^{\ast} \ar[r]^{\pi^{\ast}} \ar[r] & (\mft^{n})^{\ast} \ar[r]^{\imath^*} & \mfk^{\ast} = (\mft^{k})^{\ast} \ar[r] & 0.\\
			& & \e_{j} \ar@{|->}[r] & \lambda_{j} := \imath^* \e_{j} & \\
		}
	\end{equation}
	where $\pi$ sends $e_{j}$ to $u_{j}$. Exponentiating the first exact sequence yields an exact sequence of tori
	\begin{equation}\label{eq:Delzant-Lie-group}
		\xymatrix{
			1 \ar[r] & K = T^k \ar[r]^{\exp\iota} & T^n \ar[r]^{\exp \pi} & T^d \ar[r] & 1,}
	\end{equation}
	where 
	\[
		T^{n} = \mft^{n}/\mft_{\ZZ}^{n}, \quad T^{d} = \mft^{d}/\mft_{\ZZ}^{d}, \quad \text{ and } \quad K = \ker(\pi : T^{n} \ra T^{d}).
	\]
	Hence $K$ is a compact abelian group with Lie algebra $\mfk$, and is connected (in which case $K$ is a torus) if the vectors $\{u_{i}\}_{i=1}^{n}$ span the lattice $\mft_{\ZZ}^{d}$ over the integers.
	
	Restricting the action of $T^{n}$ on $T^{\ast}\CC^{n}$ to $K$ is hyper-Hamiltonian with hyperk\"ahler moment map
	\[
		\mu_{\HK} = (\imath^{\ast} \circ \phi_{\RR}) \oplus (\imath_{\CC}^{\ast} \circ \phi_{\CC}) : T^{\ast}\CC^{n} \ra \mfk^{\ast} \oplus \mfk_{\CC}^{\ast},
	\]
	where
	\begin{align*}%\label{eq:mu-for-K}
		\mu_{\RR}(z,w) &= \imath^{\ast}\left( \half \sum_{j=1}^{n}(|z_{j}|^{2} - |w_{j}|^{2})\e_{j} \right) \in \mfk^{\ast}
		\mbox{ and}\\
		\mu_{\CC}(z,w) &= \imath_{\CC}^{\ast} \left(\sum_{j=1}^{n}(z_{j}w_{j})\e_{j} \right) \in \mfk_{\CC}^{\ast}.
	\end{align*}
	For an element $\alpha\in \mfk^{\ast}$ and a lift $l = \tuple{l_{1}}{l_{n}} \in (\mft^{n})^{\ast}$, \ie such that $\imath^{\ast}(l) = \lambda$, the K\"ahler quotient
	\[
		X = \CC^{n} \sslash_{\lambda} K = \mu^{-1}(\lambda)/K,
	\]
	is called a \emph{toric variety}, and its hyperk\"ahler analogue
	\[
		M = T^{\ast}\CC^{n} \sssslash_{\lambda} K = \left(\mu_{\RR}^{-1}(\lambda) \cap \mu_{\CC}^{-1}(0)\right)/ K,
	\]
	is called a \emph{hypertoric variety}.
	
	When $K$ is not connected, \ie when the vectors $\{u_{j}\}_{j=1}^{n}$ \emph{do not} span $\mft_{\ZZ}^{d}$ over the integers, then $X$ and $M$ are toric and hypertoric \emph{orbifolds} respectively, instead. In this case, $K \cong K_{0} \times \Gamma$, where $K_{0}$ is the connected component of $K$ generated by the identity, hence abelian and so $K_{0}$ is now a torus, whereas $\Gamma$ is a finite abelian group.
	
	Suppose for now that $K$ is connected, then both of these have a residual action of $T^{d} = T^{n}/K$ which is effective, and also Hamiltonian in the case of $X$, hyper-Hamiltonian in the case of $M$. Thus $M$ has a residual $T^{d}$ hyperk\"ahler moment map given by
	\begin{align}
		\bmu_{\HK}[z,w] = \bmu_{\RR}[z,w] \oplus \bmu_{\CC}[z,w] &= \half \sum_{j=1}^{n}(|z_{j}|^{2} - |w_{j}|^{2} - l_{j})\e_{j} \oplus \sum_{j=1}^{n}(z_{j}w_{j})\e_{j} \\
		&\in \ker(\imath^{\ast}) \oplus \ker(\imath_{\CC}^{\ast}) = (\mft^{d})^{\ast} \oplus (\mft_{\CC}^{d})^{\ast}.
	\end{align}

	\subsection{Hyperplane Arrangements}
	
	The data used to construct a hypertoric orbifold $M$, consisting of the collection of non-zero vectors $\{u_{j}\}$ in $(\mft^{d})^{\ast}$ and an element $\lambda \in \mfk^{\ast}$, can be encoded into a finite \emph{hyperplane arrangement} in $(\mft^{d})^{\ast} \cong \RR^{d}$. 
	
	\begin{defn}
		Let $H \subseteq \RR^{d-1}$ be a (linear) hyperplane in $\RR^{d}$, \ie for $u \in \RR^{d}$ a non-zero fixed vector, that
		\[
			H = \Set{x \in \RR^{d} | \langle x,\, u \rangle = 0 }.
		\]
		Then:
		\begin{enumerate}
			\item[(rational) --] The hyperplane $H$ is \emph{rational} if $u$ has integer-valued components, \ie $u \in \ZZ^{d}$.
			\item[(weighted) --] The hyperplane $H$ is \emph{weighted} if $u$ is not required to be primitive.
			\item[(cooriented) --] Observe that the linear hyperplane $\bar{H} = \Set{ x \in \RR^{d} | \langle x,\, -u \rangle = 0}$ defines the same codimension-$1$ subspace of $\RR^{d}$ as $H$.  A \emph{coorientation} is a choice of sign for the vector $u$, and a hyperplane $H$ is \emph{cooriented} if it has a coorientation.
			\item[(affine) --] A hyperplane $\tilde{H}$ is \emph{affine} if it is a translate of the linear hyperplane $H$,
			\[
				\tilde{H} = \Set{x \in \RR^{d} | \langle x,\, u \rangle = a,\, a \in \RR} \subseteq \RR^{d-1}.	
			\]
		\end{enumerate}
	\end{defn}

	Applied to our case, the collection of vectors $\{u_{j}\}$ in $\mft^{d}$ and choice of lift $\lambda \in (\mft^{n})^{\ast}$ with $\alpha = \imath^{\ast}(\lambda)$ correspond to a hyperplane arrangement $\mcA = \{H_{1}, \ldots, H_{n}\}$ of rational, weighted, cooriented, affine hyperplanes $H_{j}$ in $(\mft^{d})^{\ast}$, given by
	\[
		H_{j} = \Set{ x \in (\mft^{d})^{\ast} | \langle x,\, u_{j} \rangle = \lambda_{j}}.
	\]
	Observe that if we choose a different lift $\tilde{\lambda} \in (\mft^{n})^{\ast}$ such that $\tilde{\lambda} -\lambda \in \ker(\imath^{\ast})$, then this corresponds to a translation of the arrangement $\mcA$ inside of $(\mft^{d})^{\ast}$, and is represented geometrically by shifting the K\"ahler and hyperk\"ahler moment maps by $\tilde{\lambda} - \lambda \in \ker(\imath^{\ast}) = (\mft^{d})^{\ast}$. 
	
	A hyperplane $H_{j}$ gives rise to the following two half-spaces in $(\mft^{d})^{\ast}$
	\begin{align*}
		H_{j}^{+} &= \Set{ x \in (\mft^{d})^{\ast} | \langle x,\, u_{j} \rangle \geq \lambda_{j} }, \\
		H_{j}^{-} &= \Set{ x \in (\mft^{d})^{\ast} | \langle x,\, u_{j} \rangle \leq \lambda_{j} },
	\end{align*}
	determined by their cooriented normal vector $u_{j}$, and clearly $H_{j} = H_{j}^{+} \cap H_{j}^{-}$.
	
	Let
	\[
		\Delta = \bigcap_{j=1}^{n} H_{j}^{+} = \Set{x \in (\mft^{d})^{\ast} | \langle x,\, u_{j} \rangle \geq \lambda_{j}, \text{ for each } 1 \leq j \leq n }.
	\]
	be the polytope in $(\mft^{d})^{\ast}$ determined by the hyperplane arrangement $\mcA$.
	
	\subsection{The Core}
	
	There is a residual $S^{1}$-action for each hypertoric manifold $M$, induced by an action of $S^{1}$ on the cotangent fibres of $T^{\ast}\CC^{n}$ by rotation, \ie
	\[
		t\cdot (z,w) = (z,tw).
	\]
	The complex-symplectic moment map $\mu_{\CC} : T^{\ast}\CC^{n} \ra (\mft_{\CC})^{d})^{\ast}$ is $S^{1}$-equivariant with respect to this action and the scalar multiplication of $S^{1}$ of $(\mft_{\CC^{n}})^{\ast}$,
		\[
		\mu_{\CC}\left(t\cdot(z,w)\right) = \mu_{\CC}(z,tw) = \sum_{j=1}^{n}\left(z_{j}(tw_{j})\right)e_{j} = t\left(\sum_{j=1}^{n}(z_{j}w_{j})e_{j} \right) = t\mu_{\CC}(z,w),
		\]
	therefore $S^{1}$ preserves $\mu_{\CC}^{-1}(0)$.
	
	\section{Localisation and Symplectic Cutting}
	
	\subsection{Index Theory}
	
	Let $(M, \w)$ be a symplectic manifold, and assume that the class of the symplectic form is \emph{integral}, \ie $\tfrac{1}{2\pi}[\w] \in H^{2}(M; \ZZ)$. This integrality condition is equivalent to the existence of a complex line bundle $\mcL \ra M$ with a Hermitian structure, such that its $1$\textsuperscript{st} Chern class $c_{1}(\mcL)$ equals the class of the symplectic form $\tfrac{1}{2\pi}[\w]$ \cite{Duistermaat2011}.
	
	Furthermore, if $M$ is acted upon by a compact Lie group $G$ in a Hamiltonian way with moment map $\mu : M \ra \mfg^{\ast}$, then it is possible to lift the action of $G$ to one on $\mcL$. 

	\subsection{Compactification via Symplectic Cutting}
	
	\nb{Section last edited on 9th June 2021}
	
	We will use the $S^{1}$-action to perform a symplectic cut of the toric hyperk\"ahler manifold $\mf{M}$ to compactify it, which has the effect of bounding the $\|w\|^{2}$-norm component of the real moment map $\bar{\mu}_{\RR}$ by above, and discarding the rest that lies above this bound. Consider the product $\mf{M} \times \CC$,  and let $S^{1}$ act on $\mf{M} \times \CC$ via the diagonal product action, i.e. $S^{1}$ acts on $M$ by rotating the cotangent fibre coordinates, and on $\CC$ in the standard way:
	$$
	e^{i\theta} \cdot \big( [z,w], \xi   \big) = \left( e^{i\theta} \cdot [z,w], e^{i\theta}\xi\right) = \left( [z,e^{i\theta}w], e^{i\theta}\xi\right).
	$$
	This action is Hamiltonian, and the corresponding moment map $\Phi : \mf{M} \times \CC \ra \RR_{\geq 0}$ for the $S^{1}$-action is
	\[
	\Phi\big( [z,w], \xi  \big) = \phi[z,w] + |\xi|^{2} = \|w\|^{2} + |\xi|^{2}.
	\]
	Then we have
	\begin{equation*}
		\begin{split}
			\Phi^{-1}(\e) &= \big\{ ([z,w],\xi) \in M \times \CC \st \|w\|^{2} + |\xi|^{2} = \e    \big\} \\
			&= \big\{ [z,w] \in M \st \|w\|^{2} = \e    \big\} \bigsqcup \big\{ ([z,w],\xi) \in M \times \CC \st |\xi| = \pm\sqrt{\e - \|w\|^{2}} \big\} \\
			&= \big\{ [z,w] \in M \st \|w\|^{2} = \e    \big\} \bigsqcup \big\{ ([z,w],\xi) \in M \times \CC \st \xi = e^{i\arg(\xi)}\sqrt{\e - \|w\|^{2}}    \big\} \\
			&= \phi^{-1}(\e) \bigsqcup \lbracket \mf{M} \times S^{1}\rbracket \\
			&=: \Sigma_{1} \sqcup \Sigma_{2},
		\end{split}
	\end{equation*}
	where we denote the level-set $\phi^{-1}(\e) \subseteq \mf{M}$ by $\Sigma_{1}$, and $\Sigma_{2} \cong \mf{M} \times S^{1}$ is the trivial $S^{1}$-bundle over $\Sigma_{2}$ given by the globally defined section
	\begin{equation*}
		\mf{M} \rightarrow \mf{M} \times S^{1}, \qquad [z,w] \longmapsto \big( [z,w], e^{i\theta}\sqrt{\e - \|w\|^{2}}\big), \qquad e^{i\theta} \in S^{1}.
	\end{equation*}
	Finally, taking the symplectic reduction of $\Phi^{-1}(\e)$ with respect to the $S^{1}$-action, we obtain the \emph{symplectic cut of $\mf{M}$ at level-$\e$},
	\begin{equation*}
		M_{\leq \e} := \Phi^{-1}(\e)/S^{1} = \Sigma_{1}/S^{1} \bigsqcup \Sigma_{2}/S^{1},
	\end{equation*}
	where $\Sigma_{1}/S^{1} \cong \phi^{-1}(\e)/S^{1}$ is just the usual symplectic reduction, and where $\Sigma_{2}/S^{1}$ is diffeomorphic to $\mf{M}$ for $\|w\|^{2} < \e$, which we will denote by $\mf{M}_{<\epsilon}$.
	
	\subsection{The Combinatorics of the Cut Space, $\mf{M}_{\leq \e}$}
	
	Since the residual circle $S^{1}$-action acts as a subtorus $S_{A}^{1}$ of the residual torus $T^{d}$ on each component $\mc{E}_{A}$ of the extended core, the hyperplane arrangement determined in $(\mft^{d})^{\ast}$ by the real moment map $\bar{\mu}_{\RR}$ is compactified by dropping in half-spaces with an inwards-pointing normal vector, given by $v_{A}$ when taking the cut. 
	
	Recall from the previous section that $j_{A}: S_{1} \hookrightarrow T^{n}$ denoted the inclusion homomorphism of $S^{1}$ into the original torus $T^{n}$. If we let $j_{A, \ast}: \mf{s}^{1} \rightarrow \mft^{n}$ represent the differential of this inclusion, then
	\[
	j_{A,\ast}(1) = \sum_{i \in A} e_{i} \in \mft^{n},
	\]
	and the generator $\exp(v_{A})$ of the one-parameter subgroup $S_{A}^{1}$ in $T^{d}$ is
	\[
	\exp(v_{A}) = \exp\lbracket \pi_{\ast} \circ j_{A, \ast}(1) \rbracket,
	\]
	or to be more concise,
	\[
	S_{A}^{1} = \Set{ \exp \lbracket r \cdot \sum_{i \in A} u_{i} \rbracket | r \in \RR }.
	\]
	Then the moment map for the restricted $S^{1}$-action to $\mcE_{A}$ is
	\begin{equation*}
		\phi_{A}[z,w] := \restr{\phi}{\mcE_{A}} [z,w] =  (j_{A}^{\ast} \circ \mrr)[z,w] = \left\langle \bar{\mu}_{\RR}[z,w], \sum_{i\in A} u_{i} \right\rangle,
	\end{equation*}
	where $j_{A}^{\ast} : (\mft^{n})^{\ast} \ra \RR^{\ast}$ is the transposed differential of the inclusion, $j_{A, \ast}$.
	
	As the $S_{A}^{1}$-action depends combinatorially on the component $\mcE_{A}$, the image of the real moment map in $(\mft^{d})^{\ast}$ is compactified by inserting a half-space $Z_{A}$ with inwards-pointing normal $v_{A} = \sum_{i \not\in A}u_{i}$ determining the orientation, on each component $\Delta_{A}$.
	
	\section{Hypertoric Subvarieties}
	
	
	

	
	
	
	
	
	
	
	
	
	
	
	
	
	
	
	
	
	
	
	
	
	
	\bibliographystyle{unsrt}  
	\bibliography{preprint-01_bibliography}  %%% Remove comment to use the external .bib file (using bibtex).
	%%% and comment out the ``thebibliography'' section.
	
	%%% Comment out this section when you \bibliography{references} is enabled.
	
\end{document}
