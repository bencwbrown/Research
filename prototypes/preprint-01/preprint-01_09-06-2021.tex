\documentclass{article}


\usepackage{arxiv}
%\usepackage{ebgaramond}
\usepackage{garamondx}
% \usepackage{CormorantGaramond}

\usepackage[utf8]{inputenc} % allow utf-8 input
\usepackage[T1]{fontenc}    % use 8-bit T1 fonts
\usepackage{hyperref}       % hyperlinks
\usepackage{url}            % simple URL typesetting
\usepackage{booktabs}       % professional-quality tables
\usepackage{amsfonts}       % blackboard math symbols
\usepackage{nicefrac}       % compact symbols for 1/2, etc.
\usepackage{microtype}      % microtypography
% \usepackage{lipsum}		% Can be removed after putting your text content
\usepackage{amsmath} 
\usepackage{amssymb}
\usepackage{graphicx}
\usepackage{epstopdf}
\usepackage{url}
\usepackage{setspace}
\usepackage{amsthm}
\usepackage{mathrsfs}
\usepackage{enumitem}
\usepackage{parskip}
\usepackage{IEEEtrantools}
\usepackage{mathtools}
\usepackage{tensor}
\usepackage{yfonts}
\usepackage{dsfont}

%%%%%%%%%%%%%%%%%%% Custom packages
\usepackage{braket}
\usepackage{todo}
\usepackage{xargs}                      % Use more than one optional parameter in a new commands
\usepackage{tikz}
\usepackage{tikz-cd}

%%%% Draft  package and commands %%%%
\usepackage{draft}
%\draftfalse
\newnote{nb}{red}

%%%%%%%%%%%%%%%%%%

\usetikzlibrary{arrows}

\newtheorem{theorem}{Theorem}[section]
\newtheorem{lemma}[theorem]{Lemma}
\newtheorem*{lemma*}{Lemma}
\newtheorem{prop}[theorem]{Proposition}
\newtheorem{corollary}[theorem]{Corollary}
\newtheorem{defn}[theorem]{Definition\rm}
\newtheorem{conjecture}[theorem]{Conjecture}
\newtheorem{remark}{\it Remark\/}
\newtheorem{example}{Example}
\newtheorem{fact}{Fact}

\newcommand{\st}{\ensuremath{:}}% such that
\newcommand{\ie}{\emph{i.e.} }
\newcommand{\eg}{\emph{e.g.} }
\newcommand{\cf}{\emph{cf.} }
\newcommand{\ra}{\rightarrow}
\newcommand{\la}{\leftarrow}
\newcommand{\lra}{\longrightarrow}
\newcommand{\lla}{\longleftarrow}
\newcommand{\lbracket}{\left(}
\newcommand{\rbracket}{\right)}

\newcommand{\al}{\alpha}
\newcommand{\w}{\omega}
\newcommand{\W}{\Omega}
\newcommand{\m}{\mu}
\newcommand{\n}{\nu}
\newcommand{\e}{\epsilon}
\newcommand{\K}{K\"ahler }
\newcommand{\HK}{hyperk\"ahler }
\newcommand{\into}{\hookrightarrow}
\newcommand{\PP}{\mathbb{P}}
\newcommand{\RR}{\mathbb{R}}
\newcommand{\CC}{\mathbb{C}}
\newcommand{\QQ}{\mathbb{Q}}
\newcommand{\FF}{\mathbb{F}}
\newcommand{\ZZ}{\mathbb{Z}}
\newcommand{\NN}{\mathbb{N}}
\newcommand{\HH}{\mathbb{H}}
\newcommand{\vp}{\varphi}
\newcommand{\mcA}{\mathcal{A}}
\newcommand{\mcE}{\mathcal{E}}
\newcommand{\mcF}{\mathcal{F}}
\newcommand{\mcG}{\mathcal{G}}
\newcommand{\mcH}{\mathcal{H}}
\newcommand{\mcL}{\mathcal{L}}
\newcommand{\mcO}{\mathcal{O}}
\newcommand{\mcR}{\mathcal{R}}
\newcommand{\mfg}{\mathfrak{g}}
\newcommand{\mfh}{\mathfrak{h}}
\newcommand{\mfk}{\mathfrak{k}}
\newcommand{\mft}{\mathfrak{t}}
\newcommand{\mc}[1]{\mathcal{#1}}
\newcommand{\mf}[1]{\mathfrak{#1}}
\newcommand{\krr}{k_{\RR}}
\newcommand{\kcc}{k_{\CC}}

\newcommand{\sslash}{\mathbin{/\mkern-6mu/}}
\newcommand{\sssslash}{\mathbin{/\mkern-6mu/\mkern-6mu/\mkern-6mu/}}

\newcommand{\pbrackets}[1]{\left( #1 \right)}
\newcommand{\bbrackets}[1]{\left[ #1 \right]}
\newcommand{\norm}[1]{|#1|^{2}}

\newcommand{\dbar}{\bar{\partial}}
\newcommand{\mrr}{\mu_{\mathbb{R}}}
\newcommand{\mcc}{\mu_{\mathbb{C}}}
\newcommand{\prr}{\phi_{\mathbb{R}}}
\newcommand{\pcc}{\phi_{\mathbb{C}}}

\DeclareMathOperator{\Lie}{Lie}
\DeclareMathOperator{\Aut}{Aut}
\DeclareMathOperator{\Tr}{Tr}
\DeclareMathOperator{\Image}{Im}
\DeclareMathOperator{\Ad}{Ad}
\DeclareMathOperator{\Diff}{Diff}
\DeclareMathOperator{\Vect}{Vect}
\DeclareMathOperator{\Sympl}{Sympl}
\DeclareMathOperator{\Span}{Span}
\DeclareMathOperator{\ind}{ind}
\DeclareMathOperator{\Td}{Td}
\DeclareMathOperator{\Ch}{Ch}
\DeclareMathOperator{\Ind}{Ind}
\DeclareMathOperator{\pt}{pt}
\DeclareMathOperator{\rk}{rk}
\DeclareMathOperator{\coker}{coker}
\DeclareMathOperator{\Pf}{Pf}
\DeclareMathOperator{\Vol}{Vol}
\DeclareMathOperator{\Res}{Res}

\DeclareMathOperator{\GL}{GL}
\DeclareMathOperator{\SO}{SO}
\DeclareMathOperator{\UU}{U}

\newcommand\restr[2]{{% we make the whole thing an ordinary symbol
		\left.\kern-\nulldelimiterspace % automatically resize the bar with \right
		#1 % the function
		\vphantom{\big|} % pretend it's a little taller at normal size
		\right|_{#2} % this is the delimiter
}}

\title{Geometric Quantisation of Hypertoric Manifolds by Symplectic Cutting}

\date{}	% Here you can change the date presented in the paper title
%\date{} 					% Or removing it

%\author{
%  David S.~Hippocampus\thanks{Use footnote for providing further
%    information about author (webpage, alternative
%    address)---\emph{not} for acknowledging funding agencies.} \\
%  Department of Computer Science\\
%  Cranberry-Lemon University\\
%  Pittsburgh, PA 15213 \\
%  \texttt{hippo@cs.cranberry-lemon.edu} \\
%% examples of more authors
%   \And
% Elias D.~Striatum \\
%  Department of Electrical Engineering\\
%  Mount-Sheikh University\\
%  Santa Narimana, Levand \\
%  \texttt{stariate@ee.mount-sheikh.edu} \\
%% \AND
%% Coauthor \\
%% Affiliation \\
%% Address \\
%% \texttt{email} \\
%% \And
%% Coauthor \\
%% Affiliation \\
%% Address \\
%% \texttt{email} \\
%% \And
%% Coauthor \\
%% Affiliation \\
%% Address \\
%% \texttt{email} \\
%}

\begin{document}
	\maketitle
	
	\begin{abstract}
		Lorem ipsum.
	\end{abstract}
	
	\section{Introduction}
	
	Lorem ipsum.
	
	\section{Background}
	
	\subsection{Hypertoric Geometry}
	
	\subsection{Localisation and Symplectic Cutting}
	
	
	
	\section{Hypertoric Varieties}

	\subsection{Compactification via Symplectic Cutting}
	
	We will use the $S^{1}$-action to perform a symplectic cut of the toric \HK manifold $\mf{M}$ to compactify it, which has the effect of bounding the $\|w\|^{2}$-norm component of the real moment map $\bar{\mu}_{\RR}$ by above, and discarding the rest that lies above this bound. Consider the product $\mf{M} \times \CC$,  and let $S^{1}$ act on $\mf{M} \times \CC$ via the diagonal product action, i.e. $S^{1}$ acts on $M$ by rotating the cotangent fibre coordinates, and on $\CC$ in the standard way:
	$$
	e^{i\theta} \cdot \big( [z,w], \xi   \big) = \left( e^{i\theta} \cdot [z,w], e^{i\theta}\xi\right) = \left( [z,e^{i\theta}w], e^{i\theta}\xi\right).
	$$
	This action is Hamiltonian, and the corresponding moment map $\Phi : \mf{M} \times \CC \ra \RR_{\geq 0}$ for the $S^{1}$-action is
	\[
	\Phi\big( [z,w], \xi  \big) = \phi[z,w] + |\xi|^{2} = \|w\|^{2} + |\xi|^{2}.
	\]
	Then we have
	\begin{equation*}
		\begin{split}
			\Phi^{-1}(\e) &= \big\{ ([z,w],\xi) \in M \times \CC \st \|w\|^{2} + |\xi|^{2} = \e    \big\} \\
			&= \big\{ [z,w] \in M \st \|w\|^{2} = \e    \big\} \bigsqcup \big\{ ([z,w],\xi) \in M \times \CC \st |\xi| = \pm\sqrt{\e - \|w\|^{2}} \big\} \\
			&= \big\{ [z,w] \in M \st \|w\|^{2} = \e    \big\} \bigsqcup \big\{ ([z,w],\xi) \in M \times \CC \st \xi = e^{i\arg(\xi)}\sqrt{\e - \|w\|^{2}}    \big\} \\
			&= \phi^{-1}(\e) \bigsqcup \lbracket \mf{M} \times S^{1}\rbracket \\
			&=: \Sigma_{1} \sqcup \Sigma_{2},
		\end{split}
	\end{equation*}
	where we denote the level-set $\phi^{-1}(\e) \subseteq \mf{M}$ by $\Sigma_{1}$, and $\Sigma_{2} \cong \mf{M} \times S^{1}$ is the trivial $S^{1}$-bundle over $\Sigma_{2}$ given by the globally defined section
	\begin{equation*}
		\mf{M} \rightarrow \mf{M} \times S^{1}, \qquad [z,w] \longmapsto \big( [z,w], e^{i\theta}\sqrt{\e - \|w\|^{2}}\big), \qquad e^{i\theta} \in S^{1}.
	\end{equation*}
	Finally, taking the symplectic reduction of $\Phi^{-1}(\e)$ with respect to the $S^{1}$-action, we obtain the \emph{symplectic cut of $\mf{M}$ at level-$\e$},
	\begin{equation*}
		M_{\leq \e} := \Phi^{-1}(\e)/S^{1} = \Sigma_{1}/S^{1} \bigsqcup \Sigma_{2}/S^{1},
	\end{equation*}
	where $\Sigma_{1}/S^{1} \cong \phi^{-1}(\e)/S^{1}$ is just the usual symplectic reduction, and where $\Sigma_{2}/S^{1}$ is diffeomorphic to $\mf{M}$ for $\|w\|^{2} < \e$, which we will denote by $\mf{M}_{<\epsilon}$.
	
	\subsection{The Combinatorics of the Cut Space, $\mf{M}_{\leq \e}$}
	
	Since the residual circle $S^{1}$-action acts as a subtorus $S_{A}^{1}$ of the residual torus $T^{d}$ on each component $\mc{E}_{A}$ of the extended core, the hyperplane arrangement determined in $(\mft^{d})^{\ast}$ by the real moment map $\bar{\mu}_{\RR}$ is compactified by dropping in half-spaces with an inwards-pointing normal vector, given by $v_{A}$ when taking the cut. 
	
	Recall from the previous section that $j_{A}: S_{1} \hookrightarrow T^{n}$ denoted the inclusion homomorphism of $S^{1}$ into the original torus $T^{n}$. If we let $j_{A, \ast}: \mf{s}^{1} \rightarrow \mft^{n}$ represent the differential of this inclusion, then
	\[
	j_{A,\ast}(1) = \sum_{i \in A} e_{i} \in \mft^{n},
	\]
	and the generator $\exp(v_{A})$ of the one-parameter subgroup $S_{A}^{1}$ in $T^{d}$ is
	\[
	\exp(v_{A}) = \exp\lbracket \pi_{\ast} \circ j_{A, \ast}(1) \rbracket,
	\]
	or to be more concise,
	\[
	S_{A}^{1} = \Set{ \exp \lbracket r \cdot \sum_{i \in A} u_{i} \rbracket | r \in \RR }.
	\]
	Then the moment map for the restricted $S^{1}$-action to $\mcE_{A}$ is
	\begin{equation*}
		\phi_{A}[z,w] := \restr{\phi}{\mcE_{A}} [z,w] =  (j_{A}^{\ast} \circ \mrr)[z,w] = \left\langle \bar{\mu}_{\RR}[z,w], \sum_{i\in A} u_{i} \right\rangle,
	\end{equation*}
	where $j_{A}^{\ast} : (\mft^{n})^{\ast} \ra \RR^{\ast}$ is the transposed differential of the inclusion, $j_{A, \ast}$.
	
	As the $S_{A}^{1}$-action depends combinatorially on the component $\mcE_{A}$, the image of the real moment map in $(\mft^{d})^{\ast}$ is compactified by inserting a half-space $Z_{A}$ with inwards-pointing normal $v_{A} = \sum_{i \not\in A}u_{i}$ determining the orientation, on each component $\Delta_{A}$.
	
	\section{Hypertoric Subvarieties}
	
	\subsection{Universal Modifications}
	
	Let $(M, \w_{HK}, \Phi_{HK})$ be a tri-Hamiltonian $K$-manifold , where $\w_{HK} = \w_{\RR} + \w_{\CC}$ and $\Phi_{HK} = \Phi_{\RR} + \Phi_{\CC}$. Define $j_{\CC} : M \ra M \times M \times T^{\ast}K_{\CC}$ by $j(m) = (m, 1, \Phi_{\CC}(m))$, and $j_{\RR} : M \ra M \times T^{\ast}K$ by $j_{\RR}(m) = (m, 1, \Phi_{\RR})$.
		
		\begin{lemma}
			Let $(M, I, \w, K_{\CC}, \mu)$ be a complex-symplectic $K$-Hamiltonian manifold. Then the complex-symplectic quotient $(M \times T^{\ast}K_{\CC}) \sslash_{0} K_{\CC}$ is isomorphic to $M$ as complex-symplectic manifolds. Here, the complex-symplectic quotient $(M \times T^{\ast}K_{\CC}) \sslash_{0} K_{\CC}$ is taken with respect to the diagonal $K_{\CC}$-action, where $K_{\CC}$ acts on $T^{\ast}K_{\CC}$ from the left.
		\end{lemma}

	\begin{proof}
		The $K_{\CC}$-action on $T^{\ast}K_{\CC}$ is free and proper, and thus so is the diagonal action on the product $M \times T^{\ast}K_{\CC}$, so we are in the position to apply the holomorphic Marsden-Weinstein Reduction Theorem. Identifying via polar decomposition $T^{\ast}K_{\CC} \cong K_{\CC} \times \mfk_{\CC}^{\ast}$, the moment map for the diagonal action is
		
		\[
			\Phi : M \times K_{\CC} \times \mfk_{\CC}^{\ast} \lra \mfk^{\ast} \otimes \CC, \qquad \Phi(m, k, \lambda) = \mu(m) - \Ad_{g}^{\ast}(\lambda).
		\]
		
		\[
		\begin{tikzcd}
			(M, \w)\arrow[r, "j"] \arrow[dr, dotted, "\bar{\jmath}"] & (\Phi^{-1}(0), \w) \arrow[d, "\pi"] \arrow[dr, hook, "i"] &  \\
			& (\Phi^{-1}(0) / G_{\CC}, \eta) & (M \times T^{\ast}G_{\CC}, \w + \eta )
		\end{tikzcd}
		\]
	\end{proof}
	
	
	
	

	
	
	
	
	
	
	
	
	
	
	
	
	
	
	
	
	
	
	
	
	
	
	\bibliographystyle{unsrt}  
	\bibliography{preprint-01_bibliography}  %%% Remove comment to use the external .bib file (using bibtex).
	%%% and comment out the ``thebibliography'' section.
	
	%%% Comment out this section when you \bibliography{references} is enabled.
	
\end{document}
