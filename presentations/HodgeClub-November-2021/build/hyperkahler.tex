\section{Hyperk\"ahler Manifolds}

\begin{frame}{Definition}
    \textcolor{blue}{\underline{Def:}} A \textcolor{red}{hyperk\"ahler manifold} $M$ has: a Riemannian metric $g$, three $\mathbb{C}$-structures, $J_{1}, J_{2}, J_{3}$, such that
        \begin{itemize}
            \item $J_{1}, J_{2}, J_{3}$ are parallel with respect to $g$,
            \item $J_{1}^{2} = J_{2}^{2} = J_{3}^{2} = -1$, and $J_{1}J_{2} = -J_{2}J_{1} = J_{3}$, etc.,
            \item $M$ has three K\"ahler forms, one for each $\mathbb{C}$-structure
            \[
                \w_{1} = g(J_{1}\,,\,),\ \w_{J_{2}} = g(J_{2}\,,\,),\ \w_{J_{3}} = g(J_{3}\,,\,).    
            \]
        \end{itemize}        
    \textcolor{blue}{\underline{Fun Facts:}}
    \begin{itemize}
        \item Fixing $\mathbb{C}$-stucture $J_{1}$ for hyperk\"ahler $(M, J_{i}, \omega_{i})$, set
        \[
            \omega_{\mathbb{C}} = \omega_{2} + i\omega_{3}.    
        \]
        $\rightsquigarrow (M,J_{1}, \omega_{\mathbb{C}})$ is a \textcolor{red}{complex-symplectic} manifold, holomorphic wrt $J_{1}$.
    \end{itemize}
\end{frame}

\begin{frame}{}
    \textcolor{blue}{\underline{Fun Facts cont'd:}} 
        \begin{itemize}
            \item $\dim_{\mathbb{R}}(M) = 4n$,
            \begin{itemize}
                \item $2n$ from being symplectic,
                \item then $\times 2$ from being complex.
            \end{itemize}
            \item Holonomy group of $M \unlhd \text{Sp}(n)$,
            \item $\{$ Hyperk\"ahler manifolds $\} \subset$ $\{$ Calabi-Yau manifolds $\}$.
            \item Most hyperk\"ahler manifolds are non-compact $\rightsquigarrow$ completeness of the metric is the next best thing.
        \end{itemize}
\end{frame}

\begin{frame}{$4$-Dimensional Hyperk\"ahler Manifolds}
    \begin{itemize}
        \item For hyperk\"ahler $M$ with $\dim_{\mathbb{R}}(M) = 4$, it is \textcolor{red}{Ricci-flat}, \break $\implies$ its metric solves the \textcolor{red}{vacuum Einstein field equations},
        \item Only known \textcolor{red}{compact $4$-dimensional} examples are the $4$-torus $T^{4}$, and $K3$ surfaces,
        \item $\text{Sp}(1) = \text{SU}(2) \implies$ K\"ahler and Ricci-flat $\implies$ hyperk\"ahler.
    \end{itemize}
\end{frame}

\begin{frame}{Non-Compact Examples}
    \begin{itemize}
        \item \textcolor{red}{$\mathbb{H}$:} Let $\mathbb{H}$ be the flat quaternionic vector space. Identify
        \[
            \mathbb{H}\ \longleftrightarrow\ \mathbb{R}^{4}, \quad q = x_{0} + ix_{1} + jx_{2} + kx_{3}\ \longleftrightarrow\ (x_{0}, x_{1}, x_{2}, x_{3}).
        \]
        $\mathbb{H}$ has K\"ahler forms
        \begin{align*}
            \omega_{1} &= dx_{0} \wedge dx_{1} + dx_{2} \wedge dx_{3}, \\
            \omega_{2} &= dx_{0} \wedge dx_{2} + dx_{3} \wedge dx_{1}, \\
            \omega_{3} &= dx_{0} \wedge dx_{3} + dx_{1} \wedge dx_{2}.
        \end{align*}
        Fix $J_{1}$, set $\omega_{\mathbb{C}} = \omega_{2} + i\omega_{3}$, \textcolor{red}{$z = x_{0} + ix_{1}$}, \textcolor{red}{$w = x_{2} + ix_{3}$},
        \begin{align*}
            \omega_{\mathbb{C}} &= (dx_{0} \wedge dx_{2} + dx_{3} \wedge dx_{1}) + i(dx_{0} \wedge dx_{3} + dx_{1} \wedge dx_{2}) \\
            &= (dx_{0} + idx_{1}) \wedge (dx_{2} + i dx_{3}) = dz \wedge dw = d(-wdz),
        \end{align*}
    \end{itemize}
\end{frame}

\begin{frame}
    \[
        \omega_{\mathbb{C}} = dz \wedge dw = d(-wdz) = d\theta.
    \]
    \begin{itemize}
        \item 
        \begin{itemize}
            \item View as $z\in \mathbb{C}$ (in base) and $w \in T_{z}^{\ast}\mathbb{C}$ (in fibre).
            \item $\implies dz \wedge dw$ is the \textcolor{red}{canonical $2$-form} on $T^{\ast}\mathbb{C}$,
            \item $\theta$ its \textcolor{red}{tautological/Liouville $1$-form}.
        \end{itemize}
        \item \textcolor{red}{$T^{\ast}G_{\mathbb{C}}$:} For $G$ a compact Lie group, the cotangent bundle of its \textcolor{red}{complexification}, $T^{\ast}G_{\mathbb{C}}$, is hyperk\"ahler.
        \item \textcolor{blue}{P. Kronheimer (1988)} proved this using hyperk\"ahler reduction.
        \item Used the \textcolor{red}{Nahm equations} as the hyperk\"ahler moment map.
    \end{itemize}
\end{frame}

\begin{frame}
    \begin{itemize}
        \item \textcolor{red}{Gibbons-Hawking Ansatz:} Principal $S^{1}$-bundle over $\mathbb{R}^{3}$,
        \[
            (t,x,y,z) \ni S^{1} \times \mathbb{R}^{3} \rightarrow \mathbb{R}^{3}.
        \]
        \textcolor{blue}{G. Gibbons \& S. Hawking (1978)} used the metric
        \[
            g = V(\underline{x})^{-1}(dt + \underline{A} \cdot d\underline{x})^{2} + V(\underline{x})\, d\underline{x} \cdot d\underline{x},
        \]
        a \textcolor{red}{``\emph{gravitational instanton}''}.
        \item Here $\underline{A} = A_{i}dx^{i}$ is an $S^{1}$-connection $1$-form, and
        \[
            \underline{\nabla}V = \underline{\nabla} \times \underline{A} \implies \Delta V = 0,
        \]
        the \textcolor{red}{Bogomolny/monopole equations}.
        \item They come up as solutions to the \textcolor{red}{Yang-Mills equations}.
    \end{itemize}
\end{frame}