\section{Hyperk\"ahler Reduction}

\begin{frame}{Hyperk\"ahler Moment Maps}
    $M$ hyperk\"ahler, $G$ compact Lie group acting on $M$, preserving hyperk\"ahler stucture.
    \begin{block}{Definition}
        A \textcolor{red}{hyperk\"ahler moment map} for the $G$-action is
        \[
            \mu_{\HK} = (\mu_{\mathbb{R}}, \mu_{\mathbb{C}}) \rightarrow \mathfrak{g}^{\ast}\otimes \mathbb{R}^{3},    
        \]
        which is $G$-equivariant, and $d_{p}\mu_{i}(X) = \omega_{p}(X_{M},\ )$.
    \end{block}
    Here, $p \in M$, $X \in \mathfrak{g}$, and 
    \[
        X_{M} = \restr{\frac{d}{dt}}{t=0}\left(e^{tX}\cdot p\right)
    \]
    is the fundamental vector field of $X$, for the infinitesimal action of $\mfg$ on $M$.
\end{frame}

\begin{frame}{Hyperk\"ahler Quotients}
    \begin{block}{Definition}
        $G$ \textcolor{red}{compact} Lie group acting on hyperk\"ahler $M$, with moment map $\mu_{\HK} : M \rightarrow \mfg^{\ast}\otimes \RR^{3}$. Let $\lambda \in \mfg^{\ast} \otimes \RR^{3}$ be $\Ad_{G}^{\ast}$-invariant.
        
        The \textcolor{red}{hyperk\"ahler quotient} of $M$ by $G$ is
        \[
            M_{\lambda} := M \sssslash_{\lambda} := \mu_{\HK}^{-1}(\lambda)/G.    
        \]
    \end{block}
    \begin{itemize}
        \item $\dim_{\RR}(\mu_{\HK}^{-1}(\lambda)) = \dim_{\RR}(M) - 3\dim_{\RR}(G)$, so
    \end{itemize}
    \[
            \dim_{\RR}(M_{\lambda}) = \dim_{\RR}(\mu_{\HK}^{-1}(\lambda)) - \dim_{\RR}(G) = \dim_{\RR}(M) - 4\dim_{\RR}(G).    
    \]
    \begin{itemize}
        \item So $M$ has the right dimension to be hyperk\"ahler.
    \end{itemize}
\end{frame}

\begin{frame}{Hyperk\"ahler Reduction}
    Nothing has been said about the structure of $M_{\lambda}$ yet:
    \begin{block}{Theorem}
        Suppose furthermore $G$ acts on $M$ freely. Then the hyperk\"ahler quotient $M_{\lambda}$ has a hyperk\"ahler structure.
    \end{block}
    \begin{itemize}
        \item Proven by \textcolor{blue}{Hitchin, Karlhede, Lindstr\"om, Ro\v{c}ek, (1987)}.
        \item If instead $G$ only acts \textcolor{red}{locally freely}, $M_{\lambda}$ is then a hyperk\"ahler \textcolor{red}{orbifold}.
        \item Allows us to construct many examples from simple ones, such as the flat quaternionic space $\HH^{n}$.
    \end{itemize}
\end{frame}

\begin{frame}{Example}
    \begin{itemize}
        \item Let $M = \HH^{3} \cong T^{\ast}\CC^{3}$, and $S^{1} \acts M$ via inclusion $S^{1} \overset{\imath}{\hookrightarrow} T^{3}$, $\imath(t) = (t,t,t)$:
    \end{itemize}
    \[
        \imath(t) \cdot (z_{1},z_{2},z_{3}, w_{1}, w_{2}, w_{3}) = (tz_{1},tz_{2},tz_{3}, t^{-1}w_{1}, t^{-1}w_{2}, t^{-1}w_{3}).    
    \]
    \vspace{-20pt}
    \begin{itemize}
        \item Action is free with hyperk\"ahler moment map
    \end{itemize}
    \begin{align*}
        \mu_{\RR}(z,w) &= \tfrac{1}{2}\left(\|z\|^{2} - \|w\|^{2}\right) \in \RR, \\
        \mu_{\CC}(z,w) &= z_{1}w_{1} + z_{2}w_{2} + z_{3}w_{3} \in \CC.
    \end{align*}
    \vspace{-20pt}
    \begin{itemize}
        \item Reduce at $\lambda = (\alpha/2,\underline{0}) \in \RR^{3}$:
    \end{itemize}
    \begin{align*}
        M_{\lambda} &= \left(\mu_{\RR}^{-1}(\alpha/2) \cap \mu_{\CC}^{-1}(0)\right)/S^{1} \\
        &= \{(z,w) \in T^{\ast}\CC^{3}\, |\, \|z\|^{2} - \|w\|^{2} = \alpha,\, \langle z, w \rangle = 0 \}/S^{1} \cong T^{\ast}\CC\PP^{2}.
    \end{align*}
\end{frame}

\begin{frame}
    \begin{itemize}
        \item $T^{\ast}\CC\PP^{2}$ has a ``\emph{residual}'' action of $T^{2} \cong T^{3}/S^{1}$,
    \end{itemize}
    \[
        (s,t)\cdot [z_{1},z_{2},z_{3}, w_{1}, w_{2}, w_{3}] = [sz_{1},tz_{2},z_{3}, s^{-1}w_{1}, t^{-1}w_{2}, w_{3}].    
    \]
    \begin{itemize}
        \item $T^{2}$-action also has a moment map, $\bar{\mu}_{\HK} : T^{\ast}\CC\PP^{2} \ra \RR^{2}$.
    \end{itemize}

    \begin{itemize}
        \item $\Image{\bar{\mu}_{\RR}} =$
    \end{itemize}
\end{frame}