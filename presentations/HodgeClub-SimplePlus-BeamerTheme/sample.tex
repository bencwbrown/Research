%----------------------------------------------------------------------------------------
%	PACKAGES AND THEMES
%----------------------------------------------------------------------------------------
\documentclass[aspectratio=169,xcolor=dvipsnames]{beamer}
\usetheme{SimplePlus}

\usepackage{hyperref}
\usepackage{graphicx} % Allows including images
\usepackage{booktabs} % Allows the use of \toprule, \midrule and \bottomrule in tables
\usepackage{tikz}
\usetikzlibrary{shapes.geometric, arrows, shadows, positioning}
\usepackage{tikz-cd}
\usepackage{adforn} % ornaments, used in titlepage

\input xy
\xyoption{all}


% -----------------------------------
% math
% -----------------------------------

\usepackage{amsmath}
\usepackage{amsfonts}
\usepackage{amssymb}
\usepackage{amsthm}
\usepackage{mathrsfs}   % \mathscr
\usepackage{stmaryrd}   % \lightning
\usepackage{mathabx,epsfig}
\usepackage{physics}

% -----------------------------------
% macros
% -----------------------------------

\newcommand{\ra}{\rightarrow}
\newcommand{\la}{\leftarrow}
\newcommand{\lra}{\longrightarrow}
\newcommand{\lla}{\longleftarrow}
\newcommand{\into}{\hookrightarrow}
\newcommand{\Spec}{\operatorname{Spec}}
\newcommand{\Proj}{\operatorname{Proj}}
\newcommand{\NN}{\mathbb{N}}
\newcommand{\ZZ}{\mathbb{Z}}
\newcommand{\QQ}{\mathbb{Q}}
\newcommand{\RR}{\mathbb{R}}
\newcommand{\CC}{\mathbb{C}}
\newcommand{\PP}{\mathbb{P}}
\newcommand{\FF}{\mathbb{F}}
\newcommand{\GG}{\mathbb{G}}
\newcommand{\HH}{\mathbb{H}}
\newcommand{\kk}{\mathbb{k}}
\newcommand{\ee}{\mathbb{e}}
\newcommand{\bbA}{\mathbb{A}}
\newcommand{\TT}{\mathbb{T}}
\newcommand{\half}{\frac{1}{2}}
\newcommand{\Mat}{\text{Mat}}
\newcommand{\Hom}{\operatorname{Hom}}
\newcommand{\Span}{\text{Span}}
\newcommand{\Bl}{\operatorname{Bl}}
\newcommand{\Pic}{\operatorname{Pic}}
\newcommand{\Id}{\operatorname{Id}}
\newcommand{\Td}{\operatorname{Td}}
\newcommand{\Lie}{\operatorname{Lie}}
\newcommand{\mf}[1]{\mathfrak{#1}}
\newcommand{\mfg}{\mathfrak{g}}
\newcommand{\mfk}{\mathfrak{k}}
\newcommand{\mfn}{\mathfrak{n}}
\newcommand{\mft}{\mathfrak{t}}
\newcommand{\mc}[1]{\mathcal{#1}}
\newcommand{\mcL}{\mathcal{L}}
\newcommand{\ed}{e^{\ast}}
\newcommand{\fd}{f^{\ast}}
\newcommand{\semistable}{\text{ss}}
\newcommand{\stable}{\text{s}}
\newcommand{\dual}[1]{#1^{\ast}}
\newcommand{\w}{\omega}
\newcommand{\pt}{\text{pt}}
\newcommand{\GL}{\operatorname{GL}}
\newcommand{\SL}{\operatorname{SL}}
\newcommand{\SU}{\operatorname{SU}}
\newcommand{\Sp}{\operatorname{Sp}}
\newcommand{\PSL}{\operatorname{PSL}}
\newcommand{\Hilb}{\operatorname{Hilb}}
\newcommand{\lat}{\operatorname{lat}}
\newcommand{\gp}{\operatorname{gp}}
\newcommand{\pos}{\operatorname{pos}}
\newcommand{\Diff}{\operatorname{Diff}}
\newcommand{\ind}{\operatorname{ind}}
\newcommand{\td}{\operatorname{td}}
\newcommand{\ch}{\operatorname{ch}}
\newcommand{\coker}{\operatorname{coker}}
\newcommand{\Ver}{\operatorname{Ver}}
\newcommand{\Ad}{\operatorname{Ad}}
\newcommand{\HK}{\operatorname{HK}}
\newcommand{\Image}{\operatorname{image}}

\newcommand{\sssslash}{\mathbin{/\mkern-6mu/\mkern-6mu/\mkern-6mu/}}
\newcommand{\restr}[2]{\left.#1\right|_{#2}}

\def\acts{\mathrel{\reflectbox{$\righttoleftarrow$}}}

%----------------------------------------------------------------------------------------
%	TITLE PAGE
%----------------------------------------------------------------------------------------

\title[short title]{Simple Beamer Theme} % The short title appears at the bottom of every slide, the full title is only on the title page
\subtitle{Subtitle}

\author[Pin-Yen] {Pin-Yen Huang}

\institute[NTU] % Your institution as it will appear on the bottom of every slide, may be shorthand to save space
{
    Department of Computer Science and Information Engineering \\
    National Taiwan University % Your institution for the title page
}
\date{\today} % Date, can be changed to a custom date


%----------------------------------------------------------------------------------------
%	PRESENTATION SLIDES
%----------------------------------------------------------------------------------------

\begin{document}

\begin{frame}
    % Print the title page as the first slide
    \titlepage
\end{frame}

\begin{frame}{Overview}
    % Throughout your presentation, if you choose to use \section{} and \subsection{} commands, these will automatically be printed on this slide as an overview of your presentation
    \tableofcontents
\end{frame}

%------------------------------------------------
\section{Orbifolds}
%------------------------------------------------

\begin{frame}{Charts}
	\textcolor{blue}{\underline{Def:}} For $X$ a paracompact Hausdorff space. An \textcolor{red}{orbifold chart} is a triple $(\widetilde{U}, \Gamma, \phi)$:
	\begin{itemize}
		\item $\widetilde{U} \subseteq \RR^{n}$ is open, and $\{0\} \in \widetilde{U}$;
		\item $\Gamma$ finite group, effectively acting on $\widetilde{U}$;
		\item $\phi : \widetilde{U} \ra U$ continuous onto $U \subset X$, \newline such that $\phi \circ \gamma = \phi$, for all $\gamma \in \Gamma$;
		\item Induced $\phi : \widetilde{U}/\Gamma \ra U$ is a homeomorphism.\newline
	\end{itemize}
	
	For two charts, $(\widetilde{U}_{1}, \Gamma_{1}, \phi_{1})$, and $(\widetilde{U}_{2}, \Gamma_{2}, \phi_{2})$: \newline
	
	\textcolor{blue}{\underline{Def:}} An \textcolor{red}{embedding} between them is a smooth embedding $\lambda : \widetilde{U}_{1} \ra \widetilde{U}_{2}$ such that
	\[
	\phi_{2} \circ \lambda = \phi_{1}.    
	\]
\end{frame}

\begin{frame}{Atlases}
	\textcolor{blue}{\underline{Def:}} An \textcolor{red}{orbifold atlas} on $X$ is a family $\mathcal{U} = \{\widetilde{U}_{i}, \Gamma_{i}, \phi_{i}\}$ of charts such that
	\begin{itemize}
		\item $X = \bigcup_{i} \phi_{i}(\widetilde{U}_{i})$;
		\item Two charts $(\widetilde{U}_{1}, \Gamma_{1}, \phi_{1})$, and $(\widetilde{U}_{2}, \Gamma_{2}, \phi_{2})$, with $U_{1} = \phi_{1}(\widetilde{U}_{1})$ and $U_{2} = \phi_{2}(\widetilde{U}_{2})$, and a point $x \in U_{1} \cap U_{3}$, there is an open neighbourhood $U_{3}$ of $x$, and a chart $(\widetilde{U}_{3}, \Gamma_{3}, \phi_{3})$ such that there are embeddings
		\[
		\lambda_{31} : \widetilde{U}_{3} \ra \widetilde{U}_{1}, \quad \lambda_{32} : \widetilde{U}_{3} \ra \widetilde{U}_{2}.
		\]
	\end{itemize}
	\textcolor{blue}{\underline{Note:}} Similarly for manifolds, an orbifold atlas $\mathcal{U}$ has a \textcolor{red}{refinement} $\mathcal{V}$ if each chart $\widetilde{U} \overset{\text{injects}}{\longrightarrow} \widetilde{V}$.
\end{frame}

\begin{frame}{Orbifolds}
	\textcolor{blue}{\underline{Def:}} An \textcolor{red}{orbifold} is a paracompact Hausdorff space $X$ with an equivalence class of orbifold atlases.
	\begin{itemize}
		\item Every orbifold has a \emph{unique} maximal orbifold atlas $\mathcal{U}$;
		\item Call the pair $\mathcal{X} = (X, \mathcal{U})$ an orbifold.
	\end{itemize}
\end{frame}

\begin{frame}{Examples}
	\begin{itemize}
		\item Setting $\Gamma = \{0\}$, all manifolds are orbifolds (trivially)!
	\end{itemize}
\end{frame}

%------------------------------------------------
\section{Orbibundles}
%------------------------------------------------

\begin{frame}{Tangent Orbibundle}
	\begin{itemize}
		\item $(\widetilde{U}, H, \phi)$ chart for $\mathcal{X} = (X, \mathcal{U})$, action $\rho_{h} : \widetilde{U} \ra \widetilde{U}$,
		\[
		\rho_{h}(\tilde{x}) := h \cdot \tilde{x}, \qquad h \in H.  
		\]
		\item Equip tangent bundle $T\widetilde{U}$ with the $H$-action:
		\[
		h \cdot (\tilde{x}, v) := \left( h \cdot \tilde{x}, d_{\tilde{x}}\rho_{h}v \right).
		\]
		\item Get orbifold charts
		\[
		\left(T\widetilde{U}, H, q\right), \qquad q : T\widetilde{U} \ra T\widetilde{U}/H =: TU.
		\]
		\item Projection $\pi : T\widetilde{U} \ra \widetilde{U}$ is $H$-equivariant
		\[
		\implies |\pi| : TU \ra U \cong \widetilde{U}/H.
		\]
	\end{itemize}
\end{frame}

\begin{frame}{Tangent Orbibundle}
	\begin{itemize}
		\item Let $\phi(\tilde{x}) = x \in U$, and 
		\[
		|\pi|^{-1}(x) = \{ H \cdot (y,v)\ |\ y = \tilde{x} \}.
		\]
	\end{itemize}
	\textcolor{blue}{\underline{Claim:}} $|\pi|^{-1}(x) \cong T_{\tilde{x}}\widetilde{U}/\Gamma_{x} \subseteq TU$.
	\begin{itemize}
		\item
		\begin{itemize}
			\item Suppose $H \cdot (\tilde{x},v) = H \cdot (\tilde{x}, w)$
			\item $\iff$ there exists a $h \in H$ such that $h \cdot (\tilde{x},v) = (\tilde{x},w)$
			\item $\iff h \in H_{\tilde{x}}$ and $d_{\tilde{x}}\rho_{h} v = w$
			\item $\iff H_{\tilde{x}} \cdot v = H_{\tilde{x}} \cdot w$
		\end{itemize}
		\item Hence get a well-defined continuous bijection
		\[
		|\pi|^{-1}(x) \rightarrow T_{\tilde{x}}\widetilde{U}, \qquad H \cdot (\tilde{x}, v) \mapsto \Gamma_{x} \cdot v,
		\]
		with inverse the inclusion
		\[
		T_{\tilde{x}}\widetilde{U}/\Gamma_{x} \hookrightarrow TU.    
		\]
	\end{itemize}
\end{frame}

\begin{frame}{Tangent Orbibundle}
	\begin{itemize}
		\item Glue the orbifold charts $(T\widetilde{U}, H, q) \rightsquigarrow$ \textcolor{red}{\underline{tangent bundle}}, $T\mathcal{X}$, to $\mathcal{X} = (X, \mathfrak{U})$.
	\end{itemize}
	Let $\phi(\tilde{x}) = x$ for the chart $(\widetilde{U}, H, \phi)$:
	\begin{itemize}
		\item \textcolor{red}{\underline{Tangent space}} at $x$ is $T_{\tilde{x}}\widetilde{U}$ with $\Gamma_{x}$-action.
		\item \textcolor{red}{\underline{Tangent cone}} at $x$ is $C_{x}X := T_{\tilde{x}}\widetilde{U}/\Gamma_{x} \cong |\pi|^{-1}(x)$.
	\end{itemize}
\end{frame}

\begin{frame}{Sections of Orbibundles}
	\begin{itemize}
		\item Given a section $\sigma_{\widetilde{U}} : \widetilde{U} \ra T\widetilde{U}$, we get an \textcolor{red}{\underline{invariant section}} $\sigma_{U}$ by averaging,
		\[
		\sigma_{U} := \frac{1}{|\Gamma|}\sum_{\gamma \in \Gamma} \sigma_{\widetilde{U}} \circ \gamma.
		\]
		\item Similarly, if $\w$ is a differential form on $V \subset \phi(\widetilde{u})$,
		\[
		\int_{V} \w = \frac{1}{|\Gamma|}\int_{\phi^{-1}(V)} \w_{\widetilde{U}}.
		\]
		\item Also have de Rham theory, cohomology, etc., for orbifolds.
	\end{itemize}
\end{frame}

\begin{frame}{More Orbibundles}
	\begin{itemize}
		\item From $T\mathcal{X}$, get the cotangent bundle, tensor (symmetric/exterior) powers, etc., by considering each chart and gluing.
		\item \textcolor{blue}{\underline{Def:}} An \textcolor{red}{\underline{orbifold vector bundle}} over $\mathcal{X}$ is:
		\begin{itemize}
			\item An orbifold $\mathcal{L}$ and $\pi : \mathcal{L} \ra \mathcal{X}$;
			\item For $(\widetilde{U}, H, \phi)$ charts for $\mathcal{X}$, corresponding $(\widetilde{V}_{L}, H, \phi_{L})$ for $\mathcal{L}$;
			\item That $\psi : \widetilde{V}_{L} \ra \widetilde{U}$ is a vector bundle, and $H$ acts on $\widetilde{V}_{L}$ by vector bundle automorphisms.
		\end{itemize}
	\end{itemize}
\end{frame}

%------------------------------------------------
\section{Index Theory}
%------------------------------------------------

\begin{frame}{Symplectic Toric Manifolds}
	\textcolor{blue}{\underline{Def:}} A $2n$-dimensional \textcolor{red}{\underline{symplectic toric manifold}} is:
	\begin{itemize}
		\item A compact connected symplectic manifold $(M, \w)$;
		\item With an effective Hamiltonian $T^{n}$-action;
		\item And moment map $\mu: M \ra \RR^{n}$.
	\end{itemize}
\end{frame}

%------------------------------------------------

\begin{frame}{Blocks of Highlighted Text}
    In this slide, some important text will be \alert{highlighted} because it's important. Please, don't abuse it.

    \begin{block}{Block}
        Sample text
    \end{block}

    \begin{alertblock}{Alertblock}
        Sample text in red box
    \end{alertblock}

    \begin{examples}
        Sample text in green box. The title of the block is ``Examples".
    \end{examples}
\end{frame}

%------------------------------------------------

\begin{frame}{Multiple Columns}
    \begin{columns}[c] % The "c" option specifies centered vertical alignment while the "t" option is used for top vertical alignment

        \column{.45\textwidth} % Left column and width
        \textbf{Heading}
        \begin{enumerate}
            \item Statement
            \item Explanation
            \item Example
        \end{enumerate}

        \column{.5\textwidth} % Right column and width
        Lorem ipsum dolor sit amet, consectetur adipiscing elit. Integer lectus nisl, ultricies in feugiat rutrum, porttitor sit amet augue. Aliquam ut tortor mauris. Sed volutpat ante purus, quis accumsan dolor.

    \end{columns}
\end{frame}

%------------------------------------------------
\section{Second Section}
%------------------------------------------------

\begin{frame}{Table}
    \begin{table}
        \begin{tabular}{l l l}
            \toprule
            \textbf{Treatments} & \textbf{Response 1} & \textbf{Response 2} \\
            \midrule
            Treatment 1         & 0.0003262           & 0.562               \\
            Treatment 2         & 0.0015681           & 0.910               \\
            Treatment 3         & 0.0009271           & 0.296               \\
            \bottomrule
        \end{tabular}
        \caption{Table caption}
    \end{table}
\end{frame}

%------------------------------------------------

\begin{frame}{Theorem}
    \begin{theorem}[Mass--energy equivalence]
        $E = mc^2$
    \end{theorem}
\end{frame}

%------------------------------------------------

\begin{frame}{Figure}
    Uncomment the code on this slide to include your own image from the same directory as the template .TeX file.
    %\begin{figure}
    %\includegraphics[width=0.8\linewidth]{test}
    %\end{figure}
\end{frame}

%------------------------------------------------

\begin{frame}[fragile] % Need to use the fragile option when verbatim is used in the slide
    \frametitle{Citation}
    An example of the \verb|\cite| command to cite within the presentation:\\~

    This statement requires citation \cite{p1}.
\end{frame}

%------------------------------------------------

\begin{frame}{References}
    % Beamer does not support BibTeX so references must be inserted manually as below
    \footnotesize{
        \begin{thebibliography}{99}
            \bibitem[Smith, 2012]{p1} John Smith (2012)
            \newblock Title of the publication
            \newblock \emph{Journal Name} 12(3), 45 -- 678.
        \end{thebibliography}
    }
\end{frame}

%------------------------------------------------

\begin{frame}
    \Huge{\centerline{\textbf{The End}}}
\end{frame}

%----------------------------------------------------------------------------------------

\end{document}