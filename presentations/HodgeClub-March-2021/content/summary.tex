\section{Outlook}

\begin{frame}{Motivation}
    \begin{block}{Verlinde Formula}
        \begin{itemize}
            \item Let $\mc{N}$ be the moduli space of stable $\SL_{2}$-bundles over $\Sigma_{2}$; it is isomorphic to $\CC\PP^{3}$ \cite{NR1969}.
            \item Geometric quantisation $\mc{Q}(\mc{N}) := H^{0}(\mc{N}; \mc{L}^{\otimes k})$; its dimension equals the \textbf{Verlinde formula} \cite{verlinde1988, JW1992},
            \begin{equation*}
                \begin{split}
                    \dim \mc{Q}(\mc{N}) = \Ver(k) &= \frac{k^{3}}{6} + k^{2} + \frac{11k}{6} + 1 \\
                    &= \frac{(k+1)(k+2)(k+3)}{3!}.
                \end{split}
            \end{equation*}
            \item Named after Dutch physicist Erik Verlinde, who was working on conformal field theories.
        \end{itemize}
    \end{block}
\end{frame}

\begin{frame}{Integration}
    \begin{block}{Riemann-Roch-Hirzebruch Theorem}
        For toric $X$, lattice point count of $\Delta_{X}$ equals to Euler characteristic,
        \vspace*{-12pt}
        $$ \chi(X) = \int_{X} e^{c_{1}( \mc{O}(k) )}\cdot \Td{(TX)}. $$
        \vspace*{-12pt}
    \end{block}
    \begin{block}{Example}
        For $X = \CC\PP^{3}$, have $e^{c_{1}(\mc{O}(k))} = 1 + kH + \tfrac{k^{2}}{2}H^{2} + \tfrac{k^{3}}{6}H^{3}$, and $\Td{(T\CC\PP^{3})} = 1 + 2H + \tfrac{11}{6}H^{2} + H^{3}$:
        $$ \chi(\CC\PP^{3}) = \int_{\CC\PP^{3}} \big( \tfrac{k^{3}}{6} + k^{2} + \tfrac{11k}{6} + 1 \big) \cdot H^{3} + \ldots = \Ver(k). $$
    \end{block}
\end{frame}

\begin{frame}{Our Direction}
    \begin{block}{Lattice Points}
    \begin{itemize}
        \item So $\Ver(k) = \chi(\CC\PP^{3}) = \#( k\cdot \Delta_{3} \cap \ZZ^{3} )$.
        \item Anything similar for hypertoric manifolds?
    \end{itemize}
    \end{block}
    \begin{block}{Equivariant Verlinde Formula}
        \begin{itemize}
            \item Recently, equivariant Verlinde formula for moduli spaces Higgs bundle, $\mc{M}$, popped up \cite{GP2017}.
            \item $\dim \mc{Q} (\mc{M}) = \infty$, but $\mc{M}$ has a $\CC^{\ast}$-action.
            \item Decompose into $\CC^{\ast}$-weight spaces:
            \vspace*{-6pt}
                $$ \dim \mc{Q} (\mc{M}) = \sum_{n} t^{n} \cdot \dim \mc{Q}_{n}(\mc{M}), $$
            \vspace*{-6pt}
            but now $\dim \mc{Q}_{n}(\mc{M}) < \infty$.
        \end{itemize}
    \end{block}
\end{frame}

\begin{frame}{Localisation}
    \begin{block}{Fixed-Point Formula}
        \vspace*{-12pt}
        $$ \sum_{q \in \Delta} e^{\langle q, \phi \rangle} = \sum_{p \in M^{T}} \frac{ e^{\langle p, \phi \rangle} }{ \prod_{k = 1}^{n} (1- e^{\langle \alpha_{k}^{p}, \phi \rangle}) }, $$
            with edge vectors $\alpha_{k}^{p}$, and $\langle \alpha_{k}^{p}, \phi \rangle \neq 0$, \cite{barvinok1993}.
        \begin{itemize}
            \item Letting $\phi \ra 0$ gets the lattice point count (for Delzant $\Delta$).
        \end{itemize}
    \end{block}
    \begin{block}{Example for $T^{\ast}\CC\PP^{3}$}
        For $(T^{\ast}\CC\PP^{3})_{\epsilon-\text{cut}}$, get
        $$ \frac{(\epsilon + 1)(\epsilon + 2)(\epsilon + 3)}{3!} \cdot \frac{(k + \epsilon +1)(k + \epsilon + 2)(k + \epsilon + 3)}{3!}. $$
        Observe that for $\epsilon = 0$, it becomes $\Ver(k)$.
    \end{block}   
\end{frame}

\begin{frame}{Other Hypertoric Manifolds}
    \begin{block}{Non-Convex Core}
        Want to see what happens for hypertoric manifolds with non-convex cores.
    \end{block}
    \begin{block}{Example}
        \vspace{136pt}
    \end{block}      
\end{frame}