\documentclass{article}


\usepackage{arxiv}

\usepackage[utf8]{inputenc} % allow utf-8 input
\usepackage[T1]{fontenc}    % use 8-bit T1 fonts
\usepackage{hyperref}       % hyperlinks
\usepackage{url}            % simple URL typesetting
\usepackage{booktabs}       % professional-quality tables
\usepackage{amsfonts}       % blackboard math symbols
\usepackage{nicefrac}       % compact symbols for 1/2, etc.
\usepackage{microtype}      % microtypography
\usepackage{lipsum}		% Can be removed after putting your text content
\usepackage{amsmath} 
\usepackage{amssymb}
\usepackage{graphicx}
\usepackage{epstopdf}
\usepackage{url}
\usepackage{setspace}
\usepackage{amsthm}
\usepackage{mathrsfs}
\usepackage{enumitem}
\usepackage{parskip}
\usepackage{IEEEtrantools}
\usepackage{mathtools}
\usepackage{tensor}
\usepackage{yfonts}
\usepackage{dsfont}

\newtheorem{theorem}{Theorem}[section]
\newtheorem{lemma}[theorem]{Lemma}
\newtheorem*{lemma*}{Lemma}
\newtheorem{prop}[theorem]{Proposition}
\newtheorem{corollary}[theorem]{Corollary}
\newtheorem{defn}[theorem]{Definition\rm}
\newtheorem{conjecture}[theorem]{Conjecture}
\newtheorem{remark}{\it Remark\/}
\newtheorem{example}{Example}

\newcommand{\st}{\ensuremath{:}}% such that
\newcommand{\ie}{\emph{i.e.} }
\newcommand{\eg}{\emph{e.g.} }
\newcommand{\cf}{\emph{cf.} }
\newcommand{\ra}{\rightarrow}
\newcommand{\la}{\leftarrow}
\newcommand{\lra}{\longrightarrow}
\newcommand{\lla}{\longleftarrow}
\newcommand{\lbracket}{\left(}
\newcommand{\rbracket}{\right)}


\newcommand{\al}{\alpha}
\newcommand{\w}{\omega}
\newcommand{\W}{\Omega}
\newcommand{\m}{\mu}
\newcommand{\n}{\nu}
\newcommand{\e}{\epsilon}
\newcommand{\K}{K\"ahler }
\newcommand{\HK}{hyperk\"ahler }
\newcommand{\into}{\hookrightarrow}
\newcommand{\PP}{\mathbb{P}}
\newcommand{\RR}{\mathbb{R}}
\newcommand{\CC}{\mathbb{C}}
\newcommand{\QQ}{\mathbb{Q}}
\newcommand{\FF}{\mathbb{F}}
\newcommand{\ZZ}{\mathbb{Z}}
\newcommand{\NN}{\mathbb{N}}
\newcommand{\HH}{\mathbb{H}}
\newcommand{\vp}{\varphi}
\newcommand{\mcA}{\mathcal{A}}
\newcommand{\mcE}{\mathcal{E}}
\newcommand{\mcF}{\mathcal{F}}
\newcommand{\mcG}{\mathcal{G}}
\newcommand{\mcH}{\mathcal{H}}
\newcommand{\mcL}{\mathcal{L}}
\newcommand{\mcO}{\mathcal{O}}
\newcommand{\mfg}{\mathfrak{g}}
\newcommand{\mfh}{\mathfrak{h}}
\newcommand{\mft}{\mathfrak{t}}
\newcommand{\mc}[1]{\mathcal{#1}}
\newcommand{\mf}[1]{\mathfrak{#1}}

\newcommand{\dbar}{\bar{\partial}}
\newcommand{\mrr}{\mu_{\mathbb{R}}}
\newcommand{\mcc}{\mu_{\mathbb{C}}}
\newcommand{\prr}{\phi_{\mathbb{R}}}
\newcommand{\pcc}{\phi_{\mathbb{C}}}

\DeclareMathOperator{\Lie}{\text{Lie}}
\DeclareMathOperator{\Aut}{Aut}
\DeclareMathOperator{\Tr}{Tr}
\DeclareMathOperator{\Image}{Im}
\DeclareMathOperator{\Ad}{Ad}
\DeclareMathOperator{\Diff}{Diff}
\DeclareMathOperator{\Vect}{Vect}
\DeclareMathOperator{\Sympl}{Sympl}
\DeclareMathOperator{\Span}{Span}
\DeclareMathOperator{\ind}{ind}
\DeclareMathOperator{\Td}{Td}
\DeclareMathOperator{\Ch}{Ch}
\DeclareMathOperator{\Ind}{Ind}
\DeclareMathOperator{\pt}{pt}
\DeclareMathOperator{\rk}{rk}

\title{Equivariant Localisation and Fixed-Point Theorems}

\date{12th March 2021}	% Here you can change the date presented in the paper title
%\date{} 					% Or removing it

%\author{
%  David S.~Hippocampus\thanks{Use footnote for providing further
%    information about author (webpage, alternative
%    address)---\emph{not} for acknowledging funding agencies.} \\
%  Department of Computer Science\\
%  Cranberry-Lemon University\\
%  Pittsburgh, PA 15213 \\
%  \texttt{hippo@cs.cranberry-lemon.edu} \\
  %% examples of more authors
%   \And
% Elias D.~Striatum \\
%  Department of Electrical Engineering\\
%  Mount-Sheikh University\\
%  Santa Narimana, Levand \\
%  \texttt{stariate@ee.mount-sheikh.edu} \\
  %% \AND
  %% Coauthor \\
  %% Affiliation \\
  %% Address \\
  %% \texttt{email} \\
  %% \And
  %% Coauthor \\
  %% Affiliation \\
  %% Address \\
  %% \texttt{email} \\
  %% \And
  %% Coauthor \\
  %% Affiliation \\
  %% Address \\
  %% \texttt{email} \\
%}

\begin{document}
\maketitle

\begin{abstract}


\end{abstract}

\section{Introduction}

Often in mathematics, we are tasked with the problem of evaluating an integral over some space, that is trying to evaluate

$$ \int_{M}\w $$

for some space $M$. Depending on the form $\w$, this integral can be related to finding the volume, to finding topological or enumerative invariants, integrating characteristic classes, or computing partition functions of physical systems. Such computations can be difficult, and there are many ways that we can tackle the integral. Two methods are particularly fruitful - that of \emph{localisation} and \emph{symmetry}.

\subsection{Symmetry and Localisation}

By symmetry, we mean that we have a group $G$ acting on $M$, and by identifying orbits we reduce the problem to that over a smaller space, $M/G$. Such an approach comes up in symplectic reduction, gauge theory, and integrable systems.

By localisation, informally this means that we relate global calculations to ones that are local. The Poincaré-Hopf theorem is an example of this, which relates the Euler characteristic of a compact manifold $M$ to the sum of the indices of the zeros of a vector field on it:

$$ \chi(M) = \int_{M} e(TM) = \sum\limits_{\mathbf{V}(p)=0} \Ind(\mathbf{V}). $$

Unsurprisingly, it is often easier to consider a finite set of points rather than the global space $M$. The notion of localisation in algebra is a similar notion, in which we consider a single point (a prime) at a time.

Symmetry and localisation synergise through the Atiyah-Bott fixed-point theorem; in the situation that we have a smooth manifold $M$ together with the action of a compact connected Lie group $G$, then the integral on $M$ localises on the (isolated) fixed-point set $M^{G} := F \subseteq M$ of the $G$-action. If $i : F \hookrightarrow M$ is the inclusion, and $e(\nu_{p})$ is the Euler class of the normal bundle $\nu_{p}$ to the fixed-point $p \in F$, then

$$ \int_{M} \w = \sum\limits_{p \in F} \int_{p} \frac{i^{\ast}\w}{e(\nu_{p})}. $$

What I want to do in this talk is discuss some interesting cases when the fixed-point formula can be applied, and investigate how to use it in these examples. 
Consider the following geometric sum:

\begin{equation*}
	\begin{split}
		\sum\limits_{k=0}^{10000} q^{k} &= 1 + q + q^{2} + \ldots + q^{10001} \\
		&= \left(\frac{1-q}{1-q}\right)\cdot(1 + q + q^{2} + \ldots + q^{10001}) \\
		&= \frac{1 - q^{10001}}{1 - q} \\
		&= \frac{1}{1 - q} + \frac{1 - q^{10000}}{1 - q^{-1}}.
	\end{split}
\end{equation*}

To evaluate the left-hand side, we need to know the value of each term at the 10001 integral points inside of the closed interval $[0, 10000]$, whereas the right-hand side only needs the two terms to be evaluated. So we can say that this sum \emph{localises} at the end points.

\section{Equivariant Cohomology}

Let $G$ be a compact Lie group acting on a toplogical space $M$. If $G$ acts freely on $M$, then the quotient space $M/G$ is usually as nice as the space $M$ is itself; for instance, if $M$ is a manifold then so is $M/G$.

The idea behind an equivariant cohomology group, $H_{G}^{\ast}(M)$, is that the equivariant cohomology groups of $M$ should just be the cohomology groups of $M/G$:

\begin{equation*}
	H_{G}^{\ast}(M) = H^{\ast}(M/G), \qquad \text{when the action is free.}
\end{equation*}

For example, if $G$ acts on itself by left multiplication, then

\begin{equation*}
	H_{G}^{\ast}(G) = H^{\ast}(\pt).
\end{equation*}

However, if the action is not free, then the space $M/G$ might not behave very nicely from a cohomological point of view. Then the idea is that $H_{G}^{\ast}(M)$ should be the ``correct'' subsitute for $H^{\ast}(M/G)$.

\subsection{Classifying Bundles}

As cohomology is unchanged under homotopy equivalence, our guiding idea is that the equivariant cohomology of $M$ should be the ordinary cohomology of $M^{\ast}/G$, where $M^{\ast}$ is some topological space homotopy equivalent to $M$ and on which $G$ acts freely. The standard way of constructing such a space is to take it to be the product $M^{\ast} = M \times E$, where $E$ is some contractible space on which $G$ acts freely. Then the equivariant cohomology groups of $M$ are defined by the recipe

\begin{equation*}
	H_{G}^{\ast}(M) := H^{\ast}\left( (M \times E)/G \right).
\end{equation*}

Note that if $G$ acts freely on $M$ then the projection

\begin{equation*}
	(M \times E)/G \lra M/G
\end{equation*}

is a fibration with typical fibre $E$. Then as $E$ is contractible, we get that

\begin{equation*}
	H_{G}^{\ast}(M) = H^{\ast}\left( (M \times E)/G \right) = H^{\ast}(M/G),
\end{equation*}

so we arrive at the same situation if $G$ acts freely on $M$.

\subsection{The Cartan Model}

Let $M$ be an $n$-dimensional manifold acted on by a Lie group $G$ with with Lie algebra $\mfg$. A \emph{$G$-equivariant differential form} on $M$ is defined to be a polynomial map $\alpha : \mfg \ra \W(M)$ such that

\begin{equation*}
	\alpha(gX) = g \cdot \alpha(X), \qquad \text{for } g \in G.
\end{equation*}

Let $\CC[\mfg]$ denote the algebra of $\CC$-valued polynomial functions on $\mfg$. Then we can view the tensor product

\begin{equation*}
	\CC[\mfg] \otimes \W(M),
\end{equation*}

as the algebra of polynomial maps from $\mfg$ to $\W$. The group $G$ acts on an element $\alpha \in \CC[\mfg] \otimes \W(M)$ by the formula\footnote{$G$ acts on $\W(M)$ by the induced $G$-action on $M$, and on $\mfg$ by the adjoint action.}

\begin{equation*}
	(g \cdot \alpha)(X) := g \cdot \left( \alpha ( g^{-1} \cdot X )  \right), \qquad \text{for all } g \in G, \text{ and } X \in \mfg.
\end{equation*}

Let $\W^{G}(M) = \left( \CC[\mfg] \otimes \W(M) \right)^{G}$ be the subalgebra of $G$-invariant elements; an element $\alpha \in \W^{G}(M)$ thus satisfies $\alpha(g \cdot X) = g\cdot \alpha(M)$, hence is an equivariant differential form. Equip $\CC[\mfg] \otimes \W(M$ with the following $\ZZ$-grading,

\begin{equation*}
	\deg(P \otimes \alpha) := 2\cdot \deg(P) + \deg(\alpha),
\end{equation*}

for the polynomial $P \in \CC[\mfg]$, and $\alpha \in \W(M)$. Define the \emph{equivariant exterior differential}, or \emph{Cartan differential}, $d_{G}$ by

\begin{equation*}
	(d_{G}\alpha)(X) := (d - \imath_{X_{M}}) \alpha(X),
\end{equation*}

where $d :\W^{k}(M) \ra \W^{k+1}(M)$ is the usual de Rham differential, $X_{M}$ is the fundamental vector field of $X \in \mfg$ on $M$, and $\imath_{X} : \W^{k}(M) \ra \W^{k-1}(M)$ is the contraction of $X$ on a differential form.

\begin{prop}
	The Cartan differential $d_{G}$ is closed on $\W_{G}^{\ast}$, \ie $d_{G}^{2} = 0$.
\end{prop}

\begin{proof}
	Let us compute $d_{G}^{2}$ on $\CC[\mfg] \otimes \W^{\ast}(M)$. Some notation: set $\CC[\mfg] \cong \CC[\phi^{1}, \ldots, \phi^{\rk \mfg}]$, for a set of generators $\{\phi^{\mu} \}$ of $\mfg$, then $d_{G}$ can be written as
	
		\begin{equation*}
			d_{G} = d \otimes 1 + i_{\mu} \otimes \phi^{\mu}.
		\end{equation*}

	Then the homotopy formula $\imath_{\mu} d + d \imath_{\mu} = \mcL_{\mu}$, where $\mcL_{\mu}$ is the Lie derivative of $M$ along a vector field generated by $V_{\mu}$, gets us that
	
	\begin{equation*}
		\begin{split}
			d_{G}^{2} &= (d \otimes 1 + \imath_{\mu} \otimes \phi^{\mu}) (d \otimes 1 + \imath_{\mu} \otimes \phi^{\mu}) \\
			&= d^{2} \otimes 1 + \left( (\imath_{\mu}d + d\imath_{\mu})  \otimes \phi^{\mu} \right) + (\imath_{\mu}\imath_{\nu}) \otimes (\phi^{\mu}\phi^{\nu}) \\
			&= (\imath_{\mu}d + d\imath_{\mu})  \otimes \phi^{\mu} \\
			&= \mcL_{\mu} \otimes \phi^{\mu}.
		\end{split}
	\end{equation*}
	
	TODO: FINISH!
\end{proof}

\begin{corollary}
	The space of equivariant differential forms $\W_{G}^{\ast}(M) = \left( \CC[\mfg] \otimes \W^{\ast}(M) \right)^{G}$, equipped with the Cartan differential $d_{G}$ forms a complex, called the \textbf{Cartan complex}:
	
	\begin{equation*}
		\left( \W_{G}^{\ast}(M), d_{G} \right) = \left( (\CC[\mfg] \otimes \W^{\ast}(M)   )^{G}, d_{G} \right).
	\end{equation*}

\end{corollary}

\begin{defn}
	The \textbf{equivariant cohomology} $H_{G}^{\ast}(M)$ of $M$ is the cohomology of the Cartan complex, $(\W_{G}^{\ast}(M), d_{G})$.
\end{defn}

\subsection{Equivariantly Closed Forms}

In the Cartan model, an equivariant two-form

	\begin{equation*}
		\tilde{\w} \in \W_{G}^{2}(M) = \left( \left( \W_{G}^{2}(M) \otimes \CC_{0}[\mfg] \right) \oplus \left( \W^{0}(M) \otimes \CC_{1}[\mfg] \right) \right)^{G},
	\end{equation*} 

can be written as

	\begin{equation*}
		\tilde{\w} = \w - \mu,
	\end{equation*}

where $\w \in \W^{2}(M)$ is a two-form invariant under $G$ and $\mu \in \left( \W^{0}(M) \otimes \mfg \right)^{G}$ can be considered as a $G$-equivariant map,

	\begin{equation*}
		\mu : \mfg \lra \W^{0}(M) = \mc{C}^{\infty}(M),
	\end{equation*}

from the Lie algebra, $\mfg$, to the space of smooth functions on $M$. For each $\xi \in \mfg$, $\mu(\xi)$ is a smooth function on $M$, and this function depends linearly on $\xi$. Therefore, for each $p \in M$, the value $\mu(\xi)(p)$ depends linearly on $\xi$, and thus we can think of $\mu$ as defining a map from $M$ to the dual Lie algebra, $\mfg^{\ast}$:

	\begin{equation*}
		\mu : M \lra \mfg^{\ast}; \qquad \langle \mu(p), \xi \rangle := \mu(\xi)(p).
	\end{equation*}











\subsection{The Berline-Vergne-Atiyah-Bott Fixed Point Theorem}

When a manifold has a torus action, the equivariant localisation formula is a powerful tool for doing calculations in \textbf{ordinary} cohomology, despite being formulated in \textbf{equivariant} cohomology.

\begin{theorem}[\textbf{Atiyah-Bott, Berline-Vergne Theorem}]
	Suppose an $m$-dimensional torus $T$ acts on a compact oriented manifold $M$ with fixed-point set $F := M^{T}$. If $\phi$ is an equivariant closed form on $M$ and $i_{F}: F \hookrightarrow M$ is the inclusion map, then
	\begin{equation*}
		\int_{M} \phi = \int_{F} \frac{i_{F}^{\ast} \phi}{e_{T}(\nu_{F})},
	\end{equation*}
	as elements of $H_{T}^{\ast}(\pt) = \RR[u_{1}, \ldots, u_{l}]$. Here, $\nu_{F}$ is the normal bundle of $F$ in $M$, and $e_{T}$ is the $T$-equivariant Euler class.
\end{theorem}

\section{Examples}

\subsection{Stationary Phase and Duistermaat-Heckman}

Let $M$ be a compact, oriented $2n$-manifold, $f : M \ra \RR$ a function, and $\tau \in \W^{2n}(M)$.

TODO: SEE LORING TU'S BOOK.

\subsection{Riemann-Roch-Hirzebruch Theorem}

For $\mcL \ra M$ a holomorphic line bundle over a complex manifold $M$. The Hirzebruch-Riemann-Roch Theorem then states that the Euler characteristic, $\chi(M; \mcL)$, is equal to the characteristic number

\begin{equation*}
	\chi(M; \mcL) =	\int_{M} e^{c_{1}(\mcL)}\, \Td(TM).
\end{equation*}

Here, $c_{1}(\mcL)$ is the \emph{1st Chern class} of $\mcL$, and $\Td(M)$ is the \emph{Todd class} of the complex vector bundle $TM \ra M$.







\bibliographystyle{unsrt}  
%\bibliography{references}  %%% Remove comment to use the external .bib file (using bibtex).
%%% and comment out the ``thebibliography'' section.


%%% Comment out this section when you \bibliography{references} is enabled.

\end{document}
