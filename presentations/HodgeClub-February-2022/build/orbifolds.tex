\section{Orbifolds}

\begin{frame}{Charts}
    \textcolor{blue}{\underline{Def:}} For $X$ a paracompact Hausdorff space. An \textcolor{red}{orbifold chart} is a triple $(\widetilde{U}, \Gamma, \phi)$:
        \begin{itemize}
            \item $\widetilde{U} \subseteq \RR^{n}$ is open, and $\{0\} \in \widetilde{U}$;
            \item $\Gamma$ finite group, effectively acting on $\widetilde{U}$;
            \item $\phi : \widetilde{U} \ra U$ continuous onto $U \subset X$, \newline such that $\phi \circ \gamma = \phi$, for all $\gamma \in \Gamma$;
            \item Induced $\phi : \widetilde{U}/\Gamma \ra U$ is a homeomorphism.\newline
        \end{itemize}
                
        For two charts, $(\widetilde{U}_{1}, \Gamma_{1}, \phi_{1})$, and $(\widetilde{U}_{2}, \Gamma_{2}, \phi_{2})$: \newline
        
        \textcolor{blue}{\underline{Def:}} An \textcolor{red}{embedding} between them is a smooth embedding $\lambda : \widetilde{U}_{1} \ra \widetilde{U}_{2}$ such that
        \[
            \phi_{2} \circ \lambda = \phi_{1}.    
        \]
\end{frame}



\begin{frame}{Atlases}
    \textcolor{blue}{\underline{Def:}} An \textcolor{red}{orbifold atlas} on $X$ is a family $\mathcal{U} = \{\widetilde{U}_{i}, \Gamma_{i}, \phi_{i}\}$ of charts such that
        \begin{itemize}
            \item $X = \bigcup_{i} \phi_{i}(\widetilde{U}_{i})$;
            \item Two charts $(\widetilde{U}_{1}, \Gamma_{1}, \phi_{1})$, and $(\widetilde{U}_{2}, \Gamma_{2}, \phi_{2})$, with $U_{1} = \phi_{1}(\widetilde{U}_{1})$ and $U_{2} = \phi_{2}(\widetilde{U}_{2})$, and a point $x \in U_{1} \cap U_{3}$, there is an open neighbourhood $U_{3}$ of $x$, and a chart $(\widetilde{U}_{3}, \Gamma_{3}, \phi_{3})$ such that there are embeddings
            \[
                \lambda_{31} : \widetilde{U}_{3} \ra \widetilde{U}_{1}, \quad \lambda_{32} : \widetilde{U}_{3} \ra \widetilde{U}_{2}.
            \]
        \end{itemize}
    \textcolor{blue}{\underline{Note:}} Similarly for manifolds, an orbifold atlas $\mathcal{U}$ has a \textcolor{red}{refinement} $\mathcal{V}$ if each chart $\widetilde{U} \overset{\text{injects}}{\longhookrightarrow} \widetilde{V}$.
\end{frame}

\begin{frame}{Orbifolds}
    \textcolor{blue}{\underline{Def:}} An \textcolor{red}{orbifold} is a paracompact Hausdorff space $X$ with an equivalence class of orbifold atlases.
    \begin{itemize}
        \item Every orbifold has a \emph{unique} maximal orbifold atlas $\mathcal{U}$;
        \item Call the pair $\mathcal{X} = (X, \mathcal{U})$ an orbifold.
    \end{itemize}
\end{frame}

\begin{frame}{Examples}
    \textcolor{blue}{\underline{Examples of orbifolds:}}
    \begin{itemize}
        \item Any manifold $X$ is an orbifold with trivial $\Gamma$.
        \item Let $T^{4} = (\RR/\ZZ)^{4}$, and introduce the $(\ZZ/2\ZZ)$-action:
        \[
            (t_{1}, t_{2}, t_{3}, t_{4}) \mapsto (t_{1}^{-1}, t_{2}^{-1}, t_{3}^{-1}, t_{4}^{-1}).
        \]
        $X := T^{4}/\ZZ_{2}$ quotient is a \textcolor{red}{Kummer surface}, with $|X^{\ZZ_{2}}| = 16$.
    \end{itemize}
\end{frame}

\begin{frame}{Examples}
    \begin{itemize}
        \item A \textcolor{red}{``mirror''}, i.e. $\RR^{2}/(x \sim -x) \cong \{(x,y) \in \RR^{2}\ |\ x \geq 0\}$.
        \item A \textcolor{red}{``barber shop''}, i.e. $\RR^{2}$ quotiented by the relations
        \[
            \{(x \sim -x),\ (1 + x \sim 1 - x)\} \cong (\ZZ/2\ZZ)^{2}.
        \]
        $\rightsquigarrow \RR^{2} / (\ZZ/2\ZZ)^{2} \cong \{(x,y) \in \RR^{2}\ |\ 0 \leq x \leq 1 \}$.
    \end{itemize}
\end{frame}

\begin{frame}{Examples}
    \begin{itemize}
        \item The \textcolor{red}{``teardrop''}; let $(z_{1}, z_{2}) \in \CC^{2} - \{0\}$ and $t \in \CC^{\ast}$.
        \[
            t \cdot (z_{1}, z_{2}) = (tz_{1}, t^{p}z_{2}), \quad p \in \NN.
        \]
        Quotient $\left(\CC^{2} - \{0\}\right)/\CC^{\ast}$ topologically is $\CC\PP^{1}$.\\
                
        \item Fixed-points are $[z_{1}:0]$ and $[0:z_{2}]$.
        \begin{itemize}
            \item $[z_{1}:0]$ is smooth;
            \item $[0:z_{2}]$ is singular, with $\Gamma \cong \ZZ_{p}$.
        \end{itemize}
    \end{itemize}
\end{frame}

\begin{frame}{Examples}
    \begin{itemize}
        \item The \textcolor{red}{``spindle''}; as before, with,
        \[
            t \cdot (z_{1}, z_{2}) = (t^{p}z_{1}, t^{q}z_{2}), \quad \gcd(p,q) = 1.
        \]

        \begin{itemize}
            \item $[z_{1}:0]$ has $\Gamma \cong \ZZ_{p}$;
            \item $[0:z_{2}]$ has $\Gamma \cong \ZZ_{q}$.
        \end{itemize}
    \end{itemize}
\end{frame}