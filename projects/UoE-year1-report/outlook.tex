\chapter{Outlook}

In this report we have discussed the construction of a toric symplectic manifold if given a Delzant polytope, as well as the opposite direction if one is provided with a toric symplectic manifold, by means of the AGS convexity theorem. This allowed us to then gently introduce the quaternionic analogue to a toric manifold, which are aptly named \HK analogues and which are special cases of the more general toric \HK varieties.

We then discussed the author's original work, which was a discussion into how one can compactify a \HK analogue by the means of the residual $\CC^{\ast}$-action on the cotangent fibres. The cases for $T^{\ast}\CC\PP^{2}$ and $T^{\ast}\CC\PP^{3}$ were mostly discussed: the first because its associated hyperplane arrangement is in $(\RR^{2})^{\ast}$, and thus easy to visualise; and the latter was considered with foresight to geometric quantisation and the (equivariant) Verlinde formula.

It should not be too much of a surprise however that the number of integral lattice points in the moment polytope $k\Delta \subset (\RR^{3})^{\ast}$ to $\CC\PP^{3} = \mu^{-1}(k)/S^{1}$ coincided with the Verlinde formula for $\mc{M}_{\text{flat}}(\Sigma_{2};SU(2))$, since the two spaces can be identified \cite{NR69}, \ie
$$
	\CC\PP^{3} \cong \mc{M}_{\text{flat}}(\Sigma_{2};SU(2)).
$$
Finally, we now ask what $\mc{Q}(\mf{M})$ is, or rather what the dimension is for the weight spaces of the $\CC^{\ast}$-action, for any general hypertoric variety/analogue to a \K $\mf{X}$. The case for $T^{\ast}\CC\PP^{3}$ will serve as our toy model, since then
$$
	T^{\ast}\CC\PP^{3} \cong \mc{M}_{\text{Higgs}}(\Sigma_{2};SU(2)),
$$
where the right-hand side is now the moduli space of $SU(2)$-Higgs bundles over a genus 2 Riemann surface $\Sigma_{2}$. The reason that this will serve as our toy model is that recently the ``\emph{equivariant Verlinde formula}'' has been defined for the Higgs bundle moduli \cite{GP17}, and the author hopes that the dimension of the $\CC^{\ast}$-weight spaces should coincide with that of the equivariant Verlinde formula.