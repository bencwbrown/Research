\chapter{Geometric Quantisation}

\section{Geometric Quantisation}

The mathematical procedure of quantisation associates to a symplectic manifold $(M,\w)$ a Hilbert space $\mc{Q}(M)$. The motivation behind this procedure comes from physics at the quantum level, where the states of a physical system are represented by the rays in a Hilbert space $\mc{H}$ and the observables by a collection $\mc{O}$ of symmetric operators on $\mc{H}$, whilst on the other hand the classical state space is a symplectic manifold $(M,\w)$ and the observables are smooth functions on $M$. As a question, quantisation asks whether given such $(M,\w)$ is it possible to reconstruct $\mc{H}$ and $\mc{O}$, and as a procedure it represents the possible ways one may try to answer this question.

The procedure is governed by Dirac's general principals of quantum mechanics; that the canonical transformations of $M$ generated by the classical observables should correspond to the unitary transformations of $\mc{H}$ generated by the quantum observables, and Poisson brackets of classical observables should correspond to commutators of quantum observables.

\subsection{Pre-Quantisation}

Consider a manifold $M$ along with a closed two-form $\w$.

\begin{defn}
	A \emph{pre-quantisation line bundle} for $(M,\w)$ is a complex line bundle $\mc{L}$ whose curvature class is the cohomology class $[\w]$. Equivalently, the image of its first Chern class $c_{1}(\mc{L})$ maps to $\tfrac{1}{2\pi}[\w]$ under the natural inclusion homomorphism
	\begin{equation*}
		i: H^{2}(M;\ZZ) \hookrightarrow H^{2}(M;\RR).
	\end{equation*}
\end{defn}

Since complex line bundles are determined by $H^{2}(M;\ZZ)$ via the map $\mc{L} \mapsto c_{1}(\mc{L}) \in H^{2}(M;\ZZ)$, it follows that the manifold $(M,\w)$ is pre-quantisable if and only if $\tfrac{1}{2\pi}[\w]$ is integral, \ie $\tfrac{1}{2\pi}[\w] \in H^{2}(M;\ZZ)$ originally.

The statement that $\tfrac{1}{2\pi}[\w]$ must be integral can be interpreted geometrically; it is the curvature form of a connection of a principal $U(1)$-bundle over $M$, whose first Chern class is $\tfrac{1}{2\pi}[\w]$.

\begin{defn}
	A \emph{pre-quantisation} of $(M,\w)$ is a Hermitian line bundle $(\mc{L}, \langle\ ,\ \rangle)$ equipped with a Hermitian connection $\nabla$ whose curvature is $\w$. Thus, if we are provided with a pre-quantisation $(\mc{L}, \langle\ ,\ \rangle, \nabla)$ for a symplectic manifold $(M,\w)$, then $\mc{L}$ is a pre-quantisation line bundle on $(M,\w)$.
\end{defn}

Provided that a pre-quantisation of $(M,\w)$ exists, we say that $M$ is \emph{pre-quantisable}, and the \emph{quantisation space} $\mc{Q}(M)$ of $M$ is constructed from sections of the pre-quantisation line bundle $\mc{L} \rightarrow M$.

However the space of such sections is too big in that it does not satisfy the ``minimality condition'' of Dirac's axioms. Hence 

\begin{defn}
	A \emph{polarisation} of $M$ is an integrable sub-bundle $F$ of the complexified tangent bundle $TM\otimes \CC$ of $M$ such that at each point $p \in M$, $F_{p}$ is a complex Lagrangian subspace of the complex symplectic space $T_{p}M \otimes \CC$. Given a polarisation of $M$, a local section $s:(U \subset M) \rightarrow \mc{L}$
\end{defn}

































\newpage

\section{Moduli Spaces of Vector Bundles over a Riemann Surface}

Let $M$ be a connect compact Riemann surface. 
















































\newpage

\section{The Verlinde Formula}
