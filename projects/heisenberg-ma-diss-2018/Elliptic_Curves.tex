\section{Elliptic Curves over $\CC$}

The aim of this section is to introduce elliptic curves over $\CC$ and to make firm the link between them and complex tori, which serves as our motivation to study theta functions and the elliptic curves they give rise to in the subsequent sections. To avoid unnecessary obfuscation from the point we are trying to make here, we will quote most of the results here without proof, unless such a proof offers an enlightening discussion to our theory. That said, most of this section follows \cite{Silverman_2009} and \cite{Hart_1977}.

\subsection{The Theorem of Riemann-Roch}

In this section, we shall use the word \emph{curve} to mean a complete, non-singular curve over the field $\CC$. For such a curve, we define the \emph{divisor group of $C$}, denoted by $\Div(C)$, to be the free abelian group generated by points of $C$. So a divisor is a formal sum
\begin{equation*}
	D = \sum_{P\in C} n_{P}[P]
\end{equation*}
with $n_{P} \in \ZZ$, such that $n_{P} = 0$ for all but finitely many points $P \in \CC$, and its \emph{degree} is $\deg D := \sum n_{P}$. If we assume the curve $C$ is smooth, and let $f \in \CC(C)^{\ast}$, then we can associate to $f$ the divisor $\princdiv(f)$ given by
\begin{equation*}
	\princdiv(f) := \sum_{P\in C}\ord_{P}(f)[P],
\end{equation*}
and we call any divisor $D \in \Div(C)$, such that $D = \princdiv(f)$ for some $f \in \CC(C)^{\ast}$, a \emph{principal divisor}. The set of all principal divisors form a subgroup of $\Div(C)$, which we denote $\Prin(C)$. Any two divisors are \emph{linearly equivalent}, written $D_{1} \sim D_{2}$, if $D_{1} - D_{2}$ is principal. The \emph{Picard group} of $C$, denoted $\Pic(C)$, is defined as the quotient of $\Div(C)$ by the subgroup $\Prin(C)$, that is
	\begin{equation*}
		\Pic(C) := \Div(C)/\Prin(C).
	\end{equation*}
A divisor $D = \sum n_{P}[P]$ on $C$ is \emph{effective} if $n_{P} \geq 0$ for all $P \in C$, and for two divisors $D_{1}, D_{2}$, we denote $D_{1} \geq D_{2}$ if their difference $D_{1} - D_{2}$ is effective. The \emph{space of (meromorphic) differential forms} on $C$ is denoted by $\Omega_{C}$, and to each $\omega \in \Omega_{C}$ we can associate to it the divisor
\begin{equation*}
	\princdiv(\omega) = \sum_{P\in C}\ord_{P}(\omega)[P] \in \Div(C).
\end{equation*}
For any non-zero $\omega \in \Omega_{C}$, and divisor in the class of the image of $\princdiv(\omega)$ in $\Pic(C)$ is called a \emph{canonical divisor}, and is denoted $K_{C}$.\\

\begin{defn}[\cite{Silverman_2009}]
	The \emph{Riemann-Roch} space of a divisor $D$ on a curve $C$ is the $\CC$-vector space
	\begin{equation*}
		\mcL(D) := \{f \in \CC(C)^{\ast} \st \princdiv(f) + D \geq 0\}\cup \{0\}.
	\end{equation*}
\end{defn}
We have the following proposition:\\
\begin{prop}[\cite{Hart_1977}]
	\label{RR_prop_1}
	Let $D \in \Div(C)$.
	\begin{enumerate}
		\item 
		If $\deg D < 0$, then
		\begin{equation*}
			\mcL(D) = \{0\} \qquad \text{and}\qquad l(D) = 0.
		\end{equation*}
		\item
		$\mcL(D)$ is a finite dimensional $\CC$-vector space.
		\item If $D^{\prime} \in \Div(C)$ is linearly equivalent to $D$, then
		\begin{equation*}
			\mcL(D) \cong \mcL(D^{\prime}),\qquad\text{and so}\qquad l(D) = l(D^{\prime}).
		\end{equation*}
	\end{enumerate}
\end{prop}

\begin{theorem}[Riemann-Roch, \cite{Hart_1977}]
	\label{riemann_roch}
	Let $C$ be a smooth curve and let $K_{C}$ be a canonical divisor of $C$. There is an integer $g \geq 0$, called the \emph{genus of} $C$, such that for every $D \in \Div(C)$,
	\begin{equation*}
		l(D) - l(K_{C} - D) = \deg D - g + 1.\\
	\end{equation*}
\end{theorem}

\begin{cor}[\cite{Hart_1977}]
	\begin{enumerate}
		\label{riemann_roch_cor}
		\item $l(K_{C}) = g$.
		\item $\deg K_{C} = 2g - 2$.
		\item If $\deg D > 2g-2$, then $l(D) = \deg D - g + 1$.\\
	\end{enumerate}
\end{cor}

%\begin{prop}\cite{Silverman_2009}
%	\label{curve_points_same}
%	Let $C$ be a smooth curve with genus $g \geq 1$, and let $P, Q \in C$. If $[P] \sim [Q]$, then $P = Q$.
%\end{prop}

%\begin{proof}
%	By assumption, there is a $f \in \CC(C)^{\ast}$ such that $\princdiv(f) = [P] - [Q]$. Suppose that $P \neq Q$. For any $n \geq 0$, the function $f^{n}$ has a pole or order $n$ at $Q$ and $\princdiv(f^{n}) = n[P] - n[Q]$, so in particular $f^{n} \in \mcL(n[Q])$. Since $\deg((2g-1)[Q]) = 2g - 1 > 2g-2$, Corollary \ref{riemann_roch_cor} states that $l((2g-1)[Q]) = g$. However the set $S = \{1, f, f^{2},\ldots f^{2g-1}\}$ are linearly independent in $\mcL((2g-1)[Q])$ since each element in $S$ has a pole at $Q$ of different orders. Moreover $|S| = 2g$ whereas $l((2g-1)[Q]) = g$ which is a contradiction, so we have that $P = Q$.
%\end{proof}

\subsection{Elliptic Curves as Complex Tori}

We now will apply these results to elliptic curves, namely:\\

\begin{defn}[\cite{Silverman_2009}]
	An elliptic curve $E$ is a smooth curve of genus one, with a marked point $O \in E$.
\end{defn}

As a consequence of the Riemann-Roch Theorem \ref{riemann_roch}, any elliptic curve $E$ can be written as a Weierstrass equation in Legendre form:\\

\begin{theorem}[\cite{Hart_1977}]
	Any elliptic curve $E$ can be written in Legendre form, that is
	\begin{equation}
		\label{legendre}
		E \equiv E_{\lambda}: y^{2} = x(x-1)(x-\lambda)
	\end{equation}
	for some $\lambda \in \A^{1}\setminus\{0,1\}$.
\end{theorem}

\begin{proof}
	By assumption, $E$ has a marked point $O$ to which we can associate the divisor $D = [O]$. Then by the Riemann-Roch Theorem \ref{riemann_roch} and Proposition \ref{RR_prop_1}.1, we see that $l(nD) = n$ for all $n \geq 0$, since the genus of $E$ is 1 and $\deg K_{E} = 0$ by Corollary \ref{riemann_roch_cor}.3. Now $l(0) = 1$ and $\mcL(0)$ consists of holomorphic functions without any poles, but the only holomorphic functions on $E$ are necessarily constant, so  $\mcL(0) = \CC$. When $n\geq 1$, we have certain special cases:
	\begin{enumerate}
		\item[$n=1$:] We have $l(D) = 1$. But $\mcL(D)$ definitely contains the constant functions which have no poles, so $\mcL(0) \subseteq \mcL(D)$. Moreover as $l(0) = l(D)$, this shows that $E$ has no functions with just a simple pole, and that $\mcL(D) \cong \CC$.
		\item [$n=2$:] Now $l(2D) = 2$, so we take $\{1,x\}$ to be a basis for $\mcL(2D)$, i.e. $x$ has a double pole at $O$.
		\item [$n = 3$:] Here $l(3D) = 3$, so we take $\{1,x,y\}$ to be basis for $\mcL(3D)$, where $y$ has a triple pole at $O$.
		\item [$n = 4,5$:]  When $l(4D) = 4$, we have that $\{1, x, y, x^{2}\}$ provides a basis for $\mcL(4D)$. Similarly when $l(5D) = 5$, $\{1, x, y, x^{2}, xy\}$ proves a basis for $\mcL(5D)$.
		\item [$n = 6$:] Now the game changes; $l(6D) = 6$ but there are seven functions, $1, x, y, x^{2}, xy, x^{3}, y^{2}$ that belong to $\mcL(6D)$. It follows that there must be a linear relation between them:
	\end{enumerate}
	\begin{equation}
	\label{linear_dependence}
		a_{1}y^{2} + a_{2}xy + a_{3}y = a_{4}x^{3} + a_{5}x^{2} + a_{6}x + a_{7},
	\end{equation}
	for $a_{1}, \ldots, a_{7} \in \CC$. Moreover, the coefficients $a_{1}$ and $a_{4}$ must be non-zero, since the functions $y^{2}$ and $x^{3}$ both have a six-fold pole at $O$ and no other linear combination of the functions can provide that. As $\Char(\CC) = 0$, by completing the square in (\ref{linear_dependence}) it becomes an equation of the form $y^{2} = f(x)$, where $f(x)$ is a cubic polynomial. Furthermore as $\CC$ is algebraically closed, by considering the roots of $f(x)$ equation (\ref{linear_dependence}) can be transformed to a polynomial in the form
	\begin{equation*}
		E_{\lambda}: y^{2} = x(x-1)(x-\lambda),
	\end{equation*}
	for some $\lambda \in \A^{1}\setminus\{0,1\}$.
\end{proof}

The assumption that $E$ is non-singular asserts that $\lambda \not\in \{0,1\}$, and therefore the polynomial $f(x) = 0$ has distinct roots. Therefore the partial derivatives of $y^{2} - f(x)$ do not vanish anywhere on $E$. From the identity $y^{2} = f(x)$, we have the identity
\begin{equation*}
	2y dy = f^{\prime}(x)dx.
\end{equation*}
Now $dx/y$ is a holomorphic 1-form away from the points where $y = 0$, and in punctured neighbourhoods of such points we can instead write
\begin{equation*}
	\frac{dx}{y} = 2\frac{dy}{f^{\prime}(x)},
\end{equation*}
since $f^{\prime}(x)$ does not vanish since $f(x)$ has only simple roots. It follows that $dx/y$ extends to a holomorphic 1-form $\omega$ on $E\setminus \{y = 0\}$. In fact, $\omega$ extends to a holomorphic 1-form on the whole of $E$, \cite{Silverman_2009}. The natural map
\begin{equation*}
	E(\CC) \longrightarrow \PP^{1},\qquad (x,y)\longmapsto x,
\end{equation*}
is a double cover, ramified precisely over the four points $0,1,\lambda,\infty\in\PP^{1}$. To investigate the nature of such a map, let us for now consider instead the map
\begin{equation*}
	E(\CC) \rightarrow \CC,\qquad P \longmapsto \int_{O}^{P}\omega,
\end{equation*}
where the integral is along some path connecting $O$ to $P$. This map however if not well-defined, since it depends on the choice of path connecting $O$ to $P$. If $P = (x,y) \in E(\CC)$, we can alternatively view the map as happening in $\PP^{1}$: we are trying to compute the complex line integral
\begin{equation*}
	P = (x,y) \longmapsto \int_{\infty}^{x}\frac{dt}{\sqrt{t(t-1)(t-\lambda)}}.
\end{equation*}
This integral is path-dependent due to the presence of the square root in the denominator which is not single valued, so really we have two copies of $\PP^{1}$ to consider. However what we can do is make branch cuts, say connecting $\infty$ to $0$ and $1$ to $\lambda$, on each copy of $\PP^{1}$ and glue them together. This way, away from these branch cuts we can choose one branch value of the square root. This construction of course topologically identifies the resulting elliptic curve as a torus, by realising that $\PP^{1}$ is topologically a 2-sphere.

Returning back to the map
\begin{equation*}
E(\CC) \rightarrow \CC,\qquad P \longmapsto \int_{O}^{P}\omega,
\end{equation*}
the path-dependence from the multi-valuedness of the square root can now be explained from integrating across the branch cuts in $\PP^{1}$. The gluing of the two branch cuts on each copy of $\PP^{1}$ gives rise to two non-contractible loops on the torus; let us label them $\alpha$ and $\beta$. We then obtain two complex numbers, the \emph{periods} of $E$, given by
\begin{equation*}
	\omega_{1} = \int_{\alpha}\omega,\qquad \omega_{2} = \int_{\beta}\omega.
\end{equation*}
Moreover the paths $\alpha$ and $\beta$ generate the first homology group of the associated torus, or equivalently $H_{1}(E,\ZZ)$. Hence any two paths from $O$ to $P$ differ by a path homologous to $n_{1}\alpha + n_{2}\beta$ for some integers $n_{1}, n_{2}\in \ZZ$. Thus the integral $\int_{O}^{P}\omega$ is well-defined up to the addition of the norm $n_{1}\omega_{1} + n_{2}\omega_{2}$, suggesting that we look at the set
\begin{equation*}
	\Lambda = \{n_{1}\omega_{1} + n_{2}\omega_{2}\st n_{1}, n_{2}\in \ZZ\}.
\end{equation*}
After this discussion, we now have a well-defined map
\begin{equation*}
	F: E(\CC) \longrightarrow \CC/\Lambda,\qquad P \longmapsto \int_{O}^{P}\omega\ (\text{mod }\Lambda).
\end{equation*}
Further, the $\Lambda$ is clearly a subgroup of $\CC$, so the quotient group $\CC/\Lambda$ is a group. Moreover as $\omega$ is translation invariant, $F$ can be verified to be a group homomorphism:\\
\begin{equation*}
	\int_{O}^{P+Q}\omega \equiv \int_{O}^{P}\omega + \int_{P}^{P+Q}\omega \equiv \int_{O}^{P}\omega + \int_{O}^{Q}\tau_{P}^{\ast}\omega \equiv \int_{O}^{P}\omega + \int_{O}^{Q}\omega\quad(\text{mod }\Lambda).
\end{equation*}
\\
\begin{defn}[\cite{Silverman_2009}]
	A \emph{lattice} $\Lambda$ in the complex numbers $\CC$ is a discrete subgroup of the form $\Lambda = \ZZ\omega_{1} + \ZZ \omega_{2}$, where $\omega_{1}$ and $\omega_{2}$ are linearly independent over $\RR$. A \emph{complex torus} is a quotient group $\CC/\Lambda$ of the complex plane by a lattice, with the projection $\pi:\CC \rightarrow \CC/\Lambda$.
\end{defn}

Hence we see that we have very nearly shown that $E(\CC)$ is admits a complex analytic group homomorphism to the complex torus $T = \CC/\Lambda$, provided that the periods $\omega_{1}$ and $\omega_{2}$ are linearly independent over $\RR$. It turns out that the periods arising from this construction are linearly independent over $\RR$, but we require some more machinery before addressing this.

\subsection{Complex Tori as Elliptic Curves}

This section will be dedicated to the inverse problem to the previous section, that is, given a lattice $\Lambda \subset \CC$, how can one construct an elliptic curve? The answer is via elliptic functions:\\

\begin{defn}\cite{Silverman_2009}
	An \emph{elliptic function (relative to the lattice $\Lambda$)} is a meromorphic function $f(z)$ on $\CC$ that satisfies
	\begin{equation*}
		f(z + \omega) = f(z)\qquad\text{for all } z \in \CC\text{ and all } \omega \in \Lambda.
	\end{equation*}
\end{defn}
Denote by $\CC(\Lambda)$ the set of all elliptic functions relative to $\Lambda$. It can be shown that $\CC(\Lambda)$ is in fact a field, \cite{Silverman_2009}.\\

\begin{defn}[\cite{Silverman_2009}]
	A \emph{fundamental parallelogram} $\Pi$ for $\Lambda$ is a set of the form
	\begin{equation*}
		\Pi = \{ a + t_{1}\omega_{1} + t_{2}\omega_{2}\st 0 \leq t_{1}, t_{2} <1\},
	\end{equation*}
	where $a \in \CC$ and $\{\omega_{1}, \omega_{2}\}$ is a basis for $\Lambda$.
\end{defn}

Here we prove the complex analytic analogue that a meromorphic function with no poles is constant:\\

\begin{prop}[\cite{Silverman_2009}]
	A holomorphic elliptic function is necessarily constant. Similarly, an elliptic function with no zeros is constant.
\end{prop}

\begin{proof}
	Suppose that $f(z)\in \CC(\Lambda)$ is holomorphic, and let $D$ be a fundamental parallelogram for $\Lambda$. The periodicity of $f$ implies that
	\begin{equation*}
		\sup_{z\in\CC}|f(z)| = \sup_{z \in \bar{D}}|f(z)|.
	\end{equation*}
	The set $\bar{D}$ is compact, so $|f(z)|$ is bounded on $\bar{D}$. Therefore $f$ is in fact a bounded entire function, so Liouville's theorem tells us that $f$ is constant. This proves the first statement. Finally, if $f$ has no zeros, then $1/f$ has no poles and the previous argument applies.
\end{proof}

To circumvent this problem, we must introduce meromorphic functions with poles.\\

\begin{defn}[\cite{Silverman_2009}]
	Let $\Lambda \subset \CC$ be a lattice. The \emph{Weierstrass $\wp$-function (relative to $\Lambda$)} is defined by the series
	\begin{equation*}
		\wp(z;\Lambda) = \frac{1}{z^{2}} + \sum_{\omega\in \Lambda\setminus\{0\}}\bigg(\frac{1}{(z-\omega)^{2}} - \frac{1}{\omega^{2}}\bigg).
	\end{equation*}
	The \emph{Eisenstein series of weight $2k$ (for $\Lambda$)} is the series
	\begin{equation*}
		G_{2k}(\Lambda) = \sum_{\omega\in\Lambda\setminus\{0\}}\omega^{-2k}.
	\end{equation*}
\end{defn}

We state the following theorem without proof:\\

\begin{theorem}[\cite{Silverman_2009}]
	Let $\Lambda \subset \CC$ be a lattice.
	\begin{enumerate}
		\item[(a)] The Eisenstein series $G_{2k}(\Lambda)$ is absolutely convergent for all $k > 1$.
		\item[(b)] The series defining the Weierstrass $\wp$-function converges absolutely and uniformly on every compact subset of $\CC \setminus\Lambda$. The series defines a meromorphic function on $\CC$ having a double pole with residue $0$ at each lattice point and no other poles.
		\item[(c)] The Weierstrass $\wp$-function is an even elliptic function.
		\item[(d)] For all $z \in \CC\setminus\Lambda$, the Weierstrass $\wp$-function and its derivative satisfy the relation
		\begin{equation}
		\label{weierstrass_legendre}
			\wp^{\prime}(z)^{2} = 4\wp(z)^{3} - 60G_{4}\wp(z) - 140G_{6}.
		\end{equation}
	\end{enumerate}
\end{theorem}

With all this at hand, we can finally make the connection between elliptic curves over $\CC$ and complex tori;\\

\begin{prop}[\cite{Silverman_2009}]
	Let $E/\CC$ be an elliptic curve with Weierstrass coordinate functions $x$ and $y$.
	\begin{enumerate}
		\item[(a)] Let $\alpha$ and $\beta$ be closed paths on $E(\CC)$ that form a basis for $H_{1}(E,\ZZ)$. Then the periods
		\begin{equation*}
			\omega_{1} = \int_{\alpha}\frac{dx}{y},\quad\text{and}\quad\omega_{2} = \int_{\beta}\frac{dx}{y}
		\end{equation*}
		are $\RR$-linear independent, and hence form a lattice $\Lambda = \omega_{1}\ZZ + \omega_{2}\ZZ$.
		\item[(b)] Let $\Lambda$ be the lattice generated by $\omega_{1}$ and $\omega_{2}$. Then the map
		\begin{equation*}
			F:E(\CC) \longrightarrow \CC/\Lambda,\qquad F(P) = \int_{O}^{P}\frac{dx}{y}\ (\text{mod }\Lambda),
		\end{equation*}
		is a complex analytic isomorphism of Lie groups.
	\end{enumerate}
\end{prop}

\begin{proof}
	\begin{enumerate}
		\item[(a)] There exists some lattice $\Lambda_{1}$ such that the map
		\begin{equation*}
			\phi_{1}:\CC/\Lambda_{1}\longrightarrow E(\CC),\qquad \phi_{1}(z) = [\wp(z;\Lambda_{1}): \wp^{\prime}(z;\lambda_{1}):1],
		\end{equation*}
		is a complex analytic isomorphism. It follows that $\phi_{1}^{-1}\circ\alpha$ and $\phi_{1}^{-1}\circ\beta$ are a basis for $H_{1}(\CC/\Lambda_{1},\ZZ)$, where we view $\alpha$ and $\beta$ as maps $\alpha,\beta:S^{1}\rightarrow E(\CC)$. We observe that $H_{1}(\CC/\Lambda,\ZZ)$ is natural isomorphic to the lattice $\Lambda_{1}$ via the map $\gamma\mapsto\int_{\lambda}dz$, while the differential $dx/y$ on $E$ pulls back to
		\begin{equation*}
			\phi_{1}^{\ast}\bigg(\frac{dx}{y}\bigg) = \frac{d\wp(z)}{\wp^{\prime}(z)} = dz\qquad\text{on } \CC/\Lambda_{1}.
		\end{equation*}
		Therefore the periods
		\begin{equation*}
			\omega_{1} = \int_{\alpha}\frac{dx}{y} = \int_{\phi_{1}^{-1}\circ\alpha}dz\quad\text{and}\quad \omega_{2} = \int_{\beta}\frac{dx}{y} = \int_{\phi_{1}^{-1}\circ\beta}dz
		\end{equation*}
		are a basis for $\Lambda_{1}$, so in particular they are linearly independent.
		
		\item[(b)] We have just shown that the lattice $\Lambda_{1}$ corresponding to $E$ is precisely the lattice generated by the periods of $E$. The composition $F\circ\phi$ then gives an analytic map
		\begin{equation*}
			F\circ\phi:\CC/\Lambda \longrightarrow\CC/\Lambda,\qquad (F\circ\phi)(z) = \int_{O}^{(\wp(z), \wp^{\prime}(z))}\frac{dx}{y}.
		\end{equation*}
		Since
		\begin{equation*}
			F^{\ast}(dz) = \frac{dx}{y}\quad\text{and}\quad \phi^{\ast}\bigg(\frac{dx}{y}\bigg) = \frac{d\wp(z)}{\wp^{\prime}(z)} = dz,
		\end{equation*}
		we see that
		\begin{equation*}
			(F\circ \phi)^{\ast}dz = dz.
		\end{equation*}
		On the other hand, any analytic map $\CC/\Lambda \rightarrow \CC/\Lambda$ is of the form $\psi_{a}(z) = az$ for some number $a \in \CC^{\ast}$. Since $\psi_{a}^{\ast}(z) = adz$, we see that $(F\circ \phi)(z) = z$, that is, the composition $F \circ \phi$ is just the identity map. But we know that $\phi$ is an analytic isomorphism, and consequently $F = \phi^{-1}$ is too.
	\end{enumerate}
\end{proof}

With the correspondence between elliptic curves and complex tori firmly established, we can easily deduce the following:\\

\begin{prop}\cite{Silverman_2009}
	\label{torsion_prop}
	Let $E/\CC$ be an elliptic curve and let $n \geq 1$ be an integer.
	\begin{enumerate}
		\item[(a)] There is an isomorphism of abstract groups
		\begin{equation*}
			E[n] \cong \ZZ/n\ZZ \times \ZZ/n\ZZ.
		\end{equation*}
		\item[(b)] The multiplication-by-$n$ map $[n]:E \rightarrow E$ has degree $n^{2}$.
	\end{enumerate}
\end{prop}

\begin{proof}
	\begin{enumerate}
		\item[(a)] Since $E(\CC)$ is isomorphic to $\CC/\Lambda$ for some lattice $\Lambda \subset \CC$, we have
		\begin{equation*}
		E[n] \cong \bigg(\frac{\CC}{\Lambda}\bigg)[n] \cong \frac{\frac{1}{n}\Lambda}{\Lambda} \cong \bigg(\frac{\ZZ}{n\ZZ}\bigg)^{2}.
		\end{equation*}
		\item[(b)] As $\Char(\CC) = 0$ and the map $[n]$ is unramified, the degree of $[n]$ is equal to the number of points in $E[n] = [n]^{-1}\{O\}$.
	\end{enumerate}
\end{proof}
