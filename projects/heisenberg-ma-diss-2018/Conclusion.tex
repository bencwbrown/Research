\section{Conclusion}

We have used the basis vector space of theta functions of weight $n$ to embed elliptic curves as different models in projective space $\PP^{n-1}$, and found that when $n = 3, 4, 5$, the model is the Hesse pencil, the Fermat quartic, and the Bianchi quintic respectively. Moreover, we have provided the general formula for the models in higher dimensional projective space.

Furthermore the multiple symmetries of the embedded curves can be described through the action of the Heisenberg group, which acts as a representation group for the subgroup of $n$-torsion points of the curve. Even more interesting is the action of the normaliser of the Heisenberg group on the curve, which acts as a group of automorphisms on the family when $n = 3, 5$, and whose action of the singular members can be identified with the action of the tetrahedral and icosahedral groups on the 2-sphere, respectively.

The first natural direction for further study would be to investigate the nature of the quadric intersections of (\ref{quadric}) for larger values of $n$. Indeed when $n = 7$, there are 14 quadrics from (\ref{quadric}), and  Gross shows in \cite{Gross_1996} that the coefficient matrix of the Pfaffians of (\ref{gross_matrix}) gives rise to Klein's quartic
\begin{equation*}
	K = \{ (y_{1}:y_{2}:y_{3}) \in \PP^{2} \st y_{1}^{3}y_{2} + y_{2}^{3}y_{3} + y_{3}^{3}y_{1} = 0\},
\end{equation*}
which is also the unique $\PSL(2,\ZZ/7\ZZ)$ invariant of degree $\leq 4$, and is therefore the isomorphic image in $\PP^{6}$ of the modular curve $X(7)$. A similar example arises when $n = 11$, when an analogous cases arise for the modular curve $X(11)$, \cite{Gross_1996}.

It would also be interesting to investigate more thoroughly the action of the Heisenberg group on the elliptic fibrations that the curves give rise to in a similar vein to Proposition (\ref{shioda}), and their relation to Shioda's modular surfaces $S(n)$ as has been done in \cite{Barth_1985}.
