\section{Theta Functions}

We now want to find a non-constant map $\phi:E = \CC/\Lambda \rightarrow \PP^{n}$ to embed our elliptic curve $E$ in projective space for some $n$. Recall that the complex projective space $\PP^{n}$ is defined as the set of non-zero vectors in $\CC^{n+1}$ up to multiplication of some non-zero scalar. As a complex manifold, $\PP^{n}$ is covered by $n+1$ subsets $U_{i} = \{ (z_{0},\ldots, z_{n})\in \CC^{n+1} \st z_{i} \neq 0\}$, and a holomorphic map $\phi:E \rightarrow \PP^{n}$ is defined, after composition with the natural map $\CC \rightarrow \CC/\Lambda$, by $n+1$ holomorphic functions $f_{0},\ldots f_{n}$ on $\CC$. These $n+1$ functions need not be periodic with respect to $\Lambda$, but must satisfy the weaker property, namely that they must be \emph{theta functions}: \\

\begin{defn}[\cite{Dolgachev_1997}]
	A holomorphic function $f(z)$ on $\CC$ is called a \emph{theta function (relative to the lattice $\Lambda$)} if, for any $\lambda \in \Lambda$ there exists an invertible holomorphic function $e_{\lambda}(z)$ such that
	\begin{equation*}
		f(z + \lambda) = e_{\lambda}(z)f(z)\qquad\text{for all } \lambda \in \Lambda.
	\end{equation*}
	The set $\{e_{\lambda}(z)\}_{\lambda \in \Lambda}$ is called the \emph{theta factor} for $f$.
\end{defn}

As our first example of a theta function, we consider the \emph{Riemann theta function} $\vt(z,\tau)$ on the lattice $\Lambda_{\tau}$, defined by
\begin{equation*}
	\vt(z,\tau) = \sum_{n\in \ZZ}\exp\big(\pi i(2zn + n^{2}\tau)\big).
\end{equation*}
It can easily be shown to converge uniformly on $\CC\times \mcH$, see e.g. \cite{Mumford_1983}, and it satisfies
\begin{equation*}
	\begin{split}
	\vt(z + m + n\tau,\tau) = \exp(-\pi i (2nz + n^{2}\tau))\vt(z,\tau),
	\end{split}
\end{equation*}
so $\vt(z,\tau)$ is a theta function with the theta factor
\begin{equation*}
	e_{m+n\tau}(z) = \exp(-\pi i (2nz + n^{2}\tau)).
\end{equation*}

Now the question is what sort of form does a theta function $f(z)$ take on?  First of all, let us consider the lattice $\Lambda_{\tau} = \{ n + m\tau \st n,m \in \ZZ,\ \im\tau > 0 \}$; any lattice $\Lambda$ is equivalent to such a $\Lambda_{\tau}$ by homothety. Secondly, Liouville's theorem tells us that $f(z)$ cannot be doubly-periodic with respect to $\Lambda_{\tau}$, so now we focus our attention to finding entire functions $f(z)$ with the simplest quasi-periodic behaviour with respect to $\Lambda_{\tau}$, namely that $f(z)$ behaves as
\begin{equation*}
	f(z+1) = f(z),\qquad f(z +\tau) = \be(-(az + b))\cdot f(z),
\end{equation*}
where we write $\be(z) = \exp(2\pi i z)$, and which we call the functional equations for $f(z)$. Since $f(z)$ is periodic in $z$ with respect to $z \mapsto z+1$, we can expand it as a Fourier series
\begin{equation*}
	f(z) = \sum_{n\in\ZZ}a_{n}\be(n z),\qquad a_{n}\in\CC.
\end{equation*}
Then writing $f(z + 1 + \tau)$ in terms of $f(z)$ by combining the functional equations in either order, we find that
\begin{equation*}
	f(z + 1 + \tau) = f(z+\tau) = \be(-(az+b)),
\end{equation*}
and also
\begin{equation*}
	f(z + 1 + \tau) = \be(-(a(z+1) + b))f(z+1) = \be(-a)\be(-(az+b))f(z),
\end{equation*}
so $a = k$ for some $k \in \ZZ$. Substituting the Fourier series into the second functional equation, we find that
\begin{equation*}
	\begin{split}
	\sum_{n \in\ZZ}a_{n} \be( n \tau)\cdot\be(n z) &= f(z+\tau)\\
	&=\be(-(kz + b))\cdot f(z)\\
	&= \sum_{n\in\ZZ}a_{n}\be((n-k)z)\cdot\be(- b)\\
	&= \sum_{n\in\ZZ} a_{n+k}\be(-b)\cdot \be(n z).
	\end{split}
\end{equation*}
Comparing the coefficients of the first and last term, we get the recursive relation
\begin{equation}
\label{recursive}
	a_{n+k} = a_{n}\be((n\tau + b)).
\end{equation}
Now if $k = 0$, then for at most one $n$ we have that $a_{n} \neq 0$, and we have the uninteresting possibility that $f(z) = \be(z)$. If $k \neq 0$, then there is the recursive relation for solving for $a_{n + pk}$ in terms of $a_{n}$ for all $p \in \ZZ$, but when $k \leq -1$ we see that the recursive relation leads to rapidly growing coefficients $a_{n}$, and so there cannot be any entire functions $f(z)$. However, when $k \geq 1$ this is not the case, and we find a $k$-dimensional vector space of possibilities for $f(z)$, as each $f(z)$ is determine by its Fourier coefficients $a_{0},\ldots, a_{k-1}$. In fact, we can solve the recursive equation (\ref{recursive}) explicitly; to simplify things, let us replace $f(z)$ by $f(z + \tau/2 - b/k)$, then
\begin{equation*}
\begin{split}
f(z + \tau/2 - b/k + \tau) &= \be(-(k(z + \tau/2 - b/k) + b))\cdot f(z + \tau/2 - b/k)\\
&\be(-k(z + \tau/2))\cdot f(z + \tau/2 - b/k),
\end{split}
\end{equation*}
so we may assume that $b = k\tau/2$. Then in letting $i \in \{0,\ldots, k-1\}$, we get \cite{Dolgachev_1997}
\begin{equation*}
	a_{i + pn} = \be \big( \tfrac{1}{2}\big( (i + pn)^{2}\tau  \big)/n\big)\cdot a_{i}
\end{equation*}
as our explicit solution to the recurrence relation (\ref{recursive}). This shows that each $f(z)$ with the theta factor
\begin{equation}
\label{theta_factor}
	e_{k\tau + l}(z) = \be\big(-n(kz + \tfrac{k^{2}}{2}\tau )\big)
\end{equation}
can be written in the form
\begin{equation*}
	f(z) = \sum_{i=0}^{n-1} c_{i}\cdot \Theta_{i}(z,\tau)_{n},
\end{equation*}
where
\begin{equation*}
	\Theta_{i}(z,\tau)_{n}= \sum_{r\in\ZZ} \be\big(\tfrac{1}{2}(i + rn)^{2}\tau/n\big)\cdot \be(z(i + rn)),\qquad i = 0,\ldots, n-1.
\end{equation*}
We will see shortly that it will be more convenient to rewrite these functions in the form
\begin{equation}
	\label{big_theta}
	\Theta_{i}(z,\tau)_{n}= \sum_{r\in\ZZ} \be\bigg(\tfrac{1}{2}\big(\tfrac{i}{n} + r\big)^{2}n\tau\bigg)\cdot \be \bigg(nz\big(\tfrac{i}{n} + r\big)\bigg).
\end{equation}

It is easy to see using the uniqueness of the Fourier coefficients for a holomorphic function that the $\Theta_{i}(z)_{n}$ are linearly independent, and hence form a basis for the theta functions with the theta factor (\ref{theta_factor}). In summary,\\

\begin{prop}[\cite{Dolgachev_1997}]
	\label{theta_basis_one}
	Each theta factor is equivalent to the theta factor of the form
	\begin{equation*}
		e_{k+l\tau}(z) = \be\Big(-n\big(lz + \tfrac{l^{2}}{2}\tau\big)\Big).
	\end{equation*}
	The $\CC$-vector space $R_{n}(\Lambda_{\tau})$ of such functions is zero dimensional if $n<0$. For $n = 0$ it consists of constant functions, whereas for $n >0$ is if of dimension $n$ as is spanned by the functions
	\begin{equation*}
	\Theta_{i}(z,\tau)_{n}= \sum_{r\in\ZZ} \be\bigg(\tfrac{1}{2}\big(\tfrac{i}{n} + r\big)^{2}n\tau\bigg)\cdot \be \bigg(nz\big(\tfrac{i}{n} + r\big)\bigg).
	\end{equation*}
\end{prop}

A very useful definition is the following generalisation of the $\vt(z,\tau)$:\\

\begin{defn}[\cite{Mumford_1983}]
	For every $(a,b) \in \QQ^{2}$ and $(z,\tau) \in \CC \times \mcH$, the \emph{theta function of rational characteristic $(a,b)$}, is the series
	\begin{equation*}
		\vtc{a}{b}(z,\tau) = \sum_{r\in\ZZ}\be\bigg((r + a)(z + b) + \frac{1}{2}(r+a)(r+b)\tau \bigg),
	\end{equation*}
	which will be commonly abbreviated at $\vt_{a,b}(z,\tau)$.
\end{defn}

These are really just translates of $\vt(z,\tau)$ multiplied by an elementary exponential factor:
\begin{equation*}
	\vtc{a}{b}(z,\tau) = \be\big(a(z+b) + \tfrac{1}{2}a^{2}\tau\big)\cdot\vt(z + a\tau + b),\qquad\text{for all } a,b \in \QQ,
\end{equation*}

and in terms of a theta function with rational characteristics, we now see that
\begin{equation}
\label{theta_weight_n}
	\Theta_{i}(z,\tau)_{n} = \vtc{\frac{i}{n}}{0}(nz,n\tau),
\end{equation}
and we call such functions \emph{theta functions of weight $n$}, \cite{Mumford_1983}. With this in hand, we can rephrase Proposition \ref{theta_basis_one} in the following terms:\\

\begin{prop}[\cite{Mumford_1983}]
	\label{theta_basis}
	Fix a lattice $\Lambda_{\tau}\subset \CC$. Then a basis of the vector space of theta functions of weight $n$, $R_{n}(\Lambda_{\tau})$, can be given by:
	\begin{equation*}
		x_{i}(z) = \vtc{\tfrac{i}{n}}{0}(nz,n\tau),\qquad \text{for }i \in \ZZ/n\ZZ.
	\end{equation*}
\end{prop}

Now we can determine the zeros of the $\vtc{\tfrac{i}{n}}{0}(nz,n\tau)$:\\

\begin{prop}[\cite{Dolgachev_1997}]
	A non-zero function $f(z) \in R_{n}(\Lambda_{\tau})$ has exactly $n$ zeros in $\CC/\Lambda_{\tau}$ counting multiplicities.
\end{prop}

\begin{proof}
	It is well known that the number of zeros (with multiplicity) of a holomorphic function $f(z)$ on an open subset $U$ of $\CC$ inside of a compact set $K \subset U$ is equal to
	\begin{equation}
		\text{\# of zeros of } f(z) = \frac{1}{2\pi i}\int_{\partial K}d \log f(z)\ dz.
	\end{equation}
	We assume that $f(z)$ has no zeros on $\partial K$, and after a suitable translation by $z_{0}\in \CC$, we can take the fundamental parallelogram $z_{0} + \Pi$ for $\Lambda_{\tau}$ as our $K$. As $f(z) \in R_{n}(\Lambda_{\tau})$,
	\begin{equation*}
		d\log f(z+\tau) = -2\pi indz + d\log f(z)
	\end{equation*}
	as $\tau$ is fixed, and then we obtain from the above equation that
	\begin{equation*}
		\begin{split}
		2\pi i\cdot(\text{\# of zeros of } f(z)) &= \int_{\partial K}d \log f(z)\ dz\\
		&= \int_{z_{0}}^{z_{0} + 1} (d \log f(z) - d\log f(z+\tau))\ dz\\
		&- \int_{z_{0}}^{z_{0} + \tau} (d \log f(z) - d\log f(z+1))\ dz\\
		&= \int_{z_{0}}^{z_{0} + 1} 2\pi in\ dz = 2\pi i n,
		\end{split}
	\end{equation*}
	which proves our assertion.
\end{proof}

\begin{lemma}[\cite{Dolgachev_1997}]
	\label{zero_point}
	The zero of the function $\vt_{a,b}(z,\tau)$ in $\CC/\Lambda_{\tau}$ is the point
	\begin{equation}
		P = \bigg(a + \frac{1}{2}\bigg)\tau + \bigg(b + \frac{1}{2}\bigg).
	\end{equation}
\end{lemma}

\begin{proof}
	We observe that
	\begin{equation*}
		\begin{split}
		\vt_{\frac{1}{2},\frac{1}{2}}(-z,\tau) &= \sum_{m\in \ZZ}\be\big(\tfrac{1}{2}(m+1/2)^{2}\tau + (m+1/2)(-z+1/2)\big)\\
		&= \sum_{k\in \ZZ}\be\big(\tfrac{1}{2}(-k-1/2)^{2}\tau + (k+1/2)(z-1/2)\big) \qquad(\text{where }k = -m-1)\\
		&= \sum_{k\in \ZZ}\be\big(\tfrac{1}{2}(k+1/2)^{2}\tau + (k+1/2)(z+1/2)\big)\be(-(k+1/2))\\
		&= -\vt_{\frac{1}{2},\frac{1}{2}}(z,\tau),
		\end{split}
	\end{equation*}
	so the theta function $\vt_{\frac{1}{2},\frac{1}{2}}(z,\tau)$ is an odd function, thus its zero is located at $z=0$ in $\Lambda_{\tau}$. The zero point $P$ for $\vt_{a,b}(z,\tau)$ is then the one stated, as $\vt_{a,b}(z,\tau)$ is obtained from $\vt_{\frac{1}{2},\frac{1}{2}}(z,\tau)$ by translation.
\end{proof}

\begin{cor}[\cite{Dolgachev_1997}]
	\label{cor_zeros}
	The zeros of the function $\vt_{a,b}(nz,n\tau)$ in $\CC/\Lambda_{\tau}$ are the points
	\begin{equation}
		\label{zeros}
		P_{i} = \bigg(a + \frac{1}{2}\bigg)\tau + \frac{b}{n} + \frac{1}{2n} + \frac{i}{n},\qquad i= 0, \ldots, n-1.
	\end{equation}
\end{cor}

\begin{proof}
	By Lemma \ref{zero_point}, if $P$ is the zero for $\vt_{a,b}(z,\tau)$ then a zero point for $\vt_{a,b}(nz,n\tau)$ is of the form
	\begin{equation*}
		nP = \bigg(a + \frac{1}{2}\bigg)n\tau + \bigg(b + \frac{1}{2}\bigg) + \ZZ + n\tau\ZZ,
	\end{equation*}
	and consequently
	\begin{equation}
		P_{i} = \bigg(a+\frac{1}{2}\bigg)\tau + \bigg(\frac{b}{n} + \frac{1}{2n} + \frac{i}{n}\bigg) + \Lambda_{\tau},\qquad i = 0,\ldots,n-1.
	\end{equation}
\end{proof}

We can now state and prove the main theorem of this section, that lets us embed a complex elliptic curve $E_{\tau} = \CC/\Lambda_{\tau}$ into projective space by means of theta functions.\\

\begin{theorem}[\cite{Dolgachev_1997,Mumford_1983}]
	For each $n\geq 1$, the map
	\begin{equation*}
		\begin{split}
		\phi_{n}:E_{\tau}&\longrightarrow \PP^{n-1}\\
		z &\longmapsto \bigg(\vtc{0}{0}(nz,n\tau),\vtc{\frac{1}{n}}{0}(nz,n\tau),\ldots, \vtc{\frac{n-1}{n}}{0}(nz,n\tau)\bigg)
		\end{split}
	\end{equation*}
	defines a holomorphic map. If $n\geq 3$, this map is a holomorphic embedding.
\end{theorem}

\begin{proof}
	Firstly, the map is well-defined, since each theta function $\vt_{i/n,0}(nz,n\tau)$ has the same theta factor. Also from Corollary \ref{cor_zeros}, they do not all vanish at the same point, hence define the same point in projective space. The map is holomorphic since the theta functions are holomorphic functions.
	
	Let us show that it is injective when $n\geq 3$. Suppose that $\phi_{n}(z_{1}) = \phi_{n}(z_{1}^{\prime})$, or that $d\phi_{n}(z_{1}) = 0$. Then for any integers $k,l$,
	\begin{equation*}
		\vtc{\frac{i}{n}}{0}(nz + k + l\tau,n\tau) = \be(ki/s)\be(2nlz + \tfrac{ln\tau}{2})\vtc{\frac{i}{n}}{0}(nz,n\tau). 
	\end{equation*}
	This shows that $\phi_{n}(z_{1} + \tfrac{k}{n} + \tfrac{l}{n\tau}  ) = \phi_{n}(z_{1}^{\prime} + \tfrac{k}{n} + \tfrac{l}{n}\tau)$. Note that, if $n \geq 3$, we can always choose $k$ and $l$ to be such that the four points $z_{1}, z_{1}^{\prime}$, $z_{2} = z_{1} + \tfrac{k}{n} + \tfrac{l}{n}\tau, z_{2}^{\prime} = z_{1} + \tfrac{k}{n} + \tfrac{l}{n}\tau$ are distinct. The linear space generated by the functions $\vt_{i/n, 0}$ is of dimension $n$. So we can find a linear combination $f$ of these functions such that it vanishes at $z_{1}, z_{2}$, and some other $n-3$ points $z_{3},\ldots, z_{n-1}$, which are distinct modulo $\Lambda_{\tau}$. But then $f$ also vanishes at $z_{1}^{\prime}$ and $z_{2}^{\prime}$, or $f$ has a double zero at $z_{1}$ and $z_{2}$. Thus we have $n+1$ zeros of $f$ counting multiplicities, which contradicts Lemma \ref{zero_point} and proves our assertion.
\end{proof}

We finish this section by stating the image of our complex elliptic curve $E_{\tau} = \CC/\Lambda_{\tau}$ is in fact and algebraic variety, when embedded with some projective space $\PP^{n}$. This is due to Chow's Theorem, which states:\\

\begin{theorem}[Theorem of Chow, \cite{Mumford_1974}]
	Let $X$ be a complete algebraic variety and $Y$ a closed analytic subset of $X_{\text{hol}}$, where $X_{\text{hol}}$ is the canonically associated analytic space structure on the underlying set of $X$. Then $Y$ is Zariski closed in $X$.
\end{theorem}

Chow proved the theorem for $X = \PP^{n}$, and as far as we are concerned this is the case we are after, with $Y = \phi_{n}(E_{\tau})$.