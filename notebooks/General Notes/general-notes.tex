\documentclass{article}


\usepackage{arxiv}

\usepackage[utf8]{inputenc} % allow utf-8 input
\usepackage[T1]{fontenc}    % use 8-bit T1 fonts
\usepackage{hyperref}       % hyperlinks
\usepackage{url}            % simple URL typesetting
\usepackage{booktabs}       % professional-quality tables
\usepackage{amsfonts}       % blackboard math symbols
\usepackage{nicefrac}       % compact symbols for 1/2, etc.
\usepackage{microtype}      % microtypography
\usepackage{lipsum}		% Can be removed after putting your text content
\usepackage{amsmath} 
\usepackage{amssymb}
\usepackage{graphicx}
\usepackage{epstopdf}
\usepackage{url}
\usepackage{setspace}
\usepackage{amsthm}
\usepackage{mathrsfs}
\usepackage{enumitem}
\usepackage{parskip}
\usepackage{IEEEtrantools}
\usepackage{mathtools}
\usepackage{tensor}
\usepackage{yfonts}
\usepackage{dsfont}

\usepackage{pgfplots}
\pgfplotsset{compat=1.15}

\usetikzlibrary{arrows}

\definecolor{wwwwww}{rgb}{0.4,0.4,0.4}
\definecolor{wwzzff}{rgb}{0.4,0.6,1}
\definecolor{cqcqcq}{rgb}{0.7529411764705882,0.7529411764705882,0.7529411764705882}

\newtheorem{theorem}{Theorem}[section]
\newtheorem{lemma}[theorem]{Lemma}
\newtheorem*{lemma*}{Lemma}
\newtheorem{prop}[theorem]{Proposition}
\newtheorem{corollary}[theorem]{Corollary}
\newtheorem{defn}[theorem]{Definition\rm}
\newtheorem{conjecture}[theorem]{Conjecture}
\newtheorem{remark}{\it Remark\/}
\newtheorem{example}{Example}
\newtheorem{fact}{Fact}

\newcommand{\st}{\ensuremath{:}}% such that
\newcommand{\ie}{\emph{i.e.} }
\newcommand{\eg}{\emph{e.g.} }
\newcommand{\cf}{\emph{cf.} }
\newcommand{\ra}{\rightarrow}
\newcommand{\la}{\leftarrow}
\newcommand{\lra}{\longrightarrow}
\newcommand{\lla}{\longleftarrow}
\newcommand{\lbracket}{\left(}
\newcommand{\rbracket}{\right)}


\newcommand{\al}{\alpha}
\newcommand{\w}{\omega}
\newcommand{\W}{\Omega}
\newcommand{\m}{\mu}
\newcommand{\n}{\nu}
\newcommand{\e}{\epsilon}
\newcommand{\K}{K\"ahler }
\newcommand{\HK}{hyperk\"ahler }
\newcommand{\into}{\hookrightarrow}
\newcommand{\PP}{\mathbb{P}}
\newcommand{\RR}{\mathbb{R}}
\newcommand{\CC}{\mathbb{C}}
\newcommand{\QQ}{\mathbb{Q}}
\newcommand{\FF}{\mathbb{F}}
\newcommand{\ZZ}{\mathbb{Z}}
\newcommand{\NN}{\mathbb{N}}
\newcommand{\HH}{\mathbb{H}}
\newcommand{\vp}{\varphi}
\newcommand{\mcA}{\mathcal{A}}
\newcommand{\mcE}{\mathcal{E}}
\newcommand{\mcF}{\mathcal{F}}
\newcommand{\mcG}{\mathcal{G}}
\newcommand{\mcH}{\mathcal{H}}
\newcommand{\mcL}{\mathcal{L}}
\newcommand{\mcO}{\mathcal{O}}
\newcommand{\mfg}{\mathfrak{g}}
\newcommand{\mfh}{\mathfrak{h}}
\newcommand{\mft}{\mathfrak{t}}
\newcommand{\mc}[1]{\mathcal{#1}}
\newcommand{\mf}[1]{\mathfrak{#1}}

\newcommand{\pbrackets}[1]{\left( #1 \right)}
\newcommand{\bbrackets}[1]{\left[ #1 \right]}

\newcommand{\dbar}{\bar{\partial}}
\newcommand{\mrr}{\mu_{\mathbb{R}}}
\newcommand{\mcc}{\mu_{\mathbb{C}}}
\newcommand{\prr}{\phi_{\mathbb{R}}}
\newcommand{\pcc}{\phi_{\mathbb{C}}}

\DeclareMathOperator{\Lie}{Lie}
\DeclareMathOperator{\Aut}{Aut}
\DeclareMathOperator{\Tr}{Tr}
\DeclareMathOperator{\Image}{Im}
\DeclareMathOperator{\Ad}{Ad}
\DeclareMathOperator{\Diff}{Diff}
\DeclareMathOperator{\Vect}{Vect}
\DeclareMathOperator{\Sympl}{Sympl}
\DeclareMathOperator{\Span}{Span}
\DeclareMathOperator{\ind}{ind}
\DeclareMathOperator{\Td}{Td}
\DeclareMathOperator{\Ch}{Ch}
\DeclareMathOperator{\Ind}{Ind}
\DeclareMathOperator{\pt}{pt}
\DeclareMathOperator{\rk}{rk}
\DeclareMathOperator{\coker}{coker}
\DeclareMathOperator{\Pf}{Pf}
\DeclareMathOperator{\Vol}{Vol}
\DeclareMathOperator{\Res}{Res}

\DeclareMathOperator{\GL}{GL}
\DeclareMathOperator{\SO}{SO}
\DeclareMathOperator{\UU}{U}

\newcommand\restr[2]{{% we make the whole thing an ordinary symbol
		\left.\kern-\nulldelimiterspace % automatically resize the bar with \right
		#1 % the function
		\vphantom{\big|} % pretend it's a little taller at normal size
		\right|_{#2} % this is the delimiter
}}

\title{General Notes}

\date{}	% Here you can change the date presented in the paper title
%\date{} 					% Or removing it

%\author{
%  David S.~Hippocampus\thanks{Use footnote for providing further
%    information about author (webpage, alternative
%    address)---\emph{not} for acknowledging funding agencies.} \\
%  Department of Computer Science\\
%  Cranberry-Lemon University\\
%  Pittsburgh, PA 15213 \\
%  \texttt{hippo@cs.cranberry-lemon.edu} \\
%% examples of more authors
%   \And
% Elias D.~Striatum \\
%  Department of Electrical Engineering\\
%  Mount-Sheikh University\\
%  Santa Narimana, Levand \\
%  \texttt{stariate@ee.mount-sheikh.edu} \\
%% \AND
%% Coauthor \\
%% Affiliation \\
%% Address \\
%% \texttt{email} \\
%% \And
%% Coauthor \\
%% Affiliation \\
%% Address \\
%% \texttt{email} \\
%% \And
%% Coauthor \\
%% Affiliation \\
%% Address \\
%% \texttt{email} \\
%}

\begin{document}
	\maketitle
	
	\begin{abstract}
		Rough general notes.
	\end{abstract}
	
	\section{Residue Theorems}
	
	\begin{lemma}[\cite{Guillemin1999}]\label{residue:1}
		Let $A$ be a graded commutative algebra over $\CC$ and let $f = f(x)$ be a polynomial in $x$ with coefficients in $A$. Then for indeterminants $z_{1}, \ldots, z_{d}$,
		\begin{equation*}
			\Res_{x} \frac{f(x)}{(x - z_{1})\ldots(x - z_{d})} = \sum\limits_{i = 1}^{d}\frac{f(z_{i})}{\prod_{j \neq i}(z_{i} - z_{j})}.
		\end{equation*}
	\end{lemma}
	
	\begin{proof}
		Decompose into simple fractions:
		\begin{equation*}
			\frac{f(x)}{(x - z_{1})\ldots(x - z_{d})} = F(x) + \sum\limits_{i = 1}^{d} \frac{f(z_{i})}{\prod_{j \neq i} (z_{i} - z_{j})} \frac{1}{(x - z_{i})}.
		\end{equation*}
		Here $F(x)$ is a polynomial term in $x$.
	\end{proof}
	
	Let
	
	\begin{equation*}
		h = \frac{f}{ \prod_{j = 1}^{d}(x - z_{j}) } \quad \text{and} \quad h_{j} = \frac{f(z_{j})}{\prod_{r \neq j}(z_{j} - z_{r})}, \quad \text{for all } j.
	\end{equation*}
	
	\begin{lemma}[\cite{Guillemin1999}]\label{residue:2}
		$h \in A[x]$ if and only if $\Res_{x}(x^{k}h) = 0$, for all $k \geq 0$.
	\end{lemma}
	
	\begin{proof}
		From Lemma \ref{residue:1}, we get that
		\begin{equation*}
			\Res_{x}(x^{k}h) = \sum_{j = 1}^{d} (z_{j})^{k} h_{j}.
		\end{equation*}
		
		Then the condition that $\Res_{x}(x^{k}h) = 0$ for every $k = 1, \ldots, d$ can be written as
		
		\begin{equation*}
			\begin{pmatrix}
				z_{1}^{1} & \ldots & z_{j}^{1} & \ldots & z_{d}^{1} \\
				\vdots & & \vdots & & \vdots \\
				z_{1}^{k} & \ldots & z_{j}^{k} & \ldots & z_{d}^{k} \\
				\vdots & & \vdots & & \vdots \\
				z_{1}^{d} & \ldots & z_{j}^{d} & \ldots & z_{d}^{d}
			\end{pmatrix}
			\begin{pmatrix}
				h_{1} \\ \vdots \\ h_{j} \\ \vdots \\ h_{d}
			\end{pmatrix}
			= 0.
		\end{equation*}
	
		As the corresponding Van der Monde determinant is non-zero, we deduce that $h_{1} = \ldots = h_{d} = 0$, that is, $f(z_{j}) = 0$, for all $j = 1, \ldots, d$, from which we obtain that $h \in A[x]$.
	\end{proof}
	
	\begin{theorem}[\cite{CanasdaSilva1996}]\label{residue:3}
		Let $V$ be an $n$-dimensional vector space over $\CC$, and let $\tau_{k}$ be the standard representation of $\GL(V)$ on the $k$-th symmetric product, $S^{k}(V)$. Then for $z \in \CC$ large and $B \in \GL(V)$,
		\begin{equation*}
			\det(z - B)^{-1} = z^{-n}\sum\limits_{k=0}^{\infty} z^{-k} \Tr(\tau_{k}(B)).
		\end{equation*}
	\end{theorem}

	\begin{proof}
		Without loss of generality, assume that $B$ is diagonalisable with eigenvalues, $\lambda_{1}, \ldots, \lambda_{n}$. The left-hand side then becomes
		\begin{equation*}
			\det(z - B)^{-1} = z^{-n}\prod\limits_{j = 1}^{n}(1 - \lambda_{j}z^{-1})^{-1}.
		\end{equation*}
		Expanding each of the factors $(1 - \lambda_{j}z^{-1})^{-1}$ into a geometric series, we then get
		\begin{equation*}
			z^{-n}\prod\limits_{j = 1}^{n}(1 - \lambda_{j}z^{-1})^{-1} = z^{-1}\left( \sum_{k = 0}^{\infty} z^{k}t_{k} \right),
		\end{equation*}
		where
		\begin{equation*}
			t_{k} = \sum_{|I| = k} \lambda_{1}^{i_{1}}\ldots\lambda_{n}^{i_{n}} = \Tr\pbrackets{\tau_{k}(B)}.
		\end{equation*}
	\end{proof}
	
	\begin{corollary}[\cite{CanasdaSilva1996}]\label{residue:4}
		Let $\Gamma$ be a contour about the origin containing the zeroes of $\det(z - B)$. Then
		\begin{equation*}
			\frac{1}{2\pi i} \int_{\Gamma} z^{n + k -1} \det(z - B)^{-1} dz = \Tr\pbrackets{\tau_{k}(B)}.
		\end{equation*}
	\end{corollary}

	\section{Representation Theory}
	
	\begin{defn}[\cite{Sturmfels2008}]
		A \textbf{representation} of $\Gamma = \GL(\CC^{n})$ (or $\Gamma$-\textbf{module}) is a pair $(V, \rho)$, where $V$ is a $\CC$-vector space and
		\begin{equation*}
			\begin{split}
				\rho : \Gamma &\lra \GL(V), \\
				A = (a_{ij})_{1 \leq i,j \leq n} &\longmapsto \rho(A) = \pbrackets{\rho_{kl}(A)}_{1 \leq k, l \leq N}
			\end{split}
		\end{equation*}
		is a group homomorphism. The \textbf{dimension} $N$ of the representation $(V, \rho)$ is the dimension of the vector space $V$. We say that $(V, \rho)$ is a \textbf{polynomial representation} (of \textbf{degree} $d$) if the matrix entries $\rho_{kl}(A) = \rho_{kl}(a_{11}, a_{12}, \ldots, a_{nn})$ are polynomial functions (homogeneous of degree $d$).
	\end{defn}
	
	\begin{example}
		The $d$-th symmetric power representation: $V = S_{d}(\CC^{n}) =$ the space of homogeneous polynomials of degree $d$ in $x_{1}, x_{2}, \ldots, x_{n},\ \rho =$ action by linear substitution, $N = \binom{n + d -1}{d}$.
		
		For example, for $d = 3, n = 2$, we have $S_{3}(\CC^{2}) =$ binary cubics $=\Span\{x^{3}, x^{2}y, xy^{2}, y^{3}\} \cong \CC^{4}$, and $\rho$ is the group homomorphism
		\begin{equation*}
				\begin{pmatrix}
					a_{11} & a_{12} \\ 
					a_{21} & a_{22}
				\end{pmatrix}
				\mapsto
				\begin{pmatrix}
					a_{11}^{3} & a_{11}^{2}a_{12} & a_{11}a_{12}^{2} & a_{12}^{3} \\
					3a_{11}^{2} & a_{11}^{2}a_{22} + 2a_{11}a_{12}a_{21} & 2a_{11}a_{12}a_{22} + a_{12}a_{21}^{2} & 3a_{12}^{2}a_{22} \\ 
					3a_{11}^{2}a_{21}^{2} & 2a_{11}a_{21}a_{22} + a_{12}a_{21}^{2} & a_{11}a_{22}^{2} + 2a_{12}a_{21}a_{22} & 3a_{12}a_{22}^{2} \\ 
					a_{21}^{3} & a_{21}^{2}a_{22} & a_{21}a_{22}^{2} & a_{22}^{3}
				\end{pmatrix}.
		\end{equation*}
	\end{example}
	
	
	
	\section{Personal Calculations}
	
	\subsection{Index Formulae}
	
	Recall from Lemma (\ref{residue:1}) and (\ref{residue:2}):
	
	\begin{equation*}
		\Res_{x} \frac{f(x)}{(x - z_{1})\ldots(x - z_{d})} = \sum\limits_{i = 1}^{d}\frac{f(z_{i})}{\prod_{j \neq i}(z_{i} - z_{j})}.
	\end{equation*}
	
	and
	
	\begin{equation*}
		\frac{1}{2\pi i} \int_{\Gamma} z^{n + k -1} \det(z - B)^{-1} dz = \Tr\pbrackets{\tau_{k}(B)},
	\end{equation*}

	for $\Gamma$ a contour circling the origin and the zeros of $\det(z - B)$.
	
	\subsubsection{$\Ind(\CC\PP^{2}, \mcO(k), T^{2})$}
	
	\begin{equation*}
		\begin{split}
			\Ind(\CC\PP^{2}, \mcO(k), T^{2})(z_{1}, z_{2}) &= \frac{1}{(1 - z_{1})(1 - z_{2})} + \frac{z_{1}^{k}}{(1 - z_{1}^{-1})(1 - z_{1}^{-1}z_{2})} + \frac{z_{2}^{k}}{(1 - z_{2}^{-1})(1 - z_{2}^{-1}z_{1})} \\
			&= \frac{1}{(1 - z_{1})(1 - z_{2})} + \frac{z_{1}^{k+2}}{(z_{1} - 1)(z_{1} - z_{2})} + \frac{z_{2}^{k+2}}{(z_{2} - 1)(z_{2} - z_{1})}
		\end{split}
	\end{equation*}
	
	
		
	
	
	
	
	
	
	
	
	
	
	
	
	
	
	
	\bibliographystyle{unsrt}  
	\bibliography{general-notes}  %%% Remove comment to use the external .bib file (using bibtex).
	%%% and comment out the ``thebibliography'' section.
	
	%%% Comment out this section when you \bibliography{references} is enabled.
	
\end{document}
