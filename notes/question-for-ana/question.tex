\documentclass[11pt]{amsart}

\usepackage[parfill]{parskip}    
\usepackage{graphicx}
\usepackage{amssymb, amsfonts,mathabx}
\usepackage{epstopdf}

\usepackage{amsmath, amsthm}
\usepackage[all,cmtip]{xy}

\usepackage{braket}
\usepackage{tikz}
\usepackage{tikz-cd}
\usetikzlibrary{matrix,arrows,patterns,calc,through,backgrounds,fadings, decorations}
\usetikzlibrary{decorations.pathreplacing}

\newtheorem{theorem}{Theorem}[section]
\newtheorem{lemma}[theorem]{Lemma}
\newtheorem*{lemma*}{Lemma}
\newtheorem{proposition}[theorem]{Proposition}
\newtheorem{corollary}[theorem]{Corollary}
\newtheorem{definition}[theorem]{Definition\rm}
\newtheorem{conjecture}[theorem]{Conjecture}
\newtheorem{remark}{\it Remark\/}
\newtheorem{example}{Example}

\newcommand{\st}{\ensuremath{:}}% such that
\newcommand{\ie}{\emph{i.e.} }
\newcommand{\eg}{\emph{e.g.} }
\newcommand{\cf}{\emph{cf.} }
\newcommand{\ra}{\rightarrow}
\newcommand{\la}{\leftarrow}
\newcommand{\lra}{\longleftarrow}
\newcommand{\lla}{\longleftarrow}
\newcommand{\lbracket}{\left(}
\newcommand{\rbracket}{\right)}


\newcommand{\al}{\alpha}
\newcommand{\w}{\omega}
\newcommand{\m}{\mu}
\newcommand{\n}{\nu}
\newcommand{\e}{\epsilon}
\newcommand{\K}{K\"ahler }
\newcommand{\HK}{hyperk\"ahler }
\newcommand{\into}{\hookrightarrow}
\newcommand{\PP}{\mathbb{P}}
\newcommand{\RR}{\mathbb{R}}
\newcommand{\CC}{\mathbb{C}}
\newcommand{\QQ}{\mathbb{Q}}
\newcommand{\FF}{\mathbb{F}}
\newcommand{\ZZ}{\mathbb{Z}}
\newcommand{\NN}{\mathbb{N}}
\newcommand{\HH}{\mathbb{H}}
\newcommand{\vp}{\varphi}
\newcommand{\mc}[1]{\mathcal{#1}}
\newcommand{\mcE}{\mathcal{E}}
\newcommand{\mcF}{\mathcal{F}}
\newcommand{\mcG}{\mathcal{G}}
\newcommand{\mcH}{\mathcal{H}}
\newcommand{\mcL}{\mathcal{L}}
\newcommand{\mcO}{\mathcal{O}}
\newcommand{\mf}[1]{\mathfrak{#1}}
\newcommand{\mfg}{\mathfrak{g}}
\newcommand{\mfh}{\mathfrak{h}}
\newcommand{\mft}{\mathfrak{t}}

\newcommand{\dbar}{\bar{\partial}}

\DeclareMathOperator{\Lie}{\text{Lie}}
\DeclareMathOperator{\Aut}{Aut}
\DeclareMathOperator{\Tr}{Tr}
\DeclareMathOperator{\Image}{Im}
\DeclareMathOperator{\Ad}{Ad}
\DeclareMathOperator{\Diff}{Diff}
\DeclareMathOperator{\Vect}{Vect}
\DeclareMathOperator{\Sympl}{Sympl}
\DeclareMathOperator{\Span}{Span}
\DeclareMathOperator{\ind}{ind}
\DeclareMathOperator{\Td}{Td}
\DeclareMathOperator{\Ch}{Ch}

\usepackage{hyperref}

\title{Question R.E. $S^{1}$-Action via a Subtorus}
\author{Benjamin C. W. Brown}
\address[Benjamin Brown]{School of Mathematics and Maxwell Institute, The University of Edinburgh, Peter Guthrie Tait Road, Edinburgh EH9 3FD, United Kingdom}
\email{B.Brown@ed.ac.uk}
\date{\today}  
\thanks{}                                         
\begin{document}

\maketitle
 
\section{Example}

Take $(\CC^{2}, \w_{std})$ with $T^{2}$ acting on $\CC^{2}$ as
\[
	(t_{1}, t_{2}) \cdot (z_{1}, z_{2}) = (t_{1}z_{1}, t_{2}z_{2}).
\]

This action is Hamiltonian with moment map $\mu : \CC^{2} \ra \RR^{2}$ given by
\[
	\mu(z_{1},z_{2}) = \frac{1}{2}\left(|z_{1}|^{2}, |z_{2}|^{2}\right).
\]

For the symplectic cut, relabel $M:= \CC^{2}$ then consider $M \times \CC$, along with the following $S^{1}$-action:
\[
	\tau \cdot (z_{1}, z_{2}, \xi) = (\tau z_{1}, \tau z_{2}, \tau \xi),
\]

so $S^{1}$ can be thought of acting on the first factor, $M$, via the inclusion $S^{1} \hookrightarrow T^{2}$ as $\tau \mapsto (\tau, \tau)$, and then via the diagonal product action on $M \times \CC$. This action is also Hamiltonian, with moment map $\Phi : M \times \CC \ra \RR$, given by
\[
	\Phi(z_{1}, z_{2}, \xi) = \frac{1}{2}\left(|z_{1}|^{2} + |z_{2}|^{2} + |\xi|^{2}\right).
\]

Consider the preimage of $k \in \ZZ$ under $\Phi$ to get
\[
	\Phi^{-1}(k) = \Set{(z_{1} , z_{2}, 0) \in M \times \CC | \|z\|^{2} = 2k} \bigsqcup \Set{ (z_{1}, z_{2}, \xi) \in M \times \CC | \|z\|^{2} < 2k },
\]
then its quotient by the $S^{1}$-action is the symplectic cut of $\CC^{2}$:
\[
	M_{k} \cong \
\]



\providecommand{\bysame}{\leavevmode\hbox to3em{\hrulefill}\thinspace}
\providecommand{\href}[2]{#2}

\bibliographystyle{unsrt}
\bibliography{question}

\end{document}  










