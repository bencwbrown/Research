\documentclass{article}


\usepackage{arxiv}
%\usepackage{ebgaramond}
\usepackage{garamondx}
% \usepackage{CormorantGaramond}

\usepackage[utf8]{inputenc} % allow utf-8 input
\usepackage[T1]{fontenc}    % use 8-bit T1 fonts
\usepackage{hyperref}       % hyperlinks
\usepackage{url}            % simple URL typesetting
\usepackage{booktabs}       % professional-quality tables
\usepackage{amsfonts}       % blackboard math symbols
\usepackage{nicefrac}       % compact symbols for 1/2, etc.
\usepackage{microtype}      % microtypography
% \usepackage{lipsum}		% Can be removed after putting your text content
\usepackage{amsmath} 
\usepackage{amssymb}
\usepackage{graphicx}
\usepackage{epstopdf}
\usepackage{url}
\usepackage{setspace}
\usepackage{amsthm}
\usepackage{mathrsfs}
\usepackage{enumitem}
\usepackage{parskip}
\usepackage{IEEEtrantools}
\usepackage{mathtools}
\usepackage{tensor}
\usepackage{yfonts}
\usepackage{dsfont}

%%%%%%%%%%%%%%%%%%% Custom packages
\usepackage{braket}
\usepackage{todo}
\usepackage{xargs}                      % Use more than one optional parameter in a new commands
\usepackage{tikz}
\usepackage{tikz-cd}

%%%%%%%%%%%%%%%%%%


\usepackage{pgfplots}
\pgfplotsset{compat=1.15}

\usetikzlibrary{arrows}



\newtheorem{theorem}{Theorem}[section]
\newtheorem{lemma}[theorem]{Lemma}
\newtheorem*{lemma*}{Lemma}
\newtheorem{prop}[theorem]{Proposition}
\newtheorem{corollary}[theorem]{Corollary}
\newtheorem{defn}[theorem]{Definition\rm}
\newtheorem{conjecture}[theorem]{Conjecture}
\newtheorem{remark}{\it Remark\/}
\newtheorem{example}{Example}
\newtheorem{fact}{Fact}

\newcommand{\st}{\ensuremath{:}}% such that
\newcommand{\ie}{\emph{i.e.} }
\newcommand{\eg}{\emph{e.g.} }
\newcommand{\cf}{\emph{cf.} }
\newcommand{\ra}{\rightarrow}
\newcommand{\la}{\leftarrow}
\newcommand{\lra}{\longrightarrow}
\newcommand{\lla}{\longleftarrow}
\newcommand{\lbracket}{\left(}
\newcommand{\rbracket}{\right)}

\newcommand{\al}{\alpha}
\newcommand{\w}{\omega}
\newcommand{\W}{\Omega}
\newcommand{\m}{\mu}
\newcommand{\n}{\nu}
\newcommand{\e}{\epsilon}
\newcommand{\K}{K\"ahler }
\newcommand{\HK}{hyperk\"ahler }
\newcommand{\into}{\hookrightarrow}
\newcommand{\PP}{\mathbb{P}}
\newcommand{\RR}{\mathbb{R}}
\newcommand{\CC}{\mathbb{C}}
\newcommand{\QQ}{\mathbb{Q}}
\newcommand{\FF}{\mathbb{F}}
\newcommand{\ZZ}{\mathbb{Z}}
\newcommand{\NN}{\mathbb{N}}
\newcommand{\HH}{\mathbb{H}}
\newcommand{\vp}{\varphi}
\newcommand{\mcA}{\mathcal{A}}
\newcommand{\mcE}{\mathcal{E}}
\newcommand{\mcF}{\mathcal{F}}
\newcommand{\mcG}{\mathcal{G}}
\newcommand{\mcH}{\mathcal{H}}
\newcommand{\mcL}{\mathcal{L}}
\newcommand{\mcO}{\mathcal{O}}
\newcommand{\mfg}{\mathfrak{g}}
\newcommand{\mfh}{\mathfrak{h}}
\newcommand{\mft}{\mathfrak{t}}
\newcommand{\mc}[1]{\mathcal{#1}}
\newcommand{\mf}[1]{\mathfrak{#1}}

\newcommand{\sslash}{\mathbin{/\mkern-6mu/}}
\newcommand{\sssslash}{\mathbin{/\mkern-6mu/\mkern-6mu/\mkern-6mu/}}

\newcommand{\pbrackets}[1]{\left( #1 \right)}
\newcommand{\bbrackets}[1]{\left[ #1 \right]}
\newcommand{\norm}[1]{|#1|^{2}}

\newcommand{\dbar}{\bar{\partial}}
\newcommand{\mrr}{\mu_{\mathbb{R}}}
\newcommand{\mcc}{\mu_{\mathbb{C}}}
\newcommand{\prr}{\phi_{\mathbb{R}}}
\newcommand{\pcc}{\phi_{\mathbb{C}}}

\DeclareMathOperator{\Lie}{Lie}
\DeclareMathOperator{\Aut}{Aut}
\DeclareMathOperator{\Tr}{Tr}
\DeclareMathOperator{\Image}{Im}
\DeclareMathOperator{\Ad}{Ad}
\DeclareMathOperator{\Diff}{Diff}
\DeclareMathOperator{\Vect}{Vect}
\DeclareMathOperator{\Sympl}{Sympl}
\DeclareMathOperator{\Span}{Span}
\DeclareMathOperator{\ind}{ind}
\DeclareMathOperator{\Td}{Td}
\DeclareMathOperator{\Ch}{Ch}
\DeclareMathOperator{\Ind}{Ind}
\DeclareMathOperator{\pt}{pt}
\DeclareMathOperator{\rk}{rk}
\DeclareMathOperator{\coker}{coker}
\DeclareMathOperator{\Pf}{Pf}
\DeclareMathOperator{\Vol}{Vol}
\DeclareMathOperator{\Res}{Res}

\DeclareMathOperator{\GL}{GL}
\DeclareMathOperator{\SO}{SO}
\DeclareMathOperator{\UU}{U}

\newcommand\restr[2]{{% we make the whole thing an ordinary symbol
		\left.\kern-\nulldelimiterspace % automatically resize the bar with \right
		#1 % the function
		\vphantom{\big|} % pretend it's a little taller at normal size
		\right|_{#2} % this is the delimiter
}}

\title{Polyptych Isotropy Weights}

\date{}	% Here you can change the date presented in the paper title
%\date{} 					% Or removing it

%\author{
%  David S.~Hippocampus\thanks{Use footnote for providing further
%    information about author (webpage, alternative
%    address)---\emph{not} for acknowledging funding agencies.} \\
%  Department of Computer Science\\
%  Cranberry-Lemon University\\
%  Pittsburgh, PA 15213 \\
%  \texttt{hippo@cs.cranberry-lemon.edu} \\
%% examples of more authors
%   \And
% Elias D.~Striatum \\
%  Department of Electrical Engineering\\
%  Mount-Sheikh University\\
%  Santa Narimana, Levand \\
%  \texttt{stariate@ee.mount-sheikh.edu} \\
%% \AND
%% Coauthor \\
%% Affiliation \\
%% Address \\
%% \texttt{email} \\
%% \And
%% Coauthor \\
%% Affiliation \\
%% Address \\
%% \texttt{email} \\
%% \And
%% Coauthor \\
%% Affiliation \\
%% Address \\
%% \texttt{email} \\
%}

\begin{document}
	\maketitle
	
	\begin{abstract}
		Calculations for the isotropy data of the compactified hypertoric manifolds.
	\end{abstract}
	
	\section{Example: $M = T^{\ast}\CC\PP^{1}$}
	
	Short exact sequence for the usual Delzant construction of $\CC\PP^{1}$:
	\[
		\begin{tikzcd}
				\{1\} \arrow[rr] &  & K \cong S^{1} \arrow[rr, "{t \longmapsto (t,t)}", hook] &  & T^{2} \arrow[rr, "{(a,b) \longmapsto ab^{-1}}"', two heads] &  & T^{2}/K \cong T^{1} \arrow[rr] &  & \{1\}.
		\end{tikzcd}
	\]
	The induced action of $K$ on $T^{\ast}\CC^{2}$ is thus
	\[
		t \cdot (z|w) \longmapsto (t, t) \cdot (z_{1}, z_{2}\, |\, w_{1}, w_{2}) = (t z_{1}, t z_{2}\, |\, t^{-1} w_{1}, t^{-1} w_{2}),
	\]
	which is Hamiltonian with associated moment map
	\[
		\mu_{\RR} : T^{\ast}\CC \lra \RR; \qquad  \mu_{\RR}(z\, |\, w) = |z_{1}|^{2} + |z_{2}|^{2} - |w_{1}|^{2} - |w_{2}|^{2}.
	\]
	For some $a \in \ZZ_{>0}$, take the \HK quotient of $T^{\ast}\CC$ to get
	\[
		M := T^{\ast}\CC \sssslash K
	\]

	\section{Example: $M = T^{\ast}(\CC\PP^{2} \times \CC\PP^{2})$ (Non-Convex Core)}
	
	\subsection{Construction}
	
	Quotient relations arising from $K \cong T^{2}$:
	\[
		[sz_{1} : sz_{2} : sz_{3} : z_{4} \, | \, s^{-1}w_{1} : s^{-1}w_{2} : s^{-1}w_{3} : w_{4}] = [z_{1} : z_{2} : z_{3} : z_{4} \, | \, w_{1} : w_{2} : w_{3} : w_{4} ],
	\]
	and
	\[
		[z_{1} : tz_{2} : z_{3} : tz_{4} \, | \, w_{1} : t^{-1}w_{2} : w_{3} : t^{-1}w_{4}] = [z_{1} : z_{2} : z_{3} : z_{4} \, | \, w_{1} : w_{2} : w_{3} : w_{4} ].
	\]
	
	\subsection{Figure}
	
	\definecolor{ududff}{rgb}{0.30196078431372547,0.30196078431372547,1}
	\definecolor{aqaqaq}{rgb}{0.6274509803921569,0.6274509803921569,0.6274509803921569}
	\definecolor{ffffff}{rgb}{1,1,1}
	\definecolor{xdxdff}{rgb}{0.49019607843137253,0.49019607843137253,1}

	\scalebox{0.1}{
	\begin{tikzpicture}[line cap=round,line join=round,>=triangle 45,x=1cm,y=1cm]
		\clip(-80.91285845240186,-49.5404739373378) rectangle (93.8413205283568,54.72638210324665);
		\fill[line width=2pt,color=aqaqaq,fill=aqaqaq,fill opacity=1] (0,0) -- (20,0) -- (0,20) -- cycle;
		\fill[line width=2pt,color=aqaqaq,fill=aqaqaq,fill opacity=1] (0,20) -- (0,30) -- (-10,30) -- cycle;
		\fill[line width=2pt,fill=black,fill opacity=0.1] (20,30) -- (-30,30) -- (-20,40) -- (0,40) -- cycle;
		\fill[line width=2pt,fill=black,fill opacity=0.1] (0,30) -- (0,20) -- (50,-30) -- (50,0) -- (20,30) -- cycle;
		\fill[line width=2pt,fill=black,fill opacity=0.1] (-30,0) -- (20,0) -- (50,-30) -- (0,-30) -- cycle;
		\fill[line width=2pt,fill=black,fill opacity=0.1] (-30,30) -- (-10,30) -- (0,20) -- (0,0) -- (-30,0) -- cycle;
		\draw [line width=0.4pt,dash pattern=on 1pt off 1pt,color=ffffff,domain=-80.91285845240186:93.8413205283568] plot(\x,{(-0-0*\x)/20});
		\draw [line width=0.4pt,dash pattern=on 1pt off 1pt,color=ffffff,domain=-80.91285845240186:93.8413205283568] plot(\x,{(-400--20*\x)/-20});
		\draw [line width=0.4pt,dash pattern=on 1pt off 1pt,color=ffffff] (0,-49.5404739373378) -- (0,54.72638210324665);
		\draw [line width=0.4pt,dash pattern=on 1pt off 1pt,color=ffffff,domain=-80.91285845240186:93.8413205283568] plot(\x,{(-300-0*\x)/-10});
		\draw [line width=2pt,color=aqaqaq] (0,0)-- (20,0);
		\draw [line width=2pt,color=aqaqaq] (20,0)-- (0,20);
		\draw [line width=2pt,color=aqaqaq] (0,20)-- (0,0);
		\draw [line width=2pt,color=aqaqaq] (0,20)-- (0,30);
		\draw [line width=2pt,color=aqaqaq] (0,30)-- (-10,30);
		\draw [line width=2pt,color=aqaqaq] (-10,30)-- (0,20);
		\draw (1.0783302495220166,46.56909547218789) node[anchor=north west] {$Q_{14}^{(1)} = [0: z_{2}:0:0\, |\, 0:0:w_{3}:w_{4}]$};
		\draw (-60.62422247258906,44.79122530900842) node[anchor=north west] {$Q_{34}^{(3)} = [0: z_{2}:0:0\, |\, w_{1}:0:0:w_{4}]$};
		\draw (0.3462660646834106,31.19574759057714) node[anchor=north west] {$P_{14}$};
		\draw (-5.719408609693611,21.260590796338903) node[anchor=north west] {$P_{13}$};
		\draw (15.92877514196231,0.5536324251897347) node[anchor=north west] {$P_{23}$};
		\draw (-5.3010862183572645,0.3444712295215612) node[anchor=north west] {$P_{12}$};
		\draw (-14.817920621259143,30.568264003572622) node[anchor=north west] {$P_{34}$};
		\draw [line width=2pt] (0,-49.5404739373378) -- (0,54.72638210324665);
		\draw [line width=2pt,domain=-80.91285845240186:93.8413205283568] plot(\x,{(-0-0*\x)/-20});
		\draw [line width=2pt,domain=-80.91285845240186:93.8413205283568] plot(\x,{(-300-0*\x)/-10});
		\draw [line width=2pt,domain=-80.91285845240186:93.8413205283568] plot(\x,{(-200--10*\x)/-10});
		\draw [line width=2pt] (0,40)-- (20,30);
		\draw [line width=2pt] (20,30)-- (50,0);
		\draw [line width=2pt] (50,0)-- (50,-30);
		\draw [line width=2pt] (50,-30)-- (0,-30);
		\draw [line width=2pt] (0,-30)-- (-30,0);
		\draw [line width=2pt] (-30,0)-- (-30,30);
		\draw [line width=2pt] (-30,30)-- (-20,40);
		\draw [line width=2pt] (-20,40)-- (0,40);
		\draw (15.196710957123704,-17.74797219577544) node[anchor=north west] {$\Delta_{2}$};
		\draw (40.296054437304484,-8.649460184209897) node[anchor=north west] {$\Delta_{23}$};
		\draw (24.71354536002558,15.717819111132307) node[anchor=north west] {$\Delta_{3}$};
		\draw (4.215748184544614,36.32019688444739) node[anchor=north west] {$\Delta_{34}$};
		\draw (-10.94843850139794,39.66677601513817) node[anchor=north west] {$\Delta_{134}$};
		\draw (-21.72024007830886,36.73851927578374) node[anchor=north west] {$\Delta_{14}$};
		\draw (-22.138562469645205,16.031560904634567) node[anchor=north west] {$\Delta_{1}$};
		\draw (-12.307986273241065,-6.5578482275281615) node[anchor=north west] {$\Delta_{12}$};
		\draw (4.215748184544614,8.815499654082585) node[anchor=north west] {$\Delta_{\emptyset}$};
		\draw (-5.196505620523178,29.62703862306584) node[anchor=north west] {$\Delta_{13}$};
		\draw (19.58909606615534,37.57516405845643) node[anchor=north west] {$Q_{14}^{(4)} = [z_{1}: z_{2}:0:0\, |\, 0:0:w_{3}:0]$};
		\draw (49.60372764453819,7.665113077907631) node[anchor=north west] {$Q_{23}^{(2)} = [z_{1}: 0 :0:z_{4}\, |\, 0:0:w_{3}:0]$};
		\draw (50.649533622879055,-22.767840891811602) node[anchor=north west] {$Q_{23}^{(3)} = [z_{1}:0:0:z_{4}\, |\, 0:w_{2} :0:0]$};
		\draw (-37.616490949090014,-29.147257359690894) node[anchor=north west] {$Q_{12}^{(1)} = [0 : 0 : z_{3} : z_{4} \, | \, 0 : w_{2} : 0 : 0]$};
		\draw (-66.79447774480018,0.5536324251897347) node[anchor=north west] {$Q_{12}^{(2)} = [0 : 0 : z_{3} : z_{4} \, | \, w_{1} : 0 : 0 : 0]$};
		\draw (-67.63112252747287,30.98658639490897) node[anchor=north west] {$Q_{34}^{(4)} = [0 : z_{2} : z_{3} : 0 \, | \, w_{1} : 0 : 0 : 0]$};
		\draw [line width=2pt] (20,30)-- (-30,30);
		\draw [line width=2pt] (-30,30)-- (-20,40);
		\draw [line width=2pt] (-20,40)-- (0,40);
		\draw [line width=2pt] (0,40)-- (20,30);
		\draw [line width=2pt] (0,30)-- (0,20);
		\draw [line width=2pt] (0,20)-- (50,-30);
		\draw [line width=2pt] (50,-30)-- (50,0);
		\draw [line width=2pt] (50,0)-- (20,30);
		\draw [line width=2pt] (20,30)-- (0,30);
		\draw [line width=2pt] (-30,0)-- (20,0);
		\draw [line width=2pt] (20,0)-- (50,-30);
		\draw [line width=2pt] (50,-30)-- (0,-30);
		\draw [line width=2pt] (0,-30)-- (-30,0);
		\draw [line width=2pt] (-30,30)-- (-10,30);
		\draw [line width=2pt] (-10,30)-- (0,20);
		\draw [line width=2pt] (0,20)-- (0,0);
		\draw [line width=2pt] (0,0)-- (-30,0);
		\draw [line width=2pt] (-30,0)-- (-30,30);
		\begin{scriptsize}
			\draw [fill=xdxdff] (0,0) circle (2.5pt);
			\draw [fill=xdxdff] (20,0) circle (2.5pt);
			\draw [fill=xdxdff] (0,20) circle (2.5pt);
			\draw [fill=xdxdff] (0,30) circle (2.5pt);
			\draw [fill=xdxdff] (-10,30) circle (2.5pt);
			\draw [fill=xdxdff] (0,40) circle (2.5pt);
			\draw [fill=xdxdff] (20,30) circle (2.5pt);
			\draw [fill=xdxdff] (50,0) circle (2.5pt);
			\draw [fill=xdxdff] (50,-30) circle (2.5pt);
			\draw [fill=ududff] (0,-30) circle (2.5pt);
			\draw [fill=ududff] (-30,0) circle (2.5pt);
			\draw [fill=ududff] (-30,30) circle (2.5pt);
			\draw [fill=ududff] (-20,40) circle (2.5pt);
		\end{scriptsize}
	\end{tikzpicture}
}
	
	\subsection{Isotropy Weights}
	
	\begin{itemize}
		\item[$Q_{12}^{(1)}$:]
		\[
			Q_{12}^{(1)} = \left( [ 0 : 0 : z_{3} : z_{4} \, | \, 0 : w_{2} : 0 : 0 ], \xi \right), \text{ with } |w_{2}|^{2} = a,\, \xi = 0.
		\]
		has isotropy weights
		\begin{align*}
			\left( [sx_{1} : tx_{2} : z_{3} : z_{4} \, | \, s^{-1}y_{1} : t^{-1}w_{2} : y_{3} : y_{4}], \xi \right) &\sim \left( [sx_{1} : tx_{2} : z_{3} : z_{4} \, | \, s^{-1}ty_{1} : w_{2} : ty_{3} : ty_{4}], t\xi \right) \\ &\implies (sz_{1}, tz_{2}, s^{-1}tw_{1}, t\xi) \longleftrightarrow (s, t, s^{-1}t, t),
		\end{align*}
	
		so normal weights $(s, s^{-1}t)$ from $(z_{1}, w_{1})$ respectively, and inwards-pointing weight $t$ with multiplicity $2$ coming from $z_{2}$ and $\xi$, since $|w_{2}|$ achieves its maximum at $Q_{12}^{(1)}$.
	
		\item[$Q_{12}^{(2)}$:]
		\[
			Q_{12}^{(2)} = \left( [ 0 : 0 : z_{3} : z_{4} \, | \, w_{1} : 0 : 0 : 0 ], \xi \right)
		\]
		has isotropy weights
		\begin{align*}
			\left( [sx_{1} : tx_{2} : z_{3} : z_{4} \, | \, s^{-1}w_{1} : t^{-1}y_{2} : y_{3} : y_{4}], \xi \right) &\sim \left( [sx_{1} : tx_{2} : z_{3} : z_{4} \, | \, w_{1} : st^{-1}y_{2} : sy_{3} : sy_{4}], s\xi \right) \\ &\implies (sz_{1}, tz_{2}, s^{-1}tw_{1}, t\xi) \longleftrightarrow (s, t, st^{-1}, s).
		\end{align*}
	
		so normal weights $(t, st^{-1})$ from $(z_{2}, w_{2})$ respectively, and inwards-pointing weight $s$ with multiplicity $2$ coming from $z_{1}$ and $\xi$, since $|w_{1}|$ achieves its maximum at $Q_{12}^{(2)}$.
		
		\todo{}
	
		\item[$Q_{23}^{(2)}$:]
		\[
		Q_{23}^{(2)} = \left( [ z_{1} : 0 : 0 : z_{4} \, | \, 0 : 0 : w_{3} : 0 ], \xi \right)
		\]
		has isotropy weights
		\begin{align*}
			\left( [sz_{1} : tx_{2} : x_{3} : z_{4} \, | \, s^{-1}y_{1} : t^{-1}y_{2} : w_{3} : y_{4}], \xi \right) &\sim \left( [z_{1} : s^{-1}tx_{2} : s^{-1}x_{3} : z_{4} \, | \, y_{1} : st^{-1}y_{2} : sw_{3} : y_{4} ], \xi \right) \\ &\sim \left( [z_{1} : s^{-1}tx_{2} : s^{-1}x_{3} : z_{4} \, | \, s^{-1}y_{1} : t^{-1}y_{2} : w_{3} : s^{-1}y_{4} ], s^{-1}\xi \right) \\ &\implies (s^{-1}tz_{2}, s^{-1}w_{1}, t^{-1}w_{2}, s^{-1}\xi) \longleftrightarrow (s^{-1}t, s^{-1}, t^{-1}, s^{-1}).
		\end{align*}
	
		so normal weights $(s^{-1}t, t^{-1})$ from $(z_{2}, w_{2})$ respectively, and inwards-pointing weight $s^{-1}$ with multiplicity $2$ coming from $w_{1}$ and $\xi$, since $|z_{1}|$ achieves its maximum at $Q_{23}^{(2)}$.
	
		\item[$Q_{23}^{(3)}$:]
		\[
		Q_{23}^{(3)} = \left( [ z_{1} : 0 : 0 : z_{4} \, | \, 0 : w_{2} : 0 : 0 ], \xi \right)
		\]
		has isotropy weights
		\begin{align*}
			\left( [sz_{1} : tx_{2} : x_{3} : z_{4} \, | \, s^{-1}y_{1} : t^{-1}w_{2} : y_{3} : y_{4}], \xi \right) &\sim \left( [z_{1} : s^{-1}tx_{2} : s^{-1}x_{3} : z_{4} \, | \, y_{1} : st^{-1}w_{2} : sy_{3} : y_{4}], \xi \right) \\ &\sim \left( [z_{1} : s^{-1}tx_{2} : s^{-1}x_{3} : z_{4} \, | \, s^{-1}ty_{1} : w_{2} : ty_{3} : s^{-1}ty_{4}], s^{-1}t \xi \right) \\ &\implies (s^{-1}z_{3}, s^{-1}tw_{1}, tw_{3}, s^{-1}t\xi) \longleftrightarrow (s^{-1}, s^{-1}t, t, s^{-1}t).
		\end{align*}
	
		so normal weights $(s^{-1}, t)$ from $(z_{3}, w_{3})$ respectively, and inwards-pointing weight $s^{-1}t$ with multiplicity $2$ coming from $w_{1}$ and $\xi$, since $|z_{1}|$ and $|z_{4}|$ achieve their maximum at $Q_{23}^{(3)}$. (???)
	
		\item[$Q_{14}^{(4)}$:]
		\[
		Q_{14}^{(4)} = \left( [ z_{1} : z_{2} : 0 : 0 \, | \, 0 : 0 : w_{3} : 0 ], \xi \right)
		\]
		has isotropy weights
		\begin{align*}
			\left( [sz_{1} : tz_{2} : x_{3} : x_{4} \, | \, s^{-1}y_{1} : t^{-1}y_{2} : w_{3} : y_{4}], \xi \right) &\sim \left( [z_{1} : s^{-1}tz_{2} : s^{-1}x_{3} : x_{4} \, | \, y_{1} : st^{-1}y_{2} : sw_{3} : y_{4}], \xi \right) \\ &\sim \left( [z_{1} : z_{2} : s^{-1}x_{3} : st^{-1}x_{4} \, | \, y_{1} : y_{2} : sw_{3} : s^{-1}ty_{4}], \xi \right) \\ &\sim \left( [z_{1} : z_{2} : s^{-1}x_{3} : st^{-1}x_{4} \, | \, s^{-1}y_{1} : s^{-1}y_{2} : w_{3} : s^{-2}ty_{4}], s^{-1}\xi \right) \\ &\implies (st^{-1}z_{4}, s^{-1}w_{1}, s^{-2}tw_{4}, s^{-1}\xi) \longleftrightarrow (st^{-1}, s^{-1}, s^{-2}t, s^{-1}).
		\end{align*}
	
		so normal weights $(st^{-1}, s^{-2}t)$ from $(z_{4}, w_{4})$ respectively, and inwards-pointing weight $s^{-1}$ with multiplicity $2$ coming from $w_{1}$ and $\xi$, since $|z_{1}|$ achieves its maximum at $Q_{14}^{(4)}$.
	
	
	\end{itemize}
	

	
	
	
	
	
	
	
	
	
	
	
	
	
	
	
	
	
	
	
	
	
	
	\bibliographystyle{unsrt}  
	\bibliography{hypertoric}  %%% Remove comment to use the external .bib file (using bibtex).
	%%% and comment out the ``thebibliography'' section.
	
	%%% Comment out this section when you \bibliography{references} is enabled.
	
\end{document}
