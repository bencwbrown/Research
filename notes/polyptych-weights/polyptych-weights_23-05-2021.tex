\documentclass{article}


\usepackage{arxiv}
%\usepackage{ebgaramond}
\usepackage{garamondx}
% \usepackage{CormorantGaramond}

\usepackage[utf8]{inputenc} % allow utf-8 input
\usepackage[T1]{fontenc}    % use 8-bit T1 fonts
\usepackage{hyperref}       % hyperlinks
\usepackage{url}            % simple URL typesetting
\usepackage{booktabs}       % professional-quality tables
\usepackage{amsfonts}       % blackboard math symbols
\usepackage{nicefrac}       % compact symbols for 1/2, etc.
\usepackage{microtype}      % microtypography
% \usepackage{lipsum}		% Can be removed after putting your text content
\usepackage{amsmath} 
\usepackage{amssymb}
\usepackage{graphicx}
\usepackage{epstopdf}
\usepackage{url}
\usepackage{setspace}
\usepackage{amsthm}
\usepackage{mathrsfs}
\usepackage{enumitem}
\usepackage{parskip}
\usepackage{IEEEtrantools}
\usepackage{mathtools}
\usepackage{tensor}
\usepackage{yfonts}
\usepackage{dsfont}

%%%%%%%%%%%%%%%%%%% Custom packages
\usepackage{braket}
\usepackage{todo}
\usepackage{xargs}                      % Use more than one optional parameter in a new commands
\usepackage{tikz}
\usepackage{tikz-cd}

%%%%%%%%%%%%%%%%%%


\usepackage{pgfplots}
\pgfplotsset{compat=1.15}

\usetikzlibrary{arrows}



\newtheorem{theorem}{Theorem}[section]
\newtheorem{lemma}[theorem]{Lemma}
\newtheorem*{lemma*}{Lemma}
\newtheorem{prop}[theorem]{Proposition}
\newtheorem{corollary}[theorem]{Corollary}
\newtheorem{defn}[theorem]{Definition\rm}
\newtheorem{conjecture}[theorem]{Conjecture}
\newtheorem{remark}{\it Remark\/}
\newtheorem{example}{Example}
\newtheorem{fact}{Fact}

\newcommand{\st}{\ensuremath{:}}% such that
\newcommand{\ie}{\emph{i.e.} }
\newcommand{\eg}{\emph{e.g.} }
\newcommand{\cf}{\emph{cf.} }
\newcommand{\ra}{\rightarrow}
\newcommand{\la}{\leftarrow}
\newcommand{\lra}{\longrightarrow}
\newcommand{\lla}{\longleftarrow}
\newcommand{\lbracket}{\left(}
\newcommand{\rbracket}{\right)}

\newcommand{\al}{\alpha}
\newcommand{\w}{\omega}
\newcommand{\W}{\Omega}
\newcommand{\m}{\mu}
\newcommand{\n}{\nu}
\newcommand{\e}{\epsilon}
\newcommand{\K}{K\"ahler }
\newcommand{\HK}{hyperk\"ahler }
\newcommand{\into}{\hookrightarrow}
\newcommand{\PP}{\mathbb{P}}
\newcommand{\RR}{\mathbb{R}}
\newcommand{\CC}{\mathbb{C}}
\newcommand{\QQ}{\mathbb{Q}}
\newcommand{\FF}{\mathbb{F}}
\newcommand{\ZZ}{\mathbb{Z}}
\newcommand{\NN}{\mathbb{N}}
\newcommand{\HH}{\mathbb{H}}
\newcommand{\vp}{\varphi}
\newcommand{\mcA}{\mathcal{A}}
\newcommand{\mcE}{\mathcal{E}}
\newcommand{\mcF}{\mathcal{F}}
\newcommand{\mcG}{\mathcal{G}}
\newcommand{\mcH}{\mathcal{H}}
\newcommand{\mcL}{\mathcal{L}}
\newcommand{\mcO}{\mathcal{O}}
\newcommand{\mfg}{\mathfrak{g}}
\newcommand{\mfh}{\mathfrak{h}}
\newcommand{\mft}{\mathfrak{t}}
\newcommand{\mc}[1]{\mathcal{#1}}
\newcommand{\mf}[1]{\mathfrak{#1}}

\newcommand{\sslash}{\mathbin{/\mkern-6mu/}}
\newcommand{\sssslash}{\mathbin{/\mkern-6mu/\mkern-6mu/\mkern-6mu/}}

\newcommand{\pbrackets}[1]{\left( #1 \right)}
\newcommand{\bbrackets}[1]{\left[ #1 \right]}
\newcommand{\norm}[1]{|#1|^{2}}

\newcommand{\dbar}{\bar{\partial}}
\newcommand{\mrr}{\mu_{\mathbb{R}}}
\newcommand{\mcc}{\mu_{\mathbb{C}}}
\newcommand{\prr}{\phi_{\mathbb{R}}}
\newcommand{\pcc}{\phi_{\mathbb{C}}}

\DeclareMathOperator{\Lie}{Lie}
\DeclareMathOperator{\Aut}{Aut}
\DeclareMathOperator{\Tr}{Tr}
\DeclareMathOperator{\Image}{Im}
\DeclareMathOperator{\Ad}{Ad}
\DeclareMathOperator{\Diff}{Diff}
\DeclareMathOperator{\Vect}{Vect}
\DeclareMathOperator{\Sympl}{Sympl}
\DeclareMathOperator{\Span}{Span}
\DeclareMathOperator{\ind}{ind}
\DeclareMathOperator{\Td}{Td}
\DeclareMathOperator{\Ch}{Ch}
\DeclareMathOperator{\Ind}{Ind}
\DeclareMathOperator{\pt}{pt}
\DeclareMathOperator{\rk}{rk}
\DeclareMathOperator{\coker}{coker}
\DeclareMathOperator{\Pf}{Pf}
\DeclareMathOperator{\Vol}{Vol}
\DeclareMathOperator{\Res}{Res}

\DeclareMathOperator{\GL}{GL}
\DeclareMathOperator{\SO}{SO}
\DeclareMathOperator{\UU}{U}

\newcommand\restr[2]{{% we make the whole thing an ordinary symbol
		\left.\kern-\nulldelimiterspace % automatically resize the bar with \right
		#1 % the function
		\vphantom{\big|} % pretend it's a little taller at normal size
		\right|_{#2} % this is the delimiter
}}

\title{Polyptych Isotropy Weights}

\date{}	% Here you can change the date presented in the paper title
%\date{} 					% Or removing it

%\author{
%  David S.~Hippocampus\thanks{Use footnote for providing further
%    information about author (webpage, alternative
%    address)---\emph{not} for acknowledging funding agencies.} \\
%  Department of Computer Science\\
%  Cranberry-Lemon University\\
%  Pittsburgh, PA 15213 \\
%  \texttt{hippo@cs.cranberry-lemon.edu} \\
%% examples of more authors
%   \And
% Elias D.~Striatum \\
%  Department of Electrical Engineering\\
%  Mount-Sheikh University\\
%  Santa Narimana, Levand \\
%  \texttt{stariate@ee.mount-sheikh.edu} \\
%% \AND
%% Coauthor \\
%% Affiliation \\
%% Address \\
%% \texttt{email} \\
%% \And
%% Coauthor \\
%% Affiliation \\
%% Address \\
%% \texttt{email} \\
%% \And
%% Coauthor \\
%% Affiliation \\
%% Address \\
%% \texttt{email} \\
%}

\begin{document}
	\maketitle
	
	\begin{abstract}
		Calculations for the isotropy data of the compactified hypertoric manifolds.
	\end{abstract}
	
	\section{Example: $M = T^{\ast}\CC\PP^{1}$}
	
	\subsection{Construction}
	
	Short exact sequence for the usual Delzant construction of $\CC\PP^{1}$:
	\[
		\begin{tikzcd}
				\{1\} \arrow[rr] &  & K \cong S^{1} \arrow[rr, "{t \longmapsto (t,t)}", hook] &  & T^{2} \arrow[rr, "{(a,b) \longmapsto ab^{-1}}"', two heads] &  & T^{2}/K \cong T^{1} \arrow[rr] &  & \{1\}.
		\end{tikzcd}
	\]
	The induced action of $K$ on $T^{\ast}\CC^{2}$ is thus
	\[
		t \cdot (z|w) \longmapsto (t, t) \cdot (z_{1}, z_{2}\, |\, w_{1}, w_{2}) = (t z_{1}, t z_{2}\, |\, t^{-1} w_{1}, t^{-1} w_{2}),
	\]
	which is Hamiltonian with associated moment map
	\[
		\mu_{\RR} : T^{\ast}\CC \lra \RR; \qquad  \mu_{\RR}(z\, |\, w) = |z_{1}|^{2} + |z_{2}|^{2} - |w_{1}|^{2} - |w_{2}|^{2}.
	\]
	For some $a \in \ZZ_{>0}$, take the \HK quotient of $T^{\ast}\CC$ to get
	\[
		M := T^{\ast}\CC \sssslash K
	\]

	\section{Example: $M = T^{\ast}(\CC\PP^{2} \times \CC\PP^{2})$ (Non-Convex Core)}
	
	\subsection{Construction}
	
	Short exact sequence for the usual Delzant construction of the non-convex $T^{\ast}(\CC\PP^{2} \times \CC\PP^{2})$:
	\[
	\begin{tikzcd}
		\{1\} \arrow[rr] &  & K \cong T^{2} \arrow[rr, "{(s,t) \longmapsto (s,st,s,t)}", hook] &  & T^{4} \arrow[rr, "{(a,b,c,d) \longmapsto (ac^{-1}, bc^{-1}d^{-1})}"', two heads] &  & T^{4}/K \cong T^{2} \arrow[rr] &  & \{1\}.
	\end{tikzcd}
	\]
	The induced action of $K$ on $T^{\ast}\CC^{4}$ is thus
	\[
	(s,t) \cdot (z|w) \longmapsto (s,st,s,t) \cdot (z_{1}, z_{2}, z_{3}, z_{4}\, |\, w_{1}, w_{2}, w_{3}, w_{4}) = (s z_{1}, st z_{2}, sz_{3}, tz_{4} \, |\, s^{-1} w_{1}, s^{-1}t^{-1} w_{2}, s^{-1}w_{3}, t^{-1}w_{4}),
	\]
	which is Hamiltonian with associated moment map
	\[
	\mu_{\RR} : T^{\ast}\CC^{4} \lra \RR^{2}; \qquad  \mu_{\RR}(z\, |\, w) = \begin{pmatrix}
		|z_{1}|^{2} + |z_{2}|^{2} + |z_{3}|^{2} - |w_{1}|^{2} - |w_{2}|^{2} - |w_{3}|^{2} \\ |z_{2}|^{2} + |z_{4}|^{2} - |w_{2}|^{2} - |w_{4}|^{2}
	\end{pmatrix}.
	\]
	For some $(n,m) \in \ZZ_{>0}^{2}$, take the \HK quotient of $T^{\ast}\CC^{4}$ to get
	\[
	M := T^{\ast}\CC^{4} \sssslash_{(n,m)} K := \lbracket \mu_{\RR}^{-1}(n,m) \cap \mu_{\CC}^{-1}(0) \rbracket / K.
	\]
	
	Quotient relations arising from $K \cong T^{2}$:
	
	\begin{itemize}
		\item[$(s,1) \in S^{1} \times \{e\} < T^{2}$:]
		
	\[
		[sz_{1} : sz_{2} : sz_{3} : z_{4} \, | \, s^{-1}w_{1} : s^{-1}w_{2} : s^{-1}w_{3} : w_{4}] = [z_{1} : z_{2} : z_{3} : z_{4} \, | \, w_{1} : w_{2} : w_{3} : w_{4} ],
	\]
	
		\item[$(1,t) \in \{e\} \times S^{1} < T^{2}$:]
		
	\[
		[z_{1} : tz_{2} : z_{3} : tz_{4} \, | \, w_{1} : t^{-1}w_{2} : w_{3} : t^{-1}w_{4}] = [z_{1} : z_{2} : z_{3} : z_{4} \, | \, w_{1} : w_{2} : w_{3} : w_{4} ].
	\]
	\end{itemize}
	
	\subsection{Figure}
	
	\subsection{Isotropy Weights}
	
	\subsubsection{Interior Points}
	
	\begin{itemize}
		\item[$P_{12}$:]
		\[
		P_{12} = [ 0 : 0 : z_{3} : z_{4} \, | \, 0 : 0: 0 : 0 ],
		\]
		has isotropy weights
		\begin{align*}
			[sx_{1} : tx_{2} : z_{3} : z_{4} \, | \, s^{-1}y_{1} : t^{-1}y_{2} : y_{3} : y_{4}] & \\ &\implies (sx_{1}, tx_{2}, s^{-1}y_{1}, t^{-1}y_{2}) \longleftrightarrow (s, t, s^{-1}, t^{-1}),
		\end{align*}
		
		so tangent space weights $(s, t)$ from $(z_{1}, z_{2})$ respectively, and cotangent space weights $(s^{-1}, t^{-1})$ coming from $(w_{1}, w_{2})$ respectively.
		
		\item[$P_{23}$:]
		\[
		P_{23} = [ z_{1} : 0 : 0 : z_{4} \, | \, 0 : 0: 0 : 0 ],
		\]
		has isotropy weights
		\begin{align*}
			[sz_{1} : tx_{2} : x_{3} : z_{4} \, | \, s^{-1}y_{1} : t^{-1}y_{2} : y_{3} : y_{4}] &\sim [z_{1} : s^{-1}tx_{2} : s^{-1}x_{3} : z_{4} \, | \, y_{1} : st^{-1}y_{2} : sy_{3} : y_{4}]  \\ &\implies (s^{-1}tx_{2}, s^{-1}x_{3}, st^{-1}y_{2}, sy_{3}) \longleftrightarrow (s^{-1}t, s^{-1}, st^{-1}, s),
		\end{align*}
		
		so tangent space weights $(s^{-1}t,s^{-1})$ from $(z_{2}, z_{3})$ respectively, and cotangent space weights $(st^{-1}, s)$ coming from $(w_{2}, w_{3})$ respectively.
		
		\item[$P_{13}$:]
		\[
		P_{13} = [ 0 : z_{2} : 0 : z_{4} \, | \, 0 : 0: 0 : 0 ],
		\]
		has isotropy weights
		\begin{align*}
			[sx_{1} : tz_{2} : x_{3} : z_{4} \, | \, s^{-1}y_{1} : t^{-1}y_{2} : y_{3} : y_{4}] &\sim [st^{-1}x_{1} : z_{2} : t^{-1}x_{3} : z_{4} \, | \, s^{-1}ty_{1} : y_{2} :t y_{3} : y_{4}] \\ &\implies (st^{-1}x_{1}, t^{-1}x_{3}, s^{-1}ty_{1}, ty_{3}) \longleftrightarrow (st^{-1}, t^{-1}, s^{-1}t, t),
		\end{align*}
		
		so tangent space weights $(st^{-1},t^{-1})$ from $(z_{1}, z_{3})$ respectively, and cotangent space weights $(s^{-1}t, t)$ coming from $(w_{1}, w_{3})$ respectively.
		
		\item[$P_{14}$:]
		\[
		P_{14} = [ 0 : z_{2} : 0 : 0 \, | \, 0 : 0: w_{3} : 0 ],
		\]
		has isotropy weights
		\begin{align*}
			[sx_{1} : tz_{2} : x_{3} : z_{4} \, | \, s^{-1}y_{1} : t^{-1}y_{2} : w_{3} : y_{4}] &\sim [sx_{1} : z_{2} : x_{3} : t^{-1}x_{4} \, | \, s^{-1}y_{1} : y_{2} :w_{3} : ty_{4}] \\ &\implies (sx_{1}, t^{-1}x_{4}, s^{-1}y_{1}, ty_{4}) \longleftrightarrow (s, t^{-1}, s^{-1}, t),
		\end{align*}
		
		so tangent space weights $(s^{-1},t^{-1})$ from $(w_{1}, z_{4})$ respectively, and cotangent space weights $(s,t)$ coming from $(z_{1}, w_{4})$ respectively.
		
		\item[$P_{34}$:]
		\[
		P_{34} = [ 0 : z_{2} : 0 : 0 \, | \, w_{1} : 0: 0 : 0 ],
		\]
		has isotropy weights
		\begin{align*}
			[sx_{1} : tz_{2} : x_{3} : x_{4} \, | \, s^{-1}w_{1} : t^{-1}y_{2} : y_{3} : y_{4}] &\sim [x_{1} : s^{-1}tz_{2} : s^{-1}x_{3} : x_{4} \, | \, w_{1} : st^{-1}y_{2} : sy_{3} : y_{4}] \\ &\sim [x_{1} : z_{2} : s^{-1}x_{3} : st^{-1}x_{4} \, | \, w_{1} : y_{2} : sy_{3} : s^{-1}ty_{4}] \\ &\implies (s^{-1}x_{3}, st^{-1}x_{4}, sy_{3}, s^{-1}ty_{4}) \longleftrightarrow (s^{-1}, st^{-1}, s, s^{-1}t),
		\end{align*}
		
		so tangent space weights $(st^{-1}, s)$ from $(z_{4}, w_{3})$ respectively, and cotangent space weights $(s^{-1}, s^{-1}t)$ coming from $(z_{3}, w_{4})$ respectively.
		
	\end{itemize}
	
	
	\subsubsection{Exterior Points}
	
	\begin{itemize}
		\item[$Q_{12}^{(1)}$:]
		Locally near $Q_{12}^{(1)}$, $S^{1}_{A}$ acts as $(\tau, \tau, 1, 1)$.
		\[
			Q_{12}^{(1)} = \left( [ 0 : 0 : z_{3} : z_{4} \, | \, 0 : w_{2} : 0 : 0 ], \xi \right), \text{ with } |w_{2}|^{2} = a,\, \xi = 0.
		\]
		has isotropy weights
		\begin{align*}
			\left( [sx_{1} : tx_{2} : z_{3} : z_{4} \, | \, s^{-1}y_{1} : t^{-1}w_{2}], \xi \right) &\sim \left( [sx_{1} : tx_{2} : z_{3} : z_{4} \, | \, s^{-1}ty_{1} : w_{2}], t\xi \right) \\ &\implies (sz_{1}, tz_{2}, s^{-1}tw_{1}, t\xi) \longleftrightarrow (s, t, s^{-1}t, t),
		\end{align*}
	
		so normal weights $(s, s^{-1}t)$ from $(z_{1}, w_{1})$ respectively, and inwards-pointing weight $t$ with multiplicity $2$ coming from $z_{2}$ and $\xi$, since $|w_{2}|$ achieves its maximum at $Q_{12}^{(1)}$.
	
		\item[$Q_{12}^{(2)}$:]
		Locally near $Q_{12}^{(2)}$, $S^{1}_{A}$ acts as $(\tau, \tau, 1, 1)$.
		\[
			Q_{12}^{(2)} = \left( [ 0 : 0 : z_{3} : z_{4} \, | \, w_{1} : 0 : 0 : 0 ], \xi \right)
		\]
		has isotropy weights
		\begin{align*}
			\left( [sx_{1} : tx_{2} : z_{3} : z_{4} \, | \, s^{-1}w_{1} : t^{-1}y_{2}], \xi \right) &\sim \left( [sx_{1} : tx_{2} : z_{3} : z_{4} \, | \, w_{1} : st^{-1}y_{2} ], s\xi \right) \\ &\implies (sz_{1}, tz_{2}, st^{-1}w_{2}, s\xi) \longleftrightarrow (s, t, st^{-1}, s).
		\end{align*}
	
		so normal weights $(t, st^{-1})$ from $(z_{2}, w_{2})$ respectively, and inwards-pointing weight $s$ with multiplicity $2$ coming from $z_{1}$ and $\xi$, since $|w_{1}|$ achieves its maximum at $Q_{12}^{(2)}$.
	
		\item[$Q_{23}^{(2)}$:]
		Locally near $Q_{23}^{(2)}$, $S^{1}_{A}$ acts as $(1, \tau, \tau, 1)$.
		\[
		Q_{23}^{(2)} = \left( [ z_{1} : 0 : 0 : z_{4} \, | \, 0 : 0 : w_{3} : 0 ], \xi \right)
		\]
		has isotropy weights
		\begin{align*}
			\left( [sz_{1} : tx_{2} : x_{3} : z_{4} \, | \, s^{-1}y_{1} : t^{-1}y_{2} : w_{3} : y_{4}], \xi \right) &\sim \left( [z_{1} : s^{-1}tx_{2} : s^{-1}x_{3} : z_{4} \, | \, y_{1} : st^{-1}y_{2} : sw_{3} : y_{4} ], \xi \right) \\ &\sim \left( [z_{1} : s^{-1}tx_{2} : s^{-1}x_{3} : z_{4} \, | \, s^{-1}y_{1} : t^{-1}y_{2} : w_{3} : s^{-1}y_{4} ], s^{-1}\xi \right) \\ &\implies (s^{-1}tz_{2}, s^{-1}z_{3}, t^{-1}w_{2}, s^{-1}\xi) \longleftrightarrow (s^{-1}t, s^{-1}, t^{-1}, s^{-1}).
		\end{align*}
	
		so normal weights $(s^{-1}t, t^{-1})$ from $(z_{2}, w_{2})$ respectively, and inwards-pointing weight $s^{-1}$ with multiplicity $2$ coming from $z_{3}$ and $\xi$, since $|z_{1}|$ achieves its maximum at $Q_{23}^{(2)}$.
	
		\item[$Q_{23}^{(3)}$:]
		\[
		Q_{23}^{(3)} = \left( [ z_{1} : 0 : 0 : z_{4} \, | \, 0 : w_{2} : 0 : 0 ], \xi \right)
		\]
		has isotropy weights
		\begin{align*}
			\left( [sz_{1} : tx_{2} : x_{3} : z_{4} \, | \, s^{-1}y_{1} : t^{-1}w_{2} : y_{3} : y_{4}], \xi \right) &\sim \left( [z_{1} : s^{-1}tx_{2} : s^{-1}x_{3} : z_{4} \, | \, y_{1} : st^{-1}w_{2} : sy_{3} : y_{4}], \xi \right) \\ &\sim \left( [z_{1} : s^{-1}tx_{2} : s^{-1}x_{3} : z_{4} \, | \, s^{-1}ty_{1} : w_{2} : ty_{3} : s^{-1}ty_{4}], s^{-1}t \xi \right) \\ &\implies (s^{-1}z_{3}, s^{-1}tw_{1}, tw_{3}, s^{-1}t\xi) \longleftrightarrow (s^{-1}, s^{-1}t, t, s^{-1}t).
		\end{align*}
	
		so normal weights $(s^{-1}, t)$ from $(z_{3}, w_{3})$ respectively, and inwards-pointing weight $s^{-1}t$ with multiplicity $2$ coming from $w_{1}$ and $\xi$, since $|z_{1}|$ and $|z_{4}|$ achieve their maximum at $Q_{23}^{(3)}$. (???)
	
		\item[$Q_{14}^{(4)}$:]
		\[
		Q_{14}^{(4)} = \left( [ z_{1} : z_{2} : 0 : 0 \, | \, 0 : 0 : w_{3} : 0 ], \xi \right)
		\]
		has isotropy weights
		\begin{align*}
			\left( [sz_{1} : tz_{2} : x_{3} : x_{4} \, | \, s^{-1}y_{1} : t^{-1}y_{2} : w_{3} : y_{4}], \xi \right) &\sim \left( [z_{1} : s^{-1}tz_{2} : s^{-1}x_{3} : x_{4} \, | \, y_{1} : st^{-1}y_{2} : sw_{3} : y_{4}], \xi \right) \\ &\sim \left( [z_{1} : z_{2} : s^{-1}x_{3} : st^{-1}x_{4} \, | \, y_{1} : y_{2} : sw_{3} : s^{-1}ty_{4}], \xi \right) \\ &\sim \left( [z_{1} : z_{2} : s^{-1}x_{3} : st^{-1}x_{4} \, | \, s^{-1}y_{1} : s^{-1}y_{2} : w_{3} : s^{-2}ty_{4}], s^{-1}\xi \right) \\ &\implies (st^{-1}z_{4}, s^{-1}w_{1}, s^{-2}tw_{4}, s^{-1}\xi) \longleftrightarrow (st^{-1}, s^{-1}, s^{-2}t, s^{-1}).
		\end{align*}
	
		so normal weights $(st^{-1}, s^{-2}t)$ from $(z_{4}, w_{4})$ respectively, and inwards-pointing weight $s^{-1}$ with multiplicity $2$ coming from $w_{1}$ and $\xi$, since $|z_{1}|$ achieves its maximum at $Q_{14}^{(4)}$.
		
		\item[$Q_{34}^{(4)}$:]
		\[
			Q_{34}^{(4)} = \left( [ 0 : z_{2} : z_{3} : 0 \, | \, w_{1} : 0 : 0 : 0 ], \xi \right)
		\]
		has isotropy weights
		\begin{align*}
			\left( [sx_{1} : tz_{2} : z_{3} : x_{4} \, | \, s^{-1}w_{1} : t^{-1}y_{2} : y_{3} : y_{4}], \xi \right) &\sim \left( [sx_{1} : z_{2} : z_{3} : t^{-1}x_{4} \, | \, s^{-1}w_{1} : y_{2} : y_{3} : ty_{4}], \xi \right) \\ &\sim \left( [sx_{1} : z_{2} : z_{3} : t^{-1}x_{4} \, | \, w_{1} : sy_{2} : sy_{3} : sty_{4}], s\xi \right) \\ &\implies (sz_{1}, t^{-1}z_{4}, stw_{4}, s\xi) \longleftrightarrow (s, t^{-1}, st, s).
		\end{align*}
		
		so normal weights $(t^{-1}, st)$ from $(z_{4}, w_{4})$ respectively, and inwards-pointing weight $s$ with multiplicity $2$ coming from $z_{1}$ and $\xi$, since $|w_{1}|$ achieves its maximum at $Q_{34}^{(4)}$.
	
		\item[$Q_{34}^{(3)}$:]
		\[
			Q_{34}^{(3)} = \left( [ 0 : z_{2} : 0 : 0 \, | \, w_{1} : 0 : 0 : w_{4} ], \xi \right)
		\]
		has isotropy weights
		\begin{align*}
			& \left( [sx_{1} : tz_{2} : x_{3} : x_{4} \, | \, s^{-1}w_{1} : t^{-1}y_{2} : y_{3} : w_{4}], \xi \right) \\ \sim &\left( [x_{1} : s^{-1}tz_{2} : s^{-1}x_{3} : x_{4} \, | \, w_{1} : st^{-1}y_{2} : sy_{3} : w_{4}], \xi \right) \\ \sim &\left( [ s^{1/2}t^{-1/2}x_{1} : z_{2} : s^{-1/2}t^{1/2}x_{3} : s^{1/2}t^{-1/2}x_{4} \, | \, w_{1} : s^{1/2}t^{-1/2}y_{2} : sy_{3} : w_{4}], \xi \right) \\ \implies &(s^{-1/2}t^{-1/2}z_{3}, s^{1/2}t^{-1/2}z_{4}, sw_{3}, s^{1/2}t^{-1/2} \xi) \longleftrightarrow (s^{-1/2}t^{-1/2}, s^{1/2}t^{-1/2}, s, s^{1/2}t^{-1/2}).
		\end{align*}
		
		so normal weights $(s^{-1/2}t^{-1/2})$ from $(z_{3}, w_{3})$ respectively, and inwards-pointing weight $s^{1/2}t^{-1/2}$ with multiplicity $2$ coming from $z_{4}$ and $\xi$, since $|w_{4}|$ achieves its maximum at $Q_{34}^{(3)}$.
	
		\item[$Q_{14}^{(1)}$:]
		\[
		Q_{14}^{(1)} = \left( [ 0 : z_{2} : 0 : 0 \, | \, 0 : 0 : w_{3} : w_{4} ], \xi \right)
		\]
		has isotropy weights
		\begin{align*}
			& \left( [sx_{1} : tz_{2} : x_{3} : x_{4} \, | \, s^{-1}y_{1} : t^{-1}y_{2} : w_{3} : w_{4}], \xi \right) \\ \sim & \left( [st^{-1/2}x_{1} : z_{2} : t^{-1/2}x_{3} : t^{-1/2}x_{4} \, | \, s^{-1}y_{1} : t^{-1/2}y_{2} : w_{3} : w_{4}], t^{-1/2}\xi \right) \\ \implies &(st^{-1/2}z_{1}, t^{-1/2}z_{4}, s^{-1}w_{1}, t^{-1/2}\xi) \longleftrightarrow (st^{-1/2}, t^{-1/2}, s^{-1}, t^{-1/2}).
		\end{align*}
		
		so normal weights $(st^{-1/2}, s^{-1})$ from $(z_{1}, w_{1})$ respectively, and inwards-pointing weight $t^{-1/2}$ with multiplicity $2$ coming from $z_{4}$ and $\xi$, since $|w_{4}|$ achieves its maximum at $Q_{14}^{(1)}$.
		
	\end{itemize}

	\begin{remark}
		For the exterior fixed-points $Q_{34}^{(3)}$ and $Q_{14}^{(1)}$, where there are two non-zero cotangent $w$ coordinates, the relation
		\begin{align*}
			\left( [\tau z_{1} : \tau^{2} z_{2} : \tau z_{3} : \tau z_{4} \, | \, w_{1} : w_{2} : w_{3} : w_{4}], \xi \right) &\sim	\left( [z_{1} : \tau z_{2} : z_{3} : \tau z_{4} \, | \, \tau w_{1} : \tau w_{2} : \tau w_{3} : w_{4}], \xi \right) \\ &\sim \left( [z_{1} : z_{2} : z_{3} : z_{4} \, | \, \tau w_{1} : \tau^{2} w_{2} : \tau w_{3} : \tau w_{4}], \xi \right) \\ &\sim \left( [z_{1} : z_{2} : z_{3} : z_{4} \, | \, w_{1} : \tau w_{2} : w_{3} : w_{4}], \tau^{-1} \xi \right),
		\end{align*}
		is used.
	\end{remark}
	

	
	
	
	
	
	
	
	
	
	
	
	
	
	
	
	
	
	
	
	
	
	
	\bibliographystyle{unsrt}  
	\bibliography{hypertoric}  %%% Remove comment to use the external .bib file (using bibtex).
	%%% and comment out the ``thebibliography'' section.
	
	%%% Comment out this section when you \bibliography{references} is enabled.
	
\end{document}
